%Formattazione
%Scelgo report al posto di article per poter dividere l'elaborato in 3 capitoli
\documentclass[a4paper,10pt]{article}
\usepackage[utf8]{inputenc}
\usepackage{palatino}

%Pacchetti che avevo già in altri file
%\usepackage{ucs}
\usepackage[british]{babel}
%Per i circuiti quantistici
%\usepackage[braket, qm]{qcircuit}
\usepackage{amsmath}
\usepackage{amssymb}
\usepackage{amsthm}
\usepackage{physics}
%Mi serve solo per la freccia mapsfrom:
\usepackage{stmaryrd}

%Pacchetti aggiunti dall'internet:
%Questo Pacchetto serve per modificare le impostazioni del line spacing inline
\usepackage{setspace}
\usepackage{graphicx}
%Per centrare il circuito di QFT (Chapter_2)
\usepackage{changepage}
%Per le liste nelle descrizioni degli algoritmi
\usepackage{enumitem}
%Per la sfera di Bloch
\usepackage{tikz}
%Per i diagrammini commutativi
\usetikzlibrary{cd}
\usetikzlibrary{arrows.meta}
\usetikzlibrary{decorations.pathmorphing} %For squiggly lines
% Per sistemare il simbolo di misurazione nel testo
\usepackage{adjustbox}

%Per le appendici
\usepackage[title]{appendix}

%Per il Frontespizio
\usepackage{tabularx}
\usepackage{geometry}

%Dal Darione
\usepackage{layaureo}

\usepackage{mathtools}
\setlist[description]{leftmargin=3.2em,labelindent=3.2em}

%Comandi miei
%Sillabazione
%\hyphenation{distinto}
\newcommand{\R}{\mathbb{R}}
\renewcommand{\P}{\mathbb{P}}
\newcommand{\C}[1]{\mathbb{C}^{#1}}
\newcommand{\Z}{\mathbb{Z}}
\newcommand{\N}{\mathbb{N}}
\newcommand{\T}{\mathbb{T}}
\newcommand{\K}{\mathbb{K}}
%New useful operators:
\DeclareMathOperator{\sign}{sign}
\DeclareMathOperator{\ima}{im}
\DeclareMathOperator{\fct}{Fct}

\newtheorem{post}{Postulate}
\newtheorem{thm}{Theorem}[section]
\newtheorem{cor}[thm]{Corollary}
\newtheorem{lem}[thm]{Lemma}
\newtheorem{prop}[thm]{Proposition}

\newtheoremstyle{algoritmo}
{1.7em}%〈Space above〉
{1em}%〈Space below〉
{}%〈Body font〉
{}%〈Indent amount〉
{\bfseries}%〈Theorem head font〉
{}%〈Punctuation after theorem head〉
{.5em}%〈Space after theorem head〉
{\thmname{#1}\thmnumber{ #2}\thmnote{: #3.}}%〈Theorem head spec(can be left empty, meaning ‘normal’)〉

\newtheoremstyle{definizione}
{1.0em}
{0.7em}
{}
{}
{\bfseries}
{}
{1.0em}
{\thmname{#1}\thmnumber{ #2}\thmnote{: #3. \\*}}


\theoremstyle{algoritmo}
\newtheorem*{algo}{Algoritmo}
\newtheorem*{ex}{Example}

\theoremstyle{definizione}
\newtheorem{defn}[thm]{Definition}%[section]
\newtheorem{constr}[thm]{Construction}%[section]
%\theoremstyle{definizione}
\newtheorem{rem}{Remark}

\numberwithin{equation}{section}

%Simbolo QED
\renewcommand{\qedsymbol}{\ensuremath{\blacksquare}}
\usepackage{hyperref}

\begin{document}

\tableofcontents
\section{Affine Varieties}
\begin{defn}[Affine variety]
	We define an \textbf{affine variety} to be the solution set of systems of polynomial equations.
\end{defn}
\begin{rem}
	We will work over a fixed algebraically closed field.
	For example we will work over $\C{}$ or over the algebraic closure of the finite fields $\mathbb{F}_q$, denoted with $\overline{\mathbb{F}}_q$.
\end{rem}

\begin{defn}[Affine space]
	We denote with $\mathbb{A}^n(\mathbb{K}) = \mathbb{A}^n$ the \textbf{affine} $n$-space:
	\begin{equation}
		\mathbb{A}^n(\mathbb{K}) := \left\{ \left(a_1, a_2, \ldots, a_n\right) \in \mathbb{K}^n \right\}
	.\end{equation} 
\end{defn}

\begin{defn}[Polynomial in n indeterminates]
	A \textbf{polynomial} with coefficients in $\mathbb{K}$ in indeterminates $x_1, x_2, \ldots, x_n$ is an expression of the form
	\begin{equation}
		f \left( x_1, x_2, \ldots, x_n \right) = \sum_{I = \left( i_1, \ldots, i_n \right) \in \mathbb{N}^n}^{} c_I\, x_i^{i_1}\cdot \ldots \cdot x_n^{i_n}
	,\end{equation} 
	where $c_I \in \mathbb{K}$ for all $I$ and it is zero for all but finitely many choices of $I$.
\end{defn}

\begin{defn}[Polynomial ring]
	We define the set
	\begin{equation}
		\mathbb{K}[x_1, \ldots, x_n] := \left\{ \text{polynomials in } x_1, \ldots, x_n \text{ with coefficients in } \mathbb{K} \right\}
	.\end{equation} 
\end{defn}

\begin{defn}[(affine) Algebraic set]
	Let $S \subset \mathbb{K}[x_1, \ldots, x_n]$ a set of polynomials.
	We define the \textbf{zero set} (or \textbf{vanishing locus}) of $S$ as
	\begin{equation}
		\mathbb{V}(S) = Z(S) := \left\{ p \in \mathbb{A}^n \ \middle|\ f(p) = 0 \,\forall\, f \in S \right\} \subset \mathbb{A}^n
	.\end{equation} 
	Subsets of $\mathbb{A}^n$ of the form of $\mathbb{V}(S)$ are called \textbf{(affine) algebraic sets}.

	For finitely many polinomials $S = \left\{ f_1, \ldots, f_n \right\}$ we denote
	 \begin{equation}
		 \mathbb{V}(S) =: \mathbb{V}(f_1, \ldots, f_n)
	.\end{equation} 
\end{defn}

\begin{rem}
	The set $S$ defining the algebraic susbset $X = \mathbb{V}(S)$ is not unique. For instance:
	\begin{itemize}
		\item if $f$ and $g$ both vanish on $X$, then $f + g$ does so,
		\item if $f$ vanishes on $X$, then for all $h \in \mathbb{K}[x_1, \ldots, x_n]$, also $hf$ vanishes on $X$.
	\end{itemize}
	In particular we have, denoting by $\left( S \right)$ the ideal generated by $S$, the following equality
	\begin{equation}
		\mathbb{V}(S) = \mathbb{V}\left(\, (S)  \,\right)
	.\end{equation} 
\end{rem}

\begin{rem}
	In this course a ring $R$ will always be a commutative ring with unity.
\end{rem}

\subsection{Set-theoretical properties of algebraic sets}

\begin{lem}
	If $S_1$ and $S_2$ are sets of polynomials, then
	\begin{equation}
		\mathbb{V}(S_1) \cup \mathbb{V}(S_2) = \mathbb{V}(S_1 \cdot S_2)
	,\end{equation} 
	where $S_1 \cdot S_2 := \left\{ fg \ \middle|\ f \in S_1, g \in S_2 \right\}$.

	Moreover, for a family $\left\{ S_i \right\}_{i \in I}$, with $S_i \in \mathbb{K} \left[x_1, \ldots, x_n \right]$, we have
	\begin{equation}
		\bigcap_{i \in I}\mathbb{V}(S_i) = \mathbb{V} \big( \bigcup_{i \in I}S_i \big)
	.\end{equation} 
\end{lem} 
\begin{cor}
	Finite unions of algebraic sets and arbitrary intersections of algebraic sets are, again, algebraic sets.
\end{cor} 

\begin{defn}[Zariski topology]
	The \textbf{algebraic} subsets of $\mathbb{A}^n$ satisfy the properties for \textbf{closed} subsets in a topology: 
	\begin{itemize}
		\item $\emptyset, \mathbb{A}^n$ are algebraic,
		\item finite unions and arbitrary intersections of algebraic subsets are algebraic.
	\end{itemize}
	This means they induce a topology, which is called the \textbf{Zarisky} topology.
\end{defn}

From the above definition, in general, we will not refer to algebraic subsets of $\mathbb{A}^n$ as such, but as \textbf{closed} subsets in the \textbf{Zariski} topology.

\begin{rem}
	The \textbf{Zariski} topology is not \textbf{Hausdorff}.
	For example in $\mathbb{A}^1$ the \textbf{Zariski} topology is exactly the cofinite topology.
\end{rem}

\subsection{Hilbert's Nullstellensatz}
\begin{defn}[Associated ideal]
	Let $X \subset \mathbb{A}^n$, the ideal associated to $X$ is
	\begin{equation}
		\mathbb{I}(X) := \left\{ f \in \mathbb{K}\left[x_1, \ldots, x_n \right] \ \middle|\ f(p) = 0\ \,\forall\, p \in X \right\}
	.\end{equation} 
\end{defn}

\begin{lem}
	Let $S, T \subset \mathbb{K}\left[x_1, \ldots, x_n \right]$ and $X, Y \subset \mathbb{A}^n$, then
	\begin{enumerate}
		\item $\mathbb{V}$ and $\mathbb{I}$ reverse inclusions, i.e.
			\begin{align}
				S \subset T &\implies \mathbb{V}(T) \subset \mathbb{V}(S)\\
				X \subset Y &\implies \mathbb{I}(Y) \subset \mathbb{I}(X)
			,\end{align} 
		\item $X \subset \mathbb{V}(\mathbb{I}(X))$ and $S \subset \mathbb{I}\left( \mathbb{V}(S) \right)$,
		\item if $X$ is algebraic, then $X = \mathbb{V}\left(\mathbb{I}(X)\right)$.
	\end{enumerate}
\end{lem} 

\begin{rem}
	In general, given an ideal $\mathcal{I} \subset \mathbb{K}\left[x_1, \ldots, x_n \right]$, the identity
	\begin{equation}
		\mathcal{I} = \mathbb{I}\left( \mathbb{V}\left(\mathcal{I}\right) \right)
	\end{equation} 
	is false.
\end{rem}

\begin{lem}
	Let $\left( a_1, \ldots, a_n \right) \in \mathbb{A}^n$, then
	\begin{equation}
		\mathbb{I}\left( \left\{ a \right\} \right) = \left( x_1 - a_1, \ldots, x_n - a_n \right)
	.\end{equation} 
\end{lem} 

\begin{thm}[Hilbert's Nullstellensatz]
	Let $\mathbb{K}$ be an algebraically closed field and $f_1, \ldots, f_n \in \mathbb{K}\left[x_1, \ldots, x_n \right]$.
	If $\left( f_1, \ldots, f_n \right) \neq \left( 1 \right)$ (i.e. it is a proper ideal), then $\mathbb{V}\left(f_1, \ldots, f_n\right) \neq \emptyset$ in $\mathbb{A}^n(\mathbb{K})$.
\end{thm}

\begin{rem}
	This theorem is the reason why we work with algebraically closed fields, moreover, if $\mathbb{K}$ is not algebraically closed, then a common zero of $f_1, \ldots, f_n$ will exist over a finite extension of $\mathbb{K}$.
\end{rem}

\begin{cor}
	If $\mathbb{K}$ is algebraically closed, then $\mathbb{I}$ and $\mathbb{V}$ are bijections between the set of points in $\mathbb{A}^n$ and the set of maximal ideals in $\mathbb{K}\left[x_1, \ldots, x_n \right]$.
\end{cor} 

\begin{defn}[Radical of an ideal]
	Let $R$ be a ring and $\mathcal{I} \subset R$ be an ideal of $R$.
	We define the radical of $\mathcal{I}$, denoted with $\sqrt{\mathcal{I}}$, to be
	\begin{equation}
		\sqrt{\mathcal{I}} := \left\{ x \in R \ \middle|\ \exists\, n \in \N_+, \text{ s.t. } x^n \in \mathcal{I} \right\}
	.\end{equation} 
	If $\mathcal{I} = \sqrt{\mathcal{I}}$ we say that the ideal $\mathcal{I}$ is radical.
\end{defn}

\begin{prop}[Nullstellensatz]
	Let $\mathbb{K}$ be algebraically closed and $\mathcal{I} \subset \mathbb{K}\left[x_1, \ldots, x_n \right]$ be an ideal, then
	\begin{equation}
		\mathbb{I}\left( \mathbb{V}\left(\mathcal{I}\right) \right) = \sqrt{\mathcal{I}}
	.\end{equation} 
\end{prop} 

\begin{cor}
	The maps $\mathbb{I}$ and $\mathbb{V}$, from algebraic sets in $\mathbb{A}^n(\mathbb{K})$ to radical ideals in  $\mathbb{K}\left[x_1, \ldots, x_n \right]$ are inclusion-reversing correspondences that are inverse to one another.
\end{cor} 

Let's now state some basic operations on algebraic sets:
\begin{lem}
	Let $\mathcal{I}_1, \mathcal{I}_2 \subset \mathbb{K}\left[x_1, \ldots, x_n \right]$ ideals, then
	\begin{enumerate}
		\item $\displaystyle{\mathbb{V}\left(\mathcal{I}_1\right) \cap_{} \mathbb{V}\left(\mathcal{I}_2\right) = \mathbb{V}\left(\mathcal{I}_1 + \mathcal{I}_2\right)}$,
		\item $\displaystyle{\mathbb{V}\left(\mathcal{I}_1\right) \cup_{} \mathbb{V}\left(\mathcal{I}_2\right) = \mathbb{V}\left(\mathcal{I}_1 \mathcal{I}_2\right) = \mathbb{V}\left(\mathcal{I}_1 \cap_{} \mathcal{I}_2\right)}$.
	\end{enumerate}
\end{lem} 

\begin{lem}
	Let $X, Y \subset \mathbb{A}^n$ algebraic subsets, then
	\begin{enumerate}
		\item $\displaystyle{\mathbb{I}\left( X \cup_{} Y \right) = \mathbb{I}\left( X \right) \cap_{} \mathbb{I}\left( Y \right)}$,
		\item $\displaystyle{\mathbb{I}\left( X \cap_{} Y \right) = \sqrt{\mathbb{I}\left( X \right) + \mathbb{I}\left( Y \right)}}$.
	\end{enumerate}
\end{lem} 

\subsection{Reducibility}
\begin{defn}[Reducibile subsets]
	A topological space $X$ is \textbf{reducible} iff there exist proper closed subsets $X_1, X_2 \subsetneq X$ s.t.
	\begin{equation}
	X = X_1 \cup_{} X_2
	.\end{equation} 
	If $X$ is not reducible, it is said to be \textbf{irreducible}.
\end{defn}
\begin{defn}[Connected space]
	If $X$ is reducible and it is a disjoint union of closed subsets
	\begin{equation}
	X = X_1 \cup_{} X_2
	,\end{equation} 
	then we say that $X$ is \textbf{disconnected}.
	If $X$ is not disconnected it is said to be \textbf{connected}.
\end{defn}
\begin{rem}
	In order to be \textbf{disconnected}, a space $X$ has to be \textbf{reducible}, i.e. there exist no irreducible disconnected spaces.
\end{rem}

\begin{rem}
	If a space is \textbf{Hausdorff}, then only the singletons are \textbf{irreducible}.
\end{rem}

\begin{defn}[Affine variety]
	An irreducible algebraic set in $\mathbb{A}^n$, with the Zariski topology, is called an \textbf{affine variety}.
\end{defn}

\begin{lem}[Algebraic characterization of irreducibility]
	An algebraic set $X \subset \mathbb{A}^n$ is an affine variety iff $\mathbb{I}(X) \subset \mathbb{K}\left[x_1, \ldots, x_n \right]$ is a prime ideal.
\end{lem} 
\begin{rem}
	Recall: $\mathcal{I} \subset R$ is a prime ideal iff $R/\mathcal{I}$ is an integral domain.
\end{rem}

\begin{defn}[Noetherian topological space]
	A topological space $X$ is called \textbf{Noetherian} iff every descending chain of closed subset
	\begin{equation}
	X \supset X_1 \supset X_2 \supset X_3 \supset \ldots \supset X_n \supset X_{n+1} \supset \ldots
	\end{equation} 
	is stationary.
\end{defn}

\begin{rem}
	Each algebraic set, in the Zariski topology, is \textbf{Noetherian}.
	This follows from the fact that $\mathbb{K}\left[x_1, \ldots, x_n \right]$ (hence every quotient of this ring) is a Noetherian ring, therefore, by Nullstellensatz,
	\begin{equation}
		\mathbb{I}(X) \subset \mathbb{I}(X_1) \subset \mathbb{I}(X_2) \subset \ldots \subset \mathbb{I}(X_n) \subset \mathbb{I}(X_{n+1})
	.\end{equation} 
	Since the maps $\mathbb{V}$ and $\mathbb{I}$ reverse inclusions we have our claim. 
\end{rem}

\begin{prop}
	Every \textbf{Noetherian} topological space $C$ can be written as
	\begin{equation}
	X = X_1 \cup_{} X_2 \cup_{} \ldots \cup_{} X_r
	,\end{equation} 
	a finite union of disjoint irreducible closed subsets, with $X_1 \not\subset X_j$ for $j \neq i$.
	Moreover this decomposition is unique up to order of $X_1, \ldots, X_r$.
\end{prop} 

\begin{defn}[Irreducible components]
	The irreducible subsets $X_1, \ldots, X_r$ are called the \textbf{irreducible components} of $X$.
\end{defn}

\begin{cor}
	Every affine algebraic set $X$ is the finite union of affine varieties $X_1, \ldots, X_r$, i.e.
	\begin{equation}
	X = X_1 \cup_{} \ldots \cup_{} X_r
	.\end{equation} 
	From the algebraic point of view this can be viewed using Hilbert Nullstellesatz:
	let $\mathbb{I}(X)$ be the radical ideal associated to $X$ and $\mathbb{I}(X_i)$ the prime ideals associated to $X_i$, then
	\begin{equation}
		\mathbb{I}(X) = \mathbb{I}\left( X_1 \cup_{} \ldots \cup_{} X_r \right) = \mathbb{I}\left( X_1 \right) \cap_{} \ldots \cap_{} \mathbb{I}\left( X_r \right)
	,\end{equation} 
	i.e. every radical ideal in $\mathbb{K}\left[x_1, \ldots, x_n \right]$ is the finite intersection of prime ideals.
\end{cor} 

\begin{rem}
	If one reads (and adapts) the proof of the above proposition, one can prove that each algebraic set is the disjoit union of finitely many connected components.
\end{rem}

\begin{defn}[Dimension of a Noetherian topological space]
	Let $X$ be a nonempty irreducible topological space.
	We define the dimension of $X$ to be the largest index $n$ s.t. there exist a chain of length $n+1$ 
	\begin{equation}
	\emptyset \neq X_0 \subsetneq X_0 \subsetneq X_1 \subsetneq \ldots \subsetneq X_n = X
	\end{equation} 
	of irreducible closed subsets of $X$.

	If $X$ is an arbitrary Noetherian topological space (it might be reducible), the dimension of $X$ is defined to be the supremum (hence the maximum) of the dimensions of its irreducible components.
\end{defn}
\begin{rem}
	If all components of $X$ have the same dimension we say that $X$ is \textbf{pure-dimensional}.
	One gives special names to small pure-dimensional topological spaces:
	\begin{itemize}
		\item if $X$ is of pure dimension $1$, we call $X$ a \textbf{curve},
		\item if $X$ is of pure dimension $2$, we call $X$ a \textbf{surface.}
	\end{itemize}
\end{rem}

\begin{rem}[Characterization of irreducibility]
	The following properties are equivalent:
	\begin{itemize}
		\item $X$ is irreducible,
		\item any two nonempty open subsets of $X$ have a nonempty intersection,
		\item every nonempty open subset $U$ is dense in $X$, i.e. $\overline{U} = X$,
		\item every nonempty open subset of $X$ is connected.
	\end{itemize}
	Moreover, the image of an irreducible subset under a continuous map is again irreducible.
\end{rem}

\begin{defn}[Hypersurface]
	$X \subset \mathbb{A}^n$ is an \textbf{hypersurface} iff $X = \mathbb{V}\left( f \right)$ for a single polynomial $f$ of positive degree.
\end{defn}

\begin{ex}[Irreducible varieties]\leavevmode\vspace{-.2\baselineskip}
	\begin{itemize}
		\item Every linear subspace of $\mathbb{A}^n$ is irreducible (i.e. it is an affine variety).
			As a consequence we obtain that $\mathrm{dim}\mathbb{A}^n \geq n$.
		\item An hypersurface $X = \mathbb{V}\left( f \right)$ is irreducible iff $f$ is a power of an irreducible polynomial.
	\end{itemize}
\end{ex} 

\subsection{Functions and Morphisms}
\begin{defn}[Regular function]
	Given $X \subset \mathbb{A}^n$ a \textbf{Zariski} closed subset, a function
	\begin{equation}
	f: X \to \K
	\end{equation} 
	is said to be \textbf{regular} iff there exist $F \in \mathbb{K}\left[x_1, \ldots, x_n \right]$ s.t.
	\begin{equation}
		f(a) = F(a) \quad \,\forall\,  a \in X
	.\end{equation} 
\end{defn}
\begin{rem}
	Notice that, in the above definition, $F$ is unique up to an element in $\mathbb{I}(X)$.
\end{rem}

\begin{defn}[Coordinate ring]
	The coordinate ring of $X$ is
	\begin{equation}
		\underbrace{\K[X] := \mathbb{A}(X)}_{\text{alternative notations}} := \mathbb{K}\left[x_1, \ldots, x_n \right]/\mathbb{I}(X)
	.\end{equation} 
	It is the set of all regular functions on $X$
\end{defn}

\begin{constr}[Localization]
	Let $R$ be a ring.
	A set $S \subset R$ is \textbf{multiplicatively closed} iff $1 \in S$ and $\,\forall\, f,g \in S$ then also $fg \in S$.\newline
	We construct on $R \cross S$ the equivalence relation
	\begin{equation}
		\left(f, g\right) \sim \left(f', g'\right) \iff \exists\, h \in S \text{ s.t. } h \left( fg' - f'g \right) = 0
	.\end{equation} 
	We then define the localization of $R$ at $S$ as
	\begin{equation}
	S^{-1} R := (R \cross S)/\sim
	.\end{equation}
	We will denote the elements of the localization as fractions:
	\begin{equation}
	\left[ \left(f, g\right) \right] =: \frac{f}{g}
	.\end{equation} 
\end{constr} 
\begin{rem}\leavevmode\vspace{-.2\baselineskip}
	\begin{itemize}
		\item If $R$ is a \textbf{domain}, then we can always take $h = 1$ in the definition of the equivalence relation.
			In this case, if $S \subset S'$ are multiplicatively closed subsets, then
			\begin{align}
				S^{-1}R &\hookrightarrow (S')^{-1} R\\
				\frac{f}{g} &\mapsto \frac{f}{g}
			\end{align} 	
			is injective, provided $0 \not\in S'$.
		\item $S^{-1} R$ is a ring, with addition and multiplication defined as usual with fractions:
			\begin{align}
				\frac{f}{g} +  \frac{f'}{g'} &:= \frac{fg' + f'g}{gg'},\\
				\frac{f}{g} \cdot \frac{f'}{g'} &:= \frac{ff'}{gg'}
			.\end{align} 
	\end{itemize}
\end{rem}

\begin{defn}[Local ring]
	Let $R$ be a ring. We say that $R$ is \textbf{local} iff it has a unique maximal ideal.
\end{defn}

\begin{ex}\leavevmode\vspace{-.2\baselineskip}
	\begin{enumerate}
		\item Let $a \in R$ be an arbitrary element.
			Let $S := \left\{ a^n \ \middle|\ n \in \N \right\}$, then we define
			\begin{equation}
			R_a := S^{-1} R
			,\end{equation} 
			and we call it the localization of $R$ at $a$.
		\item Let $\mathcal{P} \subset R$ a \textbf{prime} ideal of $R$. let $S := R \setminus \mathcal{P}$, then we denote with
			\begin{equation}
				R_{\mathcal{P}} := S^{-1} R
			\end{equation} 
			the localization of $R$ at $\mathcal{P}$.
			Here we can describe the ideals of $R_{\mathcal{P}}$, they are of the form
			\begin{equation}
				\left\{ \frac{f}{g} \ \middle|\  f \in \mathcal{I}, g \not\in \mathcal{P} \right\}
			,\end{equation} 
			for some $\mathcal{I} \subset \mathcal{P}$ ideal of $R$.
			
			In particular $R_{\mathcal{P}}$ has a unique maximal ideal
			\begin{equation}
			M = \left\{ \frac{f}{g} \ \middle|\ f \in \mathcal{P}, g \not\in \mathcal{P} \right\}
			\end{equation}
			hence it is a \textbf{local} ring.
		\item Let $S := R \setminus \left\{ \text{zero divisors} \right\}$.
			This clearly is a multiplicatively closed subset, since $\left\{ \text{zero divisors} \right\}$ is a prime ideal of $R$.

			If $R$ is an integral domain, then $S = R \setminus \left\{ 0 \right\}$
			and we denote the localization as
			\begin{equation}
				Q(R) := S^{-1} R
			,\end{equation} 
			the field of fractions of R.
	\end{enumerate}
\end{ex} 

\begin{rem}
	When $X$ is an affine \textbf{variety}, then $\mathbb{I}(X)$ is prime, hence $\K[X]$ is an integral domain.
\end{rem}

\begin{defn}[Field of rational functions]
	Let $X \subset \mathbb{A}^n$ be an affine variety.
	We define $\K(X) := Q \left( \K[X] \right)$ to be the field of \textbf{rational functions} on $X$.
\end{defn}

\begin{rem}
	We call the elements of  $\K(X)$ rational functions even if their values may not be defined on all points of $X$.
\end{rem}

\begin{defn}[Regular functions]\leavevmode\vspace{-.2\baselineskip}
	\begin{enumerate}
		\item We say that $\varphi \in \K(X)$ is \textbf{regular at} $p \in X$ iff
			\begin{equation}
				\varphi = \frac{f}{g}, \text{ with }  f,g \in \K[X] \text{ and } g(p) \neq 0
			,\end{equation} 
			i.e. $\varphi$ is well defined at $p$,
		\item We define the \textbf{local ring of} $X$ at $p \in X$ to be
			\begin{equation}
				\mathcal{O}_{X,p} := \left\{ \varphi \in \K(X) \ \middle|\ \varphi \text{ is regular at } p \right\}
			,\end{equation} 
		\item For any nonempty open $U \subset X$, we define
			\begin{equation}
				\mathcal{O}_X (U) := \bigcap_{p \in U} \mathcal{O}_{X,p}
			,\end{equation} 
			the ring of regular functions on $U$.
	\end{enumerate}
\end{defn}

\begin{rem}
	In general
	\begin{equation}
		\mathcal{O}_X (U) \neq \left\{ \frac{f}{g} \ \middle|\ f,g \in \K[X],\ g(p) \neq 0\ \,\forall\, p \in U  \right\}
	.\end{equation} 
\end{rem}

\begin{defn}[$\K$-algebra]
	A $\K$-algebra is a ring $R$ which is also a $\K$-Vector Space, s.t. the ring multiplication is $\K$-bilinear, i.e.
	\begin{equation}
		\lambda(fg) = (\lambda f)g = f(\lambda g)
	,\end{equation} 
	for $f,g \in R$ and $\lambda \in \K$.

	Moreover we define morphisms of $\K$-algebras as those maps between $\K$-algebras that are both $\K$-linear and ring homomorphisms.
\end{defn}

\begin{rem}
	Rings of functions are always $\K$-algebras
\end{rem}

\begin{lem}[Identity principle]
	Let $U \subset V \subset X$ be nonempty open subsets of an affine variety $X$.
	If $\phi_1, \phi_2 \in \mathcal{O}_X(V)$ coincide when restricted to $U$, then they are the same function also on $V$, i.e. for $\phi_1, \phi_2 \in \mathcal{O}_X(V)$
	\begin{equation}
		\left.\phi_1\right|_{U} = \left.\phi_2\right|_{U} \in \mathcal{O}_X(U) \implies \phi_1 = \phi_2 \in \mathcal{O}_X(V)
	.\end{equation} 
	It is the same property that holds for holomorphic functions.
\end{lem} 

\begin{rem}
	Recall that an affine variety $X$ is irreducible, this means that any open subset in $X$ is dense.
	With this in mind the above result should not sound too strange.
	It actually seems less strong than what can be proved for holomorphic functions.
\end{rem}

\begin{defn}[Distinguished open subsets]
	Let $X \subset \mathbb{A}^n$ be an affine variety and $f \in \K[X]$ a regular function on $X$, then
	\begin{equation}
		X_f := \left\{ p \in X \ \middle|\ f(p) \neq 0 \right\} = X \setminus \mathbb{V}\left( F \right)
	,\end{equation} 
	for some $F \in \mathbb{K}\left[x_1, \ldots, x_n \right]$ s.t. $\left[ F \right] = f \in \K[X]$, is an open subset of $X$ (in the Zariski topology).
	It is in fact the complement of a hypersurface, which is closed in $X$.
	Such open subsets are called \textbf{distinguished open} subsets of $X$.
\end{defn}

\begin{rem}
	The distinguished open subsets form a base for the Zariski topology on $X$.
	It means that any open subset can be written as a union of distinguished open subsets.

	By Hilbert's basis theorem we can say even more: every closed subset is the intersection of finitely many hypersurfaces, hence (taking the complements) every open subset is a finite union of distinguished open subsets.
\end{rem}

\begin{prop}
	For any $f \in \K[X]$ we have
	\begin{equation}
		\mathcal{O}_X(X_f) = \left\{ \phi\in \K(X) \ \middle|\ \phi = \frac{g}{f^r} \text{ for some } r \geq 0,\ g \in \K[X] \right\}
	.\end{equation} 
	In particular, $\mathcal{O}_X(X_f) \cong K[X]_f$.
\end{prop} 
\begin{cor}
	If $f = 1$, then $X_f = X$, hence the proposition grants
	\begin{equation}
		\mathcal{O}_X(X) \cong \K[X]
	,\end{equation} 
	hence the rational functions which are regular at each point of $X$ are exactly the regular functions on  $X$.
\end{cor} 

\begin{rem}
	Also the local ring $\mathcal{O}_{X,p}$ is a localization of $\mathbb{K}\left[X\right]$:
	\begin{equation}
		\mathfrak{m}_{X,p} := \left\{ g \in \K[X] \ \middle|\ g(p) = 0 \right\} = \mathbb{I}(\{p\})/\mathbb{I}(X) \subset \K[X]
	\end{equation}
	is a maximal, hence prime, ideal.
	This is true, since the map
	\begin{align}
		\K[X]/\mathfrak{m}_{X,p} &\to \K\\
		\left[ g \right] &\mapsto g(p)
	\end{align} 
	is an isomorphism.
	Moreover one can prove that the natural map
	\begin{align}
		\mathcal{O}_{X,p} &\to \K[X]_{\mathfrak{m}_{X,p}}\\
		\frac{f}{g} &\mapsto \frac{f}{g}
	,\end{align} 
	with $g(p) \neq 0$, is an isomorphism.
\end{rem}

\begin{rem}
	In general, for $U \subset X$ an arbitrary open subset, in order to compute $\mathcal{O}_X(U)$ we proceed as follows:
	\begin{itemize}
		\item we write $U$ as the union of finitely many distinguished open subsets (coming from the generators of $\mathbb{I}(X\setminus U)$),
		\item we use the decomposition $U = X_{f_1} \cup_{} X_{f_2} \cup_{} \ldots \cup_{} X_{f_l}$ and $\mathcal{O}_X(X_{f_j}) = \K[X]_{f_j}$ to figure out what $\mathcal{O}_X(U)$ is.
	\end{itemize}
\end{rem}

\begin{ex}
	Let $X = \mathbb{A}^{2}$ and $X \supset U := \mathbb{A}^{2} \setminus \left\{ \left( 0,0 \right) \right\}$.
	We can write $U = \mathbb{A}_x^2 \cup \mathbb{A}_y^2$.
	Carrying out some computations on these distinguished open subsets we can arrive at the conclusion that
	\begin{equation}
	\mathcal{O}_{\mathbb{A}^{2}} \left( \mathbb{A}^{2} \setminus \left\{ \left(0, 0\right) \right\} \right) =
	\mathbb{K}[x,y] = \mathbb{K}[\mathbb{A}^{2}]
	.\end{equation} 
	This result can be interpeted as:
	"the point $\left(0, 0\right)$ is a removable singularity".
\end{ex} 

\begin{rem}
	Since we can check regularity at a neighborhood of single points, then regularity is a \textbf{local} property.
\end{rem}


\section{Sheaves}
Let $X$ be a topological space.

\begin{defn}[Presheaf]
	A \textbf{presheaf} of rings $\mathcal{F}$ on $X$ is the assignment of a ring $\mathcal{F}(U)$ to each open subsets $U \subset X$, and to each pair of comparable open subset $U,V$ (i.e. $U \subset V$), of a ring homomorphism
	\begin{equation}
		\rho_{V,U}: \mathcal{F}(V) \to \mathcal{F}(U)
	,\end{equation}
	satisfying the following requirements:
	\begin{description}
		\item[S1] $\mathcal{F}(\emptyset) = 0$,
		\item[S2] $\rho_{U,U} = id_{\mathcal{F}(U)}$,
		\item[S3] for all $U \subset V \subset W$, then
			\begin{equation}
			\rho_{W,U} = \rho_{V,U} \circ \rho_{W,V}
			.\end{equation}
	\end{description}
	We call the rings $\mathcal{F}(U)$ sections of $\mathcal{F}$ over $U$ and the morphism $\rho_{V,U}$ the restriction morphisms, which we denote as
	\begin{equation}
		\rho_{V,U}(f) =: \left.f\right|_{V} 
	.\end{equation} 
\end{defn}

\begin{defn}[Sheaf]
	A presheaf $\mathcal{F}$ of rings is called a \textbf{sheaf} if it fulfills the gluing condition:
	\begin{description}
		\item[S4] for each open subset $U \subset X$, each open covering $\left\{ U_\alpha \right\}_{\alpha \in \mathcal{A}}$ of $U$ and each family of sections $\left\{ s_\alpha \in \mathcal{F}(U_\alpha) \right\}_{\alpha \in \mathcal{A}}$ s.t. $s_\alpha$ and $s_\beta$ have the same restriction to $U_\alpha \cap_{} U_\beta$ for all $\alpha, \beta \in \mathcal{A}$, there is a \textbf{unique} section $s \in \mathcal{F}(U)$ s.t. $\rho_{U, U_\alpha}(s) = s_\alpha$ for all $\alpha \in \mathcal{A}$.
	\end{description} 
\end{defn}

\begin{rem}\leavevmode\vspace{-.2\baselineskip}
	\begin{description}
		\item[S2,S3] ensure that $\mathcal{F}$ is a contravariant functor
			\begin{equation}
				\mathcal{F}: \left\{ U \subset X \ \middle|\ U \text{ is open in } U \right\} =: \mathsf{Op}(X) \to \mathsf{Rings}
			.\end{equation} 
		\item[S1] is customary in algebraic geometry, but it is a trivial requirement:
			given a functor $\mathcal{F}$ as above, then
			\begin{equation}
				\widetilde{\mathcal{F}}(U) =
				\begin{cases}
					0 & \text{ if } U = \emptyset,\\
					\mathcal{F}(U) & \text{ if } U \neq \emptyset
				\end{cases} 
			\end{equation} 
			is a presheaf.
		\item[S4 $\implies$ S1] it follows from the case when $\mathcal{A} = \emptyset$, hence any \textbf{sheaf} is also a \textbf{presheaf}.
	\end{description} 
\end{rem}

\begin{rem}
	We can actually define (pre)-sheaves of abelian groups, by requiring  $\mathcal{F}(U)$ to be abelian groups and the maps $\rho_{U,V}$ to be group homomorphisms.
\end{rem}
\begin{rem}
	We can also define (pre-)sheaves of $\K$-algebras, as before:
	$\mathcal{F}(U)$ are $\K$-algebras, and $\rho_{U,V}$ are moprhisms of $\K$-algebras.
\end{rem}

\begin{ex}
	Let $X \subset \mathbb{A}^n$ be an affine variety, then the rings $\mathcal{O}_X(U)$, with the natural restriction maps
	\begin{equation}
		\mathcal{O}_X(V) \to \mathcal{O}_X(U) \quad \text{ for } U \subset V
	\end{equation} 
	form a sheaf of rings ($\K$-algebras).
	We denote this sheaf by $\mathcal{O}_X$ and it is called the \textbf{sheaf of regular functions}, or the \textbf{structure sheaf}, on $X$.
\end{ex} 

\begin{defn}[Ringed space]
	A pair $\left(X, \mathcal{O}_X\right)$ of a topological space $X$ and a sheaf of rings $\mathcal{O}_X$ on $X$ is called a \textbf{ringed space}.
	Moreover $\mathcal{O}_X$ is called the \textbf{structure sheaf} of the ringed space.
\end{defn}
\begin{rem}
	In general, spaces of functions easily form a presheaf of rings.
	However, they only form a sheaf if the conditions imposed on the functions are local.
	In fact the condition \textbf{S4} guarantees that we can check whether a function belongs to the sections locally.
\end{rem}

\begin{defn}[Restrictions of (pre-)sheaves]
	Let $U \subset X$ be an open subset.
	Given a (pre-)sheaf $\mathcal{F}$ on $X$, we define its restriction $\left.\mathcal{F}\right|_{U}$ as the sheaf defined by
		\begin{equation}
			\left.\mathcal{F}\right|_{U} (V) := \mathcal{F}(V)
		,\end{equation} 
		for each open subset $V \subset U$ and we take the same restriction morphisms as $\mathcal{F}$.
		Notice that, given a sheaf $\mathcal{F}$ also $\left.\mathcal{F}\right|_{U}$ is a sheaf.
\end{defn}

Generalization of the \textbf{local rings} $\mathcal{O}_{X,p}$ to sheaves:
\begin{defn}[Stalk of a (pre-)sheaf]
	Consider $\mathcal{F}$ a presheaf on $X$ and a point $p \in X$.
	Consider the quotient space
	\begin{equation}
		\mathcal{F}_p := \left\{ \left(U, \phi\right) \ \middle|\ p \in U \stackrel{\text{open}}{\subset} X, \phi \in \mathcal{F}(U)  \right\}/\sim
	,\end{equation} 
	where the equivalence relation is defined as follows:
	$\left(U_1, \phi_1, \right) \sim \left(U_2, \phi_2\right)$ iff $\exists\, V \stackrel{\text{open}}{\subset} X$ s.t.
	$p \in V \subset U_1 \cap_{} U_2$, and $\left.\phi_1\right|_{V} = \left.\phi_2\right|_{V}$.

	This quotient space is called the \textbf{stalk} of $\mathcal{F}$ at $p$, whereas its elements $\left[ \left(U, \phi\right) \right] \in \mathcal{F}_p$ are called  the \textbf{germs} of $\mathcal{F}$ at $p$.
\end{defn}

\begin{rem}
	$\mathcal{F}_p$ inherits the algebraic structure from $\mathcal{F}(U)$. (We just need to check that the operations defined on $\mathcal{F}(U)$ are preserved by the equivalence relation).
	In other words if $\mathcal{F}$ is a sheaf of rings, then $\mathcal{F}_p$ is a ring for any $p \in X$.
\end{rem}

\begin{lem}
	Let $X$ be an affine variety and $p \in X$, then the stalk of $\mathcal{F}= \mathcal{O}_X$ at $p$ is (canonically isomorphic to) the local ring $\mathcal{O}_{X,p}$.
\end{lem} 

\section{Morphisms between affine varieties}
\begin{defn}[Morphism of ringed spaces]
	Given two ringed spaces $\left(X, \mathcal{O}_X\right)$ and $\left(Y, \mathcal{O}_Y\right)$, a \textbf{morphism of ringed spaces} from  $\left( X, \mathcal{O}_{ X } \right)$ to $\left( Y, \mathcal{O}_{ Y } \right)$ consists of:
	\begin{itemize}
		\item a continuous map $f: X \to Y$, 
		\item for each open subset $U \subset Y$, a ring homomorphism
			\begin{equation}
				f_U: \mathcal{O}_Y(U) \to \mathcal{O}_X \left( f^{-1}(U) \right)
			,\end{equation} 
			s.t. the following diagram commutes for all $U \stackrel{\text{open}}{\subset}  V \stackrel{\text{open}}{\subset} Y$
			\begin{equation}
			\begin{tikzcd}
				\mathcal{O}_Y(V) \arrow[rr, "\rho_{V,U}", rightarrow] \arrow[d, "f_V"', rightarrow] & \ & \mathcal{O}_Y(U) \arrow[d, "f_U", rightarrow] \\
				\mathcal{O}_X \left( f^{-1}(V) \right) \arrow[rr, "\rho_{f^{-1}U, f^{-1}U}"', rightarrow] & \ & \mathcal{O}_X \left( f^{-1}(U) \right)
			\end{tikzcd}
			.\end{equation} 
	\end{itemize}
\end{defn}

\begin{defn}[Isomorphism of ringed spaces]
	An \textbf{isomorphism of ringed spaces} is a morphism of ringed spaces
	\begin{equation}
	f: \left( X, \mathcal{O}_{ X } \right) \to \left( Y, \mathcal{O}_{ Y } \right)
	\end{equation} 
	s.t. there exists another morphism of ringed spaces
	\begin{equation}
	g: \left( Y, \mathcal{O}_{ Y } \right) \to \left( X, \mathcal{O}_{ X } \right)
	\end{equation} 
	with $g \circ f = id_{\left( X, \mathcal{O}_{ X } \right)}$ and $f \circ g = id_{\left( Y, \mathcal{O}_{ Y } \right)}$.
\end{defn}

\begin{rem}
	Notice that we only work with ringed spaces $\left( X, \mathcal{O}_{ X } \right)$ s.t.
	\begin{itemize}
		\item the elements of $\mathcal{O}_X(U)$, for any $U \stackrel{\text{open}}{\subset} X$ are functions $U \to\K$,
		\item the restriction maps are given by
			\begin{align}
				\mathcal{O}_X(V) &\to \mathcal{O}_X(U) \\
				\phi &\mapsto \left.\phi\right|_{U} 
			.\end{align} 
	\end{itemize}
	This implies that a continuous map $f: X \to Y$ induces, for any $U \stackrel{\text{open}}{\subset} Y$, a ring homomorphism $f^*$:
	\begin{align}
		\mathcal{O}_Y(U) &\xrightarrow{f^*} \mathcal{O}_X \left( f^{-1}(U) \right) \\
		\phi &\mapsto f^*\phi := \phi \circ f
	.\end{align} 
	We will almost always take these ring homomorphisms as our $f_U$.
\end{rem}

\begin{rem}
	In our case, then, a morphism of ringed space $\left( X, \mathcal{O}_{ X } \right) \xrightarrow{f} \left( Y, \mathcal{O}_{ Y } \right)$ is a continuous map 
	\begin{equation}
	f: X \to Y
	\end{equation} 
	such that $f^*\phi \in \mathcal{O}_X \left( f^{-1}(U) \right)$ for any $U \stackrel{\text{open}}{\subset} Y$ and $\phi \in \mathcal{O}_Y(U)$.
\end{rem}

\begin{defn}[Morphism of affine varieties]
	A \textbf{morphism of affine varieties}
	\begin{equation}
	f: X \to Y
	\end{equation} 
	is a morphism of ringed spaces $f: \left( X, \mathcal{O}_{ X } \right) \to \left( Y, \mathcal{O}_{ Y } \right)$ (with $f_U := f^*$ for all $U \stackrel{\text{open}}{\subset} Y$).
\end{defn}

\begin{lem}
	Let $X, Y$ be \textbf{affine varieties} and $f: X \to Y$ a continuous map. TFAE:
	\begin{enumerate}
		\item $f$ is a morphism,
		\item for every $\phi \in \mathcal{O}_Y(Y)$ we have $f^*\phi \in \mathcal{O}_{X}(X)$, i.e. the pull-back of any regular function on $Y$ is regular everywhere on $X$,
		\item for every $p \in X$ and $\phi \in \mathcal{O}_{Y, f(p)}$ we have $f^*\phi \in \mathcal{O}_{X,p}$, i.e. the pull-back of a function regular at $f(p)$ is regular at $p$.
	\end{enumerate}
\end{lem} 

\begin{ex}\leavevmode\vspace{-.2\baselineskip}
	\begin{enumerate}
		\item The following is a morphism of varieties
			\begin{align}
				 f: \mathbb{A}^1 &\to \mathbb{A}^1 \\
				x &\mapsto x^2
			,\end{align} 
		\item Let $X, Y$ be affine varieties, $f_1, \ldots, f_n \in \mathbb{K}[X]$. Then a map of the following form is always a morphism
			\begin{align}
				f: X &\to Y \\
				a &\mapsto \left( f_1(a), \ldots, f_n(a) \right)
			.\end{align} 
	\end{enumerate}
\end{ex} 

\begin{lem}
	Let $X \subset \mathbb{A}^m, Y \subset \mathbb{A}^n$ be affine varieties.
	We have a one to one correspondance
%	\begin{align}
%		\left\{ 
%			\begin{matrix}
%				\text{morphisms of varieties}\\
%				f: X \to Y
%			\end{matrix} 
%		\right\} &\xrightarrow{\quad\quad} 
%		\left\{ 
%			\begin{matrix}
%				\K-\text{algebra homomorphisms}\\
%				\phi: \mathbb{K}[Y] \to \mathbb{K}[X]
%			\end{matrix} 
%		\right\}\\
%		f \quad\quad\ \ &\xmapsto{\quad\quad}\quad\quad\ \ f^*
%	.\end{align} 
	\begin{equation}
	\begin{tikzcd}[row sep=tiny]
			\left\{\begin{matrix}
				\text{ morphisms of varieties }\\
				f: X \to Y
			\end{matrix}\right\} \arrow[rr, "", leftrightarrow] & &
			\left\{  \begin{matrix}
				\ \K-\text{algebra homomorphisms }\\
				\phi: \mathbb{K}[Y] \to \mathbb{K}[X]
			\end{matrix}\right\} \\
			f \arrow[rr, "", rightarrow, maps to] & & f^*
	\end{tikzcd}
	.\end{equation} 
\end{lem} 

\begin{rem}
	Applying the lemma we can view the ring of regular functions on a variety as
	\begin{equation}
	\mathbb{K}[X] = \left\{ f: X \to \mathbb{A}^1 \ \middle|\ f  \text{ is a morphism of varieties}\, \right\}
	.\end{equation} 
\end{rem} 

\begin{lem}
	Let $f: X \to Y$ be a morphism of affine varieties and let $f^*: \mathbb{K}[Y] \to \mathbb{K}[X]$ be the associated map of $\K$-algebras.
	\begin{itemize}
		\item If $f$ is surjective, then $f^*$ is injective.
		\item $f^*$ is injective iff $f(X)$ is dense in $Y$.
		\item If $f^*$ is surjective, then $f$ is injective.
		\item $f^*$ is surjective iff $f$ restricts to an isomorphism between $X$ and $f(X)$.
			In such case we call $f$ an embedding.
	\end{itemize}
\end{lem} 

\section{Product of varieties}
\begin{lem}
	If $X \subset \mathbb{A}^m$ and $Y \subset \mathbb{A}^n$ are \textbf{affine varieties}, then $X \cross Y \subset \mathbb{A}^{m+n}$ is an \textbf{affine variety}.
\end{lem} 
\begin{rem}
	While it is true that, for generic irreducible topological spaces $X$ and $Y$, then $X \cross Y$ is irreducible with the product topology.
	It is also true that the Zariski topology on $\mathbb{A}^{m+n}$ is not the product topology, it is actually finer (there are more closed subsets than in the product topology, hence it is harder to prove irreducibility).
\end{rem}

\begin{rem}
	The natural projections
	\begin{equation}
	\begin{tikzcd}[column sep=tiny]
		& X \cross Y \arrow[ld, "\pi_X"', rightarrow] \arrow[rd, "\pi_Y", rightarrow] & \\
		X & & Y
	\end{tikzcd}
	\end{equation} 
	are morphisms of varieties.
\end{rem}

\begin{prop}[Universal property of products of affine varieties]
	Let $X$ and $Y$ be \textbf{affine varieties}, then for every $Z$ affine variety with morphisms $f_X: Z \to X$ and $f_Y: Z \to Y$, then $\exists\, ! f: Z \to X \cross Y$ s.t. the following diagram commutes
	\begin{equation}
	\begin{tikzcd}
		& & X\\
		Z \arrow[r, "\exists\, ! f", rightarrow] \arrow[rru, "f_X", rightarrow, bend left] \arrow[drr, "f_Y"', rightarrow, bend right] & X \cross Y \arrow[ru, "\pi_X"', rightarrow] \arrow[rd, "\pi_Y", rightarrow] & \\
		& & Y
	\end{tikzcd}	
	\end{equation}
	i.e. $f_X = \pi_X \circ f$ and $f_Y = \pi_Y \circ f$.

	In particular, given $Z,X,Y$ affine varieties, there is a one to one correspondance
	\begin{equation}
		\begin{tikzcd}[row sep=tiny]
			\left\{\begin{matrix}
				\text{ morphisms of varieties }\\
				f: Z \to X \cross Y
			\end{matrix}\right\} \arrow[rr, "", leftrightarrow] & &
			\left\{  \begin{matrix}
				\ \text{pairs of morphisms }\\
				\left(Z \xrightarrow{f_X} X,\, Z \xrightarrow{f_Y} Y\right)
			\end{matrix}\right\} \\
	\end{tikzcd}
	.\end{equation} 
\end{prop} 

\begin{rem}
	\begin{equation}
	\mathbb{K}[X \cross Y] \cong \mathbb{K}[X] \otimes_\K \mathbb{K}[Y]
	.\end{equation} 
	This means that regular functions on $X \cross Y$ are finite linear combinations of products $f \cdot g$, with $f \in \mathbb{K}[X]$ and $g \in \mathbb{K}[Y]$. 
\end{rem}

\begin{defn}[Affine variety as ringed spaces]
	A ringed space $\left( X, \mathcal{O}_{ X } \right)$ is an \textbf{affine variety} over $\K$ iff $\mathcal{O}_X$ is a sheaf of $\K$-algebras and $\left( X, \mathcal{O}_{ X } \right)$ is isomorphic to an affine variety $Y \subset \mathbb{A}^n(\K)$.
\end{defn}

\begin{thm}[]
	An algebra $A$ over the field $\K$ is isomorphic to the coordinate ring $\mathbb{K}[X]$ of an affine variety $X$ iff $A$ is an integral domain, finitely generated, as a $\K$-algebra.
\end{thm}

\begin{rem}
	We have an equivalence of categories:
	\begin{equation}
	\begin{tikzcd}[row sep=tiny]
			\left\{\begin{matrix}
				\text{ Affine }\\
				\text{ varieties }
			\end{matrix}\right\} \arrow[rr, "", leftrightarrow] & &
			\left\{  \begin{matrix}
				\text{ finitely generated } \K\text{-algebras }\\
				\text{ that are integral domains}
			\end{matrix}\right\} \\
			\left( X, \mathcal{O}_{ X } \right) & & A \arrow[ll, "", rightarrow, maps to]
	\end{tikzcd}
	,\end{equation} 
	with $\mathcal{O}_X(X) = A$ and $\mathcal{O}_X(U) = \left\{ \phi \in \mathcal{Q}(A) \ \middle|\ \phi \text{ is regular at } p \text{ for any } p \in U \right\}$ for any $U \stackrel{\text{open}}{\subset} X$.
\end{rem}

\begin{rem}
	One can prove that an algebra $A$ is $A \cong \mathbb{K}[X]$ for some algebraic set $X \subset \mathbb{A}^n$ iff $A$ is a finitely generated $\K$-algebra with no nilpotent elements (i.e. it is a \textbf{reduced} algebra).

	Recall that $f$ is nilpotent iff $\exists\, r \in \N$ s.t. $f^r = 0$.
\end{rem}

\begin{ex}[An affine variety from the new definition]
	Let $X$ be an affine variety and $f \neq 0$ a fixed regular function on $X$.
	Clearly $\left( X_f, \mathcal{O}_{ X_f } \right)$, with $\mathcal{O}_{X_f} := \left.\mathcal{O}_X\right|_{X_f}$ is a ringed space.
	$A := \mathcal{O}_X(X_f)$ is a finitely generated algebra.
	Then $X_f$ is an affine variety with coordinate ring $\mathbb{K}[X]_f$.
\end{ex} 

\begin{rem}
	Not all open subsets of an affine variety are affine varieties.
	For example $U := \mathbb{A}^2 \setminus \left\{ \left(0, 0 \right) \right\}$ is not an affine variety.

	Though we can construct $U$ by patching together two affine varieties: 
	$\mathbb{A}^{2}_x$ and $\mathbb{A}^{2}_y$.
	In fact any open subset of an affine variety can be covered by a finite family
	of distinguished open subsets (which are affine varieties indeed).
\end{rem}


\section{Prevarieties}

\begin{defn}[Prevariety]
	A \textbf{prevariety} over $\K$ is a ringed space $\left( X, \mathcal{O}_{ X } \right)$ satisfying
	\begin{enumerate}
		\item $X$ is irreducible,
		\item $\mathcal{O}_X$ is a \textbf{sheaf} of $\K$-algebras s.t. the elements of $\mathcal{O}_X(U)$ are $\K$-valued functions on $U$, hence the restriction morphisms are restrictions of functions,
		\item there is a \textbf{finite} open conver $\left\{ U_i \right\}_{i \in I}$ of $X$ s.t., for all $i$, $\left(U_i, \left.\mathcal{O}_X\right|_{U_i} \right)$ is an affine variety.
	\end{enumerate}
	Open subsets $U \subset X$ which are isomorphic to affine varieties are called affine open sets.
	Morphisms of prevarieties are just morphisms of ringed spaces.
\end{defn}
\begin{rem}[]
	An open cover $\left\{ U_i \right\}_{i \in I}$ of $X$,
	in which each $\left( U_i, \left.\mathcal{O}_{ X }\right|_{U_i}  \right)$ is an affine variety,
	is called \textbf{affine open} cover of $X$.
\end{rem}


\begin{ex}\leavevmode\vspace{-.2\baselineskip}
	\begin{enumerate}
		\item Affine varieties,
		\item open subsets of affine varieties (sometimes called \textit{quasi-affine varieties}), which are finite unions of distinguished open subsets.
	\end{enumerate}
\end{ex} 

\begin{defn}[Gluing of prevarieties]
	Consider two prevarieties $X_1$ and $X_2$, and $U_1 \stackrel{\text{open}}{\subset} X_1$, $U_2 \stackrel{\text{open}}{\subset} X_2$ with an isomorphism $f: U_1 \to U_2$.
	We define
	\begin{equation}
		X = X_1 \cup_{f} X_2 := \left(X_1 \sqcup X_2 \right) / \mathcal{R}_f
	,\end{equation} 
	where $\mathcal{R}_f$ is the equivalence relation defined by
	\begin{equation}
		\mathcal{R}_f := \left\{ \left(x, x\right) \ \middle|\ x \in X_1 \sqcup X_2 \right\} \cup 
		\left\{ \left(x, f(x)\right), \left(f(x), x \right) \ \middle|\ x \in U_1 \right\}
	.\end{equation} 
	Finally we endow $X$ with the quotient topology.

	We now define a \textbf{structure sheaf} on $X$ to make it a prevariety
	\begin{equation}
		\mathcal{O}_X(U) := \left\{ \left(\phi_1, \phi_2\right) \in  \mathcal{O}_{X_1}
		\left( X_1 \cap_{} U \right) \cross \mathcal{O}_{X_2} \left( X_2 \cap_{} U \right)
		\ \middle|\ \left.\phi_1\right|_{U_1 \cap U}
		= \left.f^* \phi_2\right|_{U_2 \cap U}  \right\}
	.\end{equation} 
	(Explicitly it just states that, on the intersection of the union over $f$, the two functions have to coincide).
\end{defn}

\begin{ex}[$\mathbb{P}^1(\K)$ the projective line]
	Let $X_1 = X_2 := \mathbb{A}^1 \supset \mathbb{A}^1 \setminus \left\{ 0 \right\} =: U_1 = U_2$ 
	and
	 \begin{align}
		f: U_1 &\to U_2 \\
		x &\mapsto \frac{1}{x}
	.\end{align} 
	We then define the projective space
	 \begin{equation}
		 \mathbb{P}^1(\K) := \mathbb{P}^1 := \mathbb{A}^1 \cup_{f} \mathbb{A}^1
	.\end{equation} 
	Set theoretically $\mathbb{P}^1 = \mathbb{A}^1 \cup_{} \left\{ \infty \right\} $.
\end{ex} 
\begin{rem}
	The set of fixed points, by a morphism of prevarieties, can be an open subset of a variety, a different behaviour from what we'd expect from morphisms of algebraic sets.
\end{rem} 

\begin{lem}[gluing of prevarieties]
	Let $X_1, \ldots, X_r$ be \textbf{prevarieties},
	$\left\{ \left(U_{ij}, f_{ij}\right) \right\}_{1 \leq i, j \leq r}$ with $U_{ij} \stackrel{\text{open}}{\subset} X_i$, $f_{ij}: U_{ij} \xrightarrow{\cong} U_{ji}$ s.t.
	\begin{enumerate}
		\item $U_{ii} = X_i$ and $f_{ii} = id_{X_i}$ for any $i$,
		\item $f_{ij} \left( U_{ij} \cap_{} U_{ik} \right) \subset U_{jk}$ and $f_{jk} \circ f_{ij} = f_{ik}$ for all $i,j,k \in \left\{ 1, \ldots, r \right\}$.
	\end{enumerate}
	As a special case for the second condition we obtain that $f_{ij}^{-1} = f_{ji}$.

	Then there exists a unique prevariety $X$ up to isomorphism, obtained by gluing $X_1, \ldots, X_r$ along the $U_{ij}$ via the isomorphisms $f_{ij}$.
\end{lem} 

\subsection{Regular functions on prevarieties}
\begin{prop}
	$\mathcal{O}_{\mathbb{P}^1}(\mathbb{P}^1) = \left\{ \phi: \mathbb{P}^1 \to \K \ \middle|\ \phi \text{ is constant } \right\} \cong \K$.
\end{prop} 
\begin{cor}
	$\mathbb{P}^1$ is not an \textbf{affine variety}.
	In fact, for $X$ affine we have $\mathbb{K}[X] \cong \K \iff X = \left\{ pt \right\}$.
	(By gluing prevarieties together we can obtain spaces which are not affine varieties).
\end{cor} 
\begin{rem}
	This is analogous to the theory of holomorphic functions (again).
	(The only holomorphic functions on the Riemann sphere are the constant functions).
\end{rem} 
\begin{lem}
	Let $f: X \to Y$ a set theoretic map between two prevarieties $X$ and $Y$.
	If there exists an open cover $\left\{ U_1, \ldots, U_r \right\}$ of $X$ and an affine open cover $\left\{ V_1, \ldots, V_r \right\}$ of $Y$ s.t.
	\begin{equation}
	\begin{cases}
		f \left( U_i \right) \subset V_i\\
		\left( \left.f\right|_{U_i} \right)^* \phi \in \mathcal{O}_X(U_i)
	\end{cases} 
	\end{equation} 
	for all $i$ and all $\phi \in \mathcal{O}_{Y} \left( V_i \right)$, then $f$ is a morphism of prevarieties.
\end{lem} 

\subsection{Geometrical remarks}
\begin{lem}
	Given a prevariety $\left( X, \mathcal{O}_{ X } \right)$, then $X$ is a \textbf{noetherian} topological space.
\end{lem} 

\begin{defn}[Dimension of a prevariety]
	Given a \textbf{prevariety} $\left( X, \mathcal{O}_{ X } \right)$, we proved that $X$ is a Noetherian topological space, hence we can consider $\dim X$, as usual, as the largest integer $d$ s.t. we can construct a chain of irreducible closed subsets of $X$ 
	\begin{equation}
	\emptyset \subsetneq X_0 \subsetneq X_1 \subsetneq \ldots \subsetneq X_d = X
	.\end{equation} 
\end{defn}

\begin{lem}
	Any open subset $U$ of a prevariety $\left( X, \mathcal{O}_{ X } \right)$ is itself a prevariety, with structure sheaf $\left.\mathcal{O}_{X}\right|_{U}$.
\end{lem} 

\begin{defn}[Subprevarieties]
	Given a prevariety $\left( X, \mathcal{O}_{ X } \right)$ we define its subprevarieties corresponding to:
	\begin{itemize}
		\item $U \stackrel{\text{open}}{\subset} X$, then $\left( U, \left.\mathcal{O}_{ X }\right|_{U} \right)$ is a prevariety and it is called \textbf{open subprevariety} of $X$,
		\item $Y \subset X$ is a closed subset, we define a a sheaf $\mathcal{O}_{Y}$ on $Y$ by setting, on $U \stackrel{\text{open}}{\subset} Y$,
			\begin{align}
			\mathcal{O}_{Y} \left( U \right) :=
			\big\{ \phi: U \to \K \ \big|\ &\,\forall\, x \in U,
				\exists\, x \in V_x \stackrel{\text{open}}{\subset} X,
				\psi_x \in \mathcal{O}_{X} \left( V_x \right)\\
				&\text{ s.t. } \left.\phi\right|_{U \cap_{} V_x} =
				\left.\psi_x\right|_{U \cap_{} V_x}  \big\}
			.\end{align} 
			$\left( Y, \mathcal{O}_{ Y } \right)$ is called \textbf{closed subprevariety} of $X$,
		\item Given $Y \stackrel{\text{open}}{\subset} X \stackrel{\text{closed}}{\supset} Y$, then, by a combination of the previous constructions, $U \cap_{ } Y$ is a prevariety.
			It is called \textbf{locally closed subprevariety} of $X$.
	\end{itemize}
\end{defn}

\begin{defn}[Product of prevarieties]
	Given $\left( X, \mathcal{O}_{ X } \right)$, $\left( Y, \mathcal{O}_{ Y } \right)$ prevarieties, we define their product, as prevarieties, as the prevariety $P$, with the two morphisms
	\begin{equation}
	\begin{tikzcd}
		& P \arrow[ld, "\pi_X"', rightarrow] \arrow[rd, "\pi_Y", rightarrow] & \\
		X & & Y
	\end{tikzcd}
	\end{equation} 
	satisfying the universal property for pruducts, i.e.
	for every prevariety $Z$ with morphisms
	\begin{equation}
	\begin{tikzcd}
		& Z \arrow[ld, "f_X"', rightarrow] \arrow[rd, "f_Y", rightarrow] \\
		X & & Y
	\end{tikzcd}
	.\end{equation} 
	then there exists a unique morphism $f: Z \to P$ s.t. the following diagram commutes
	\begin{equation}
	\begin{tikzcd}
		& & X\\
		Z \arrow[r, "\exists\, ! f", rightarrow] \arrow[rru, "f_X", rightarrow, bend left] \arrow[drr, "f_Y"', rightarrow, bend right] & P \arrow[ru, "\pi_X"', rightarrow] \arrow[rd, "\pi_Y", rightarrow] & \\
		& & Y
	\end{tikzcd}	
	\end{equation}
	We can construct the product prevariety $\left( P, \mathcal{O}_{ P } \right)$ as the topological space $P := X \cross Y$.
	The structure of prevariety, then, is obtained from $X$ and $Y$ considering their open covers
	$\left\{ U_i \right\}_{i \in I}$ and $\left\{ V_j \right\}_{j \in J}$,
	where $U_i$ and $V_j$ are affine varieties, respectively for each $i$ and $j$.
	In fact, then $\left\{ U_i \cross V_j \right\}_{\left(i, j\right) \in I \cross J}$ is an open cover for $P$,
	in which every element is an affine variety (the product variety).
	We can now obtain $P$ by gluing all of these affine varieties (viewed as prevarieties) together.
\end{defn}

\subsection{Abstract varieties}
\begin{defn}[Variety]
	Let $X$ be a prevariety.
	We say that $X$ is \textbf{separated} iff, for every prevariety $Y$ and morphisms $f_1, f_2: Y \to X$ we have that
	\begin{equation}
	\left\{ p \in Y \ \middle|\ f_1(p) = f_2(p) \right\} \stackrel{\text{closed}}{\subset} Y
	.\end{equation}
	Separated prevarieties are called \textbf{varieties}.
\end{defn}

\begin{rem}
	Given a variety $X$ and two morphisms $Y\to X$ that coincide on a dense subset of $Y$, then they coincide on all of $Y$.
\end{rem}

\begin{defn}[Diagonal morphism]
	Let $X$ be a \textbf{prevariety}, then we have the diagonal morphism
	\begin{align}
		\Delta: X &\to X \cross X \\
		p &\mapsto \left(p, p\right)
	.\end{align} 
	Its image, $\Delta(X)$, is the diagonal in $X \cross X$.
	(Notice: $\Delta$, in universal property's terms is given by the morphisms $\left(id_X, id_X\right)$).
\end{defn}

\begin{lem}
	A prevariety $X$ is \textbf{separated} iff $\Delta(X)$ is closed in $X \cross X$.
	Often this is taken as the definition of separatedness.
\end{lem} 

\begin{rem}
	There is a similar result for \textbf{Hausdorff} spaces:
	a topological space $X$ is Hausdorff iff its diagonal is closed in $X \cross X$ (taken with the product topology).
	Notice that, in our situation, the topology on $X \cross X$ is not the product topology.
\end{rem}

\begin{lem}\leavevmode\vspace{-.2\baselineskip}
	\begin{itemize}
		\item Every \textbf{affine} variety is a \textbf{variety}.
		\item If $X$ ia a variety, then every locally closed subprevariety of $X$ is a variety.
	\end{itemize}
\end{lem} 

\section{Projective spaces}
\begin{defn}[Projective space]
	Let $\K$ be a field.
	The projective space of dimension $n$ over the field $\K$, denoted by $\mathbb{P}^n(\K) = \mathbb{P}^n$, is the set of all $1$-dimensional linear subspaces in $\K^{n+1}$.
\end{defn}

\begin{rem}
	Each $1$-dimensional linear subspace of $\K^{n+1}$ is spanned by a nonzero vector $\mathbf{v} \in \K^{n+1}$,
	and two vectors $\mathbf{v}, \mathbf{w} \in \K^{n+1} \setminus \left\{ \mathbf{0} \right\}$ have the same span iff
	there exists $\lambda \in \K$ s.t. $\mathbf{v} = \lambda \mathbf{w}$.
	Hence, set theoretically,
	 \begin{equation}
		 \mathbb{P}^n = \left( \K^{n+1} \setminus \left\{ \mathbf{0} \right\} \right)/\sim
	,\end{equation} 
	where $\left( v_0, \ldots, v_n \right) \sim \left( w_0, \ldots, w_n \right)$ if there is $\lambda \in \K \setminus \left\{ \mathbf{0} \right\}$ s.t. $\lambda w_i = v_i$ for every $i = 0, \ldots, n$.

	The equivalence classe of $\mathbf{v} = \left( v_0, \ldots, v_n \right)$ is denoted as
	\begin{equation}
	\left[ \mathbf{v} \right] = \left( v_0 : v_1 : \ldots : v_n \right) = \left[ v_0, \ldots, v_n \right]
	.\end{equation} 
\end{rem}

\begin{ex}
	For $\mathbb{P}^1$ we can see its points with
	\begin{align}
		U_0 := \left\{ \left[ p_0, p_1 \right] \in \mathbb{P}^1 \ \middle|\ p_0 \neq 0 \right\} &\to \mathbb{A}^1 \\
		\left[ p_0, p_1 \right] &\mapsto \frac{p_1}{p_0}\\
		\left[ 1, t \right] &\mapsfrom t
	,\end{align} 
	which is a bijection, and with the points $\left[ 0, p_1 \right]$ with $p_1 \neq 0$, which is the unique point $\left[ 0, 1 \right]$ at infinity.
	In particular 
	\begin{equation}
	\mathbb{P}^1 = \mathbb{A}^1 \cup_{} \left\{ \left[ 0, 1\right] \right\} 
	.\end{equation} 
\end{ex} 

\begin{rem}
	Analogously $\mathbb{P}^n$ can be covered by $n+1$ copies of $\mathbb{A}^n$.
	In particular we have the $n+1$ bijections
	\begin{align}
		U_0 := \left\{ \left[ p_0, \ldots, p_n \right] \in \mathbb{P}^n \ \middle|\ p_0 \neq 0 \right\} &\to \mathbb{A}^n \\
		\left[ p_0, p_1, \ldots p_n \right] &\mapsto \left( \frac{p_1}{p_0}, \ldots, \frac{p_n}{p_0} \right)\\
		\left[ 1, t_1 \ldots, t_n \right] &\mapsfrom \left( t_1, \ldots, t_n \right)
	.\end{align} 
	These give rise to the decomposition
	\begin{equation}
	\mathbb{P}^n = U_0 \cup_{} \left\{ \left[ 0, p_1 , \ldots , p_n \right] \right\} \cong \mathbb{A}^n \cup_{} \mathbb{P}^{n-1} 
	.\end{equation} 
	As expected $\mathbb{P}^n$ can be covered by the open subsets $U_i$, with $i+1$-th coordinate equal to one.
	Study what happens on the overlap $U_i \cap_{} U_j$, for $i \neq j$.
\end{rem}

\begin{rem}
	$\mathbb{P}^n(\R)$ and $\mathbb{P}^n(\C{})$ are \textbf{compact}, when endowed with the quotient topology of the standard topology on $\R^{n+1} \setminus \left\{ \mathbf{0} \right\}$ and $\C{n+1}\setminus \left\{ \mathbf{0} \right\}$ respectively.
\end{rem}

\subsection{Projective algebraic sets}
\begin{defn}[Homogeneous polynomials of degree $d$]
	We define the set
	\begin{equation}
	\mathbb{K}\left[x_1, \ldots, x_n \right]_d := \mathrm{Span}\, \left\{ x_0^{i_0}x_1^{i_1}\cdot \ldots \cdot x_n^{i_n} \ \middle|\ i_0 + i_1 + \ldots + i_n = d \right\}
	.\end{equation} 
	The elements of this set are called \textbf{homogeneous polynomials} of degree $d$.
\end{defn}

\begin{rem}
	$f \in \mathbb{K}\left[x_1, \ldots, x_n \right]_d$ implies, for any $\lambda \in \K$,
	 \begin{equation}
	f \left( \lambda x_0, \ldots, \lambda x_n \right) = \lambda^d f \left( x_0, \ldots, x_n \right)
	.\end{equation} 
	In particular the vanishing of $f$ gives a well-defined locus in $\mathbb{P}^n$.
\end{rem}

\subsection{Graded rings}
\begin{defn}[Graded ring]
	A \textbf{graded ring} $R$ is a ring with a decomposition
	\begin{equation}
	R = \bigoplus_{d \in \N} R_d
	,\end{equation} 
	with $R_d$ an abelian subgroup of $R$ for any $d \in \N$, satisfying the compatibility condition with the product:
	$f_d \cdot f_e \in R_{d+e}$ for all $f_d \in R_d$ and $f_e \in R_e$.
\end{defn}

\begin{defn}[Graded algebra]
	A graded $\K$-algebra $R$ is a graded ring which is also a $\K$-algebra and s.t. the $R_d$ are also linear subspaces.
\end{defn}

\begin{rem}
	Notice that 
	 \begin{equation}
	\mathbb{K}\left[x_1, \ldots, x_n \right] = \bigoplus_{d \in \N} \mathbb{K}\left[x_1, \ldots, x_n \right]_d
	\end{equation} 
	is a graded $\K$-algebra.
\end{rem}

\begin{defn}[Homogeneous element]
	Given a graded ring $R$, an element $f \in R$ is called \textbf{homogeneous} iff $\exists\, d \in \N$ s.t. $f \in R_d$.
	Since $R = \oplus_{d \in \N} R_d$, then any $f \in R$ has a \textit{unique} decomposition
	\begin{equation}
	f = \sum_{m \in \N}^{}  f_m
	,\end{equation} 
	with $f_m \in R_m$.
	The various  $f_m$ are called the \textbf{homogeneous components} of $f$.
\end{defn}

\begin{defn}[degree of an element]
	Given $0 \neq f \in R$ a graded ring, with the decomposition $\sum_{m \in \N}^{} f_m$ as above, the \textbf{degree} of $f$ is the maximal $m$ s.t. $f_m \neq 0$.
	Equivalently, if $d = \deg f$, then
	\begin{equation}
	f = f_0 + \ldots + f_d
	,\end{equation} 
	for $f_j \in R_j$.
\end{defn}

\begin{defn}[Homogeneous ideal]
	An ideal $\mathcal{I}$ in a graded ring $R$ is called \textbf{homogeneous} iff it is generated by homogeneous elements.
\end{defn}

\begin{lem}
	Let $\mathcal{I}, \mathcal{J}$ be ideals in a graded ring $R$.
	\begin{enumerate}
		\item $\mathcal{I}$ is \textbf{homogeneous} iff for all $f \in \mathcal{I}$, with decomposition in homogeneous components
			 \begin{equation}
			f = \sum_{d \in \N}^{} f_d
			\end{equation} 
			then $f_d \in R_d \cap_{} \mathcal{I}$ for all $d$.
		\item If $\mathcal{I}$ and $\mathcal{J}$ are both homogeneous ideals, then also $\mathcal{I} + \mathcal{J}$, $\mathcal{I} \cdot \mathcal{J}$ and $\mathcal{I} \cap_{} \mathcal{J}$ are homogeneous.
			Furthermore $\sqrt{\mathcal{J}}$ is homogeneous if $\mathcal{J}$ is.
		\item If $\mathcal{I}$ is homogeneous, then
			 \begin{equation}
				 R/\mathcal{I} = \bigoplus_{d \in \N} R_d / \left( R_d \cap_{} \mathcal{I} \right)
			\end{equation} 
			gives $R/\mathcal{I}$ the structure of graded ring.
	\end{enumerate}
\end{lem} 

\subsection{Projective algebraic sets}
\begin{defn}[Projective algebraic set]
	For any set $S \subset \mathbb{K}\left[x_0, \ldots, x_n \right]$ of \textit{homogeneous} polynomials we define the  \textbf{projective zero locus} of $S$ by
	\begin{equation}
		\mathbb{V}_p\left( S \right) := \left\{ x \in \mathbb{P}^n \ \middle|\ f(x) = 0 \text{ for all } f \in S \right\} \subset \mathbb{P}^n
	.\end{equation} 
	Analogously, for any homogeneous ideal $\mathcal{I}$ we set
	\begin{equation}
		\mathbb{V}_p\left( \mathcal{I} \right) := \left\{ x \in \mathbb{P}^n \ \middle|\ f(x) = 0 \text{ for all } f \in \mathcal{I} \right\} \subset \mathbb{P}^n
	.\end{equation} 
	Subsets of $\mathbb{P}^n$ of one of the above form are called \textbf{projective algebraic sets}.

	Furthermore, for a subset $X \subset \mathbb{P}^n$ we define the ideal of $X$ by
	\begin{equation}
		\mathbb{I}_p(X) := \left( f \in \mathbb{K}\left[x_0, \ldots, x_n \right] \ \middle|\ f \text{ is homogeneous, } f(x) = 0 \text{ for all } x \in X \right) \subset \mathbb{K}\left[x_0, \ldots, x_n \right]
	.\end{equation} 
\end{defn}

\begin{ex}\leavevmode\vspace{-.2\baselineskip}
	\begin{itemize}
		\item $\emptyset = \mathbb{V}_p\left( 1 \right)$ and $\mathbb{P}^n = \mathbb{V}_p\left( 0 \right)$ are projective algebraic sets.
		\item Given $L_0, \ldots, L_m$ homogeneous linear forms, then $\mathbb{V}_p\left( L_0, \ldots, L_m \right)$ is a projective algebraic set.
			In particular points in $\mathbb{P}^n$ are algebraic sets.
	\end{itemize}
\end{ex} 

\begin{prop}\leavevmode\vspace{-.2\baselineskip}
	\begin{enumerate}
		\item If $\left\{ \mathcal{I}_i \right\}_{i \in I} $ is a family of homogeneous ideals in $\mathbb{K}\left[x_0, \ldots, x_n \right]$, then
			\begin{equation}
			\bigcap_{i \in I} \mathbb{V}_p\left( \mathcal{I}_i \right) = \mathbb{V}_p\left( \bigcup_{i \in I} \mathcal{I}_i \right) \subset \mathbb{P}^n
			.\end{equation} 
			In other words, arbitrary intersections of projective algebraic sets are algebraic.
		\item If $\mathcal{I}_1, \mathcal{I}_2 \subset \mathbb{K}\left[x_1, \ldots, x_n \right]$ are homogeneous ideals, then
			\begin{equation}
			\mathbb{V}_p\left( \mathcal{I}_1 \right) \cup \mathbb{V}_p\left( \mathcal{I}_2 \right) = \mathbb{V}_p\left( \mathcal{I}_1 \cdot \mathcal{I}_2 \right) = \mathbb{V}_p\left( \mathcal{I}_1 \cap \mathcal{I}_2 \right) \subset \mathbb{P}^n
			.\end{equation} 
			As a consequence, finite union of algebraic sets are algebraic.
	\end{enumerate}
\end{prop} 

\begin{defn}[Zariski topology on $\mathbb{P}^n$]
	The \textbf{Zariski topology} on $\mathbb{P}^n$ is the topology whose closed subsets are precisely the projective algebraic sets.
	If $X \subset \mathbb{P}^n$ we define the Zariski topology on $X$ as the subset topology, induced by the Zariski topology on $\mathbb{P}^n$.
\end{defn}

\begin{rem}
	$\mathbb{P}^n$ is a \textbf{prevariety}, hence it is a \textbf{noetherian} topological space.
	In particular we can consider $\dim X$ for all closed subsets $X \subset \mathbb{P}^n$.
\end{rem} 

\subsection{Projective varieties}
\begin{defn}[Projective variety]
	A projective variety is a \textbf{Zariski closed} subset $X \subset \mathbb{P}^n$.
\end{defn}

\begin{defn}[Cone]
	An affine algebraic set $C \subset \mathbb{A}^{n+1}$ is called a \textbf{cone} iff 
	\begin{itemize}
		\item $C$ contains the origin,
		\item for any $x \in C$, then $\lambda x \in C$ for any $\lambda \in \K$.
	\end{itemize}
\end{defn}

\begin{defn}[Cone over an algebraic set]
	Given any $X \subset \mathbb{P}^n$ a projective algebraic set, then we can define the cone over $X$ as
	\begin{equation}
		C(X) := \left\{ \mathbf{0} \right\} \cup \left\{ x \in \mathbb{A}^{n+1}\setminus \left\{ \mathbf{0} \right\} \ \middle|\ \left[ x \right] \in X \right\}
	.\end{equation} 
	Notice that the second part of the union is the preimage of $X$ under $\mathbb{A}^{n+1}\setminus \left\{ \mathbf{0} \right\} \to \mathbb{P}^{n}$.
\end{defn}

\begin{lem}[Relation between $X$ and $C(X)$]\leavevmode\vspace{-.2\baselineskip}
	\begin{itemize}
		\item Let $X = \mathbb{V}_p\left( \mathcal{I} \right) \subset \mathbb{P}^{n}$ be defined by a homogeneous ideal $\mathcal{I} \subsetneq \left( 1 \right)$.
			Then 
			\begin{equation}
			C(X) = \mathbb{V}\left( \mathcal{I} \right) \subset \mathbb{A}^{n+1}
			.\end{equation}
		\item If $X \subset \mathbb{P}^{n}$ is a projective algebraic set, then
			\begin{equation}
				\mathbb{I}\left( C(X) \right) = \mathbb{I}_p(X)
			.\end{equation} 
	\end{itemize}
\end{lem} 

\subsection{Projective Nullstellensatz}
\begin{lem}
	Let $\mathcal{I} \subset \mathbb{K}\left[x_0, \ldots, x_n \right]$ be an homogeneous ideal.
	The following properties are equivalent:
	\begin{enumerate}
		\item $\mathbb{V}_p\left( \mathcal{I} \right) = \emptyset \subset \mathbb{P}^{n}$,
		\item either $\sqrt{\mathcal{I}} = \mathbb{K}\left[x_0, \ldots, x_n \right]$ or $\sqrt{\mathcal{I}} = \left( x_0, \ldots, x_n \right)$,
		\item there is $d \geq 1$ a degree, s.t. $\mathbb{K}\left[x_0, \ldots, x_n \right]_d \subset \mathcal{I}$.
	\end{enumerate}
\end{lem} 

\begin{thm}[Projective Nullstellenstaz]\leavevmode\vspace{-.1\baselineskip}\newline There is a one to one inclusion reversing correspondence
	\begin{equation}
	\begin{tikzcd}[row sep=tiny]
			\left\{\begin{matrix}
				\text{ algebraic sets }\\
				\text{ in } \mathbb{P}^{n}(\K)
			\end{matrix}\right\} \arrow[rr, "", leftrightarrow] & &
			\left\{  \begin{matrix}
				\text{ homogeneous }\\
				\text{ radical ideals }\\
				\mathcal{I} \neq \left( x_0, \ldots, x_n \right)_d
			\end{matrix}\right\} \\
			X \arrow[rr, "", rightarrow, maps to] & & \mathbb{I}_p(X)\\
			\mathbb{V}_p\left( \mathcal{I} \right) & & \mathcal{I} \arrow[ll, "", rightarrow, maps to]
	\end{tikzcd}
	.\end{equation} 
	Furthermore, for all subsets $X \subset \mathbb{P}^{n}$ and all homogeneous ideals $\mathcal{I}$ with $\mathbb{V}_p\left( \mathcal{I} \right) \neq \emptyset$, we have
	\begin{align}
		\mathbb{I}_p \left( \mathbb{V}_p\left( \mathcal{I} \right) \right) &= \sqrt{\mathcal{I}}\\
		\mathbb{V}_p\left( \mathbb{I}_p(X) \right) &= \overline{X}
	.\end{align} 
\end{thm}

\subsection{Regular functions on projective varieties}

\begin{defn}[Homogeneous coordinate ring]
	Given $X \subset \mathbb{P}^{n}$ a projective algebraic set, then the $\mathbb{K}$-algebra
	\begin{equation}
		S(X) := \frac{\mathbb{K}\left[x_0, \ldots, x_n \right]}{\mathbb{I}_p\left( X \right)}
	\end{equation} 
	is called the \textbf{homogeneous coordinate ring} of $X$.
\end{defn}

\begin{rem}
	Elements of $S(X)$ are functions on the cone $C(X)$, but not on $X$
	(they are not constant up to multiplication by a scalar).
	However quotients $\frac{f}{g}$ with $\deg f = \deg g$ give well-defined functions outside $\mathbb{V}\left( g \right)$.
	In fact
	\begin{equation}
	\frac{f \left( \lambda x_0, \ldots, \lambda x_n \right)}{ g \left( \lambda x_0, \ldots, \lambda x_n \right)} = \frac{\lambda^d f \left( x_0, \ldots, x_n \right)}{\lambda^d g \left( x_0, \ldots, x_n \right)} = \frac{f \left( x_0, \ldots, x_n \right)}{g \left( x_0, \ldots, x_n \right)}
	\end{equation} 
	hold for all $\lambda \neq 0$ and $ \left[ x_0 , \ldots , x_n \right] \in X$.

	Let's now introduce the notation $S(X)_d = \mathbb{K}\left[x_0, \ldots, x_n \right]_d / \mathcal{I}_d$.
\end{rem}

\begin{defn}[Field of rational functions]
	Let $X \subset \mathbb{P}^{n}$ be a projective variety.
	The \textbf{field of rational functions} on $X$ is
	\begin{equation}
		\K(X) := \left\{ \frac{f}{g} \ \middle|\ f,g \in S(X)_d, g \neq 0 \right\}
	.\end{equation} 
\end{defn}

\begin{defn}[Regular functions]\leavevmode\vspace{-\baselineskip}
	\begin{enumerate}
		\item We say that $\varphi \in \K(X)$ is \textbf{regular at} $p \in X$ iff
			\begin{equation}
				\varphi = \frac{f}{g} \in \K(X), \text{ with } g(p) \neq 0
			,\end{equation}
			i.e. $\varphi$ is well defined at $p$,
		\item We define the \textbf{local ring of} $X$ at $p$ to be
			\begin{equation}
				\mathcal{O}_{X,p} := \left\{ \varphi \in \K(X) \ \middle|\ \varphi \text{ is regular at } p \right\}
			,\end{equation}
		\item For any $\emptyset \neq U \stackrel{\text{open}}{\subset} X$, we define
			\begin{equation}
				\mathcal{O}_X (U) := \bigcap_{p \in U} \mathcal{O}_{X,p}
			,\end{equation}
			the ring of regular functions on $U$.
	\end{enumerate}
\end{defn}

\begin{rem}
	We can show that only the constant functions on $\mathbb{P}^{n}$ are regular on the whole $\mathbb{P}^{n}$, hence that
	\begin{equation}
	\mathcal{O}_{\mathbb{P}^{n}} \left( \mathbb{P}^{n} \right) = \K
	.\end{equation} 
\end{rem}

\begin{prop}
	Let $X \subset \mathbb{P}^{n}$ be a projective variety.
	Then $\left( X, \mathcal{O}_{ X } \right)$ is a prevariety.
\end{prop} 
\begin{rem}
	From the proof of the proposition we obtain a one-to-onte correspondance
	\begin{equation}
	\begin{tikzcd}[row sep=tiny]
			\left\{\begin{matrix}
				\text{ affive viarieties }\\
				\text{ in } \mathbb{A}^{n}(\K)
			\end{matrix}\right\} \arrow[rr, "", leftrightarrow] & &
			\left\{  \begin{matrix}
				\text{ projective varieties in } \mathbb{P}^{n} \text{ not }\\
				\text{ contained in the hyperplane at } \infty\\
			\end{matrix}\right\}
	\end{tikzcd}
	.\end{equation} 
\end{rem}

\begin{rem}
	All known constructions for prevarieties apply also to projective varieties.
	In particular we know how to define
	\begin{itemize}
		\item morphisms,
		\item products,
		\item the function field of $X$ as a prevariety, which asctually is the same thing as the definition $\K(X)$.
	\end{itemize}
\end{rem}

Let's now consider, for simplicity, $X_j := X \cap U_j \cong Y_j \subset \mathbb{A}^{n}$, with $j = 1$.
\begin{defn}[Homogenization of a polynomial]
	Let $f \in \mathbb{K}\left[x_1, \ldots, x_n \right], g \in \mathbb{K}\left[x_0, \ldots, x_n \right]_d$.
	The \textbf{homogenization} of $f$, with respect to $x_0$, is
	\begin{equation}
		{}^h f \left( x_0, \ldots, x_n \right) := x_0^{\deg f} f \left( \frac{x_1}{x_0}, \ldots, \frac{x_n}{x_0} \right)
		\in \mathbb{K}\left[x_0, \ldots, x_n \right]_{\deg f}
	.\end{equation} 
	Conversely, the dehomogenization of $g$, with respect to $x_0$, is
	\begin{equation}
		{}^a g \left( x_1, \ldots, x_n \right) := g \left( 1, x_1, \ldots, x_n \right) \in \mathbb{K}\left[x_1, \ldots, x_n \right]
	.\end{equation} 
\end{defn}

\begin{defn}[Homogenization of an ideal]
	Let $\mathcal{I} \subset \mathbb{K}\left[x_1, \ldots, x_n \right]$ be an ideal.
	We define its \textbf{homogenization} by
	\begin{equation}
		{}^h \mathcal{I} := \left( {}^h f \ \middle|\ f \in \mathcal{I} \right)
	.\end{equation} 
\end{defn}

\begin{rem}
	It is easy to check that ${}^h f$ is a homogeneous polynomial of degree $\deg {}^h f = \deg f$.
	With regards to the dehomogenization, instead, the degree might decrease, hence
	$\deg {}^a g \leq \deg g$, with $<$ if $x_0$ divides $g$.

	Moreover it can be easily checked that
	\begin{equation}
		{}^a \left( {}^h f \right) = f \quad \text{ and } \quad {}^h \left( {}^a g \right) = g \text{ iff } x_0 \nmid g
	.\end{equation} 
\end{rem}

\begin{lem}\leavevmode\vspace{-.2\baselineskip}
	\begin{itemize}
		\item For $X = \mathbb{V}\left( \mathcal{I} \right) \subset \mathbb{A}^{n}$ we have
			$\mathbb{V}_p\left( {}^h \mathcal{I} \right) = \overline{X}$, where
			$\overline{X}$ is the projective closure of $X$, defined as the Zariski closure of
			$X \subset \mathbb{A}^{n} \cong U_0$ in $\mathbb{P}^{n}$.
		\item If $X = \mathbb{V}\left( f \right)$, then $\overline{X} = \mathbb{V}_p\left( {}^h f \right)$.
		\item The map $\mathcal{I} \to {}^h \mathcal{I}$ induces a one-to-one correspondance
	\begin{equation}
	\begin{tikzcd}[row sep=tiny]
			\left\{\begin{matrix}
				\text{ Zariski closed }\\
				\text{ subsets of } \mathbb{A}^{n}(\K)
			\end{matrix}\right\} \arrow[rr, "", leftrightarrow] & &
			\left\{  \begin{matrix}
				\text{ Zariski closed subsets of } \mathbb{P}^{n} \text{ s.t. no }\\
				\text{ irr. comp. is contained in } H_\infty = \mathbb{V}\left( x_0 \right)\\
			\end{matrix}\right\} \\
			\mathbb{V}\left( \mathcal{I} \right) \arrow[rr, "", leftrightarrow, maps to] & & \mathbb{V}\left( {}^h \mathcal{I} \right)\\
	\end{tikzcd}
	.\end{equation} 
	\end{itemize}
\end{lem}

\begin{rem}
	For hypersurfaces we just proved that
	\begin{equation}
		X = \mathbb{V}\left( f \right) \implies \overline{X} = \mathbb{V}_p\left( {}^h f \right)
	.\end{equation} 
	For an arbitrary number of polynomials, instead, in general
	\begin{equation}
		\overline{\mathbb{V}\left( F_1, \ldots, F_r \right)} \neq \mathbb{V}_p\left( {}^h F_1, \ldots, {}^h F_r \right)
	.\end{equation} 
	Only for sufficiently good sets of generators the ideal generated by their homogeneization corresponds with the homogeneization of the ideal.
\end{rem}


\section{Morphisms and products of projective varieties}

\begin{defn}[Quasi-projective variety]
	Let $X \subset \mathbb{P}^{n}$.
	$X$ is a \textbf{quasi-projective} variety iff it is a dense open subset of a projective variety.
\end{defn}

\begin{rem}
	All quasi-projective varieties are prevarieties.
\end{rem}

\begin{lem}
	Let $X \subset \mathbb{P}^{n}$ be a quasi-projective variety.
	Let $F_0, \ldots, F_m \in \mathbb{K}\left[x_0, \ldots, x_n \right]_d$ s.t., for every $x \in X$, there is at least one $F_i$ with $F_i(x) \neq 0$.
	Then the $F_i$s define a map
	\begin{align}
		F: X &\to \mathbb{P}^{m} \\
		x &\mapsto \left[ F_0(x) , \ldots , F_m(x) \right]
	\end{align} 
	which is always a morphism of quasi-projective varieties.
\end{lem} 

\begin{rem}
	Not all morphisms $X \to \mathbb{P}^{m}$ can be obtained in this way.
\end{rem}

\begin{ex}
	The map
	\begin{align}
		f: \mathbb{P}^{1} &\to \mathbb{P}^{2} \\
		\left[ s , t \right] &\mapsto \left[ s^2, st, t^2 \right]
	\end{align} 
	is a morphism with image $X = \mathbb{V}_p\left( xz - y^2 \right)$.
	In fact it is an isomorphism on its image.
	This means that its inverse is also a morphism
	\begin{align}
		f^{-1}: X &\to \mathbb{P}^{1} \\
		\left[ x, y, z \right] &\mapsto 
		\begin{cases}
			\left[ x, y \right] & \text{ for } \left( x, y \right) \neq 0 \\
			\left[ y, z \right] & \text{ for } \left( y, z \right) \neq 0 
		\end{cases} 
	.\end{align} 
	However $f^{-1}$ cannot be defined globally by a pair of homogeneous polynomials in $x, y$ and $z$.
\end{ex} 

\begin{rem}
	As a corollary, the example shows that all irreducible conics in $\mathbb{P}^{2}$ are isomorphic to $\mathbb{P}^{1}$.
\end{rem}

In the following we'll denote $N_{n,m} := \left( n + 1 \right) \left( m+1 \right) - 1$.
\begin{defn}[Segre embedding]
	The \textbf{Segre embedding} is the morphism
	\begin{align}
		s_{n,m}: \mathbb{P}^{n} \cross \mathbb{P}^{m} &\to \mathbb{P}^{N_{n,m}} \\
		\left( \left[ x_0 , \ldots , x_n \right], \left[ y_0 , \ldots , y_m \right] \right) &\mapsto 
		\left[ z_{ij} \ \middle|\ 0 \leq i \leq n,\ 0 \leq j \leq m \right]
	,\end{align} 
	with $z_{ij} := x_iy_j$.
\end{defn}

\begin{lem}\leavevmode\vspace{-.2\baselineskip}
	\begin{enumerate}
		\item The image of $s_{n,m}$ is the projective variety $X \subset \mathbb{P}^{n}$ whose
	ideal is generated by
	\begin{equation}
	z_{ij} z_{kl} - z_{il} z_{kj} \text{ for all } 0 \leq i,k \leq n \text{ and } 0 \leq j, l \leq m
	.\end{equation} 
	$X$ is called the \textbf{Segre} variety.
		\item $s_{n,m}: \mathbb{P}^{n} \cross \mathbb{P}^{m} \to X$ is an isomorphism of prevarieties.
			In particular $\mathbb{P}^{n} \cross \mathbb{P}^{m}$ is a projective variety.
		\item The closed subsets of $\mathbb{P}^{n} \cross \mathbb{P}^{m}$ are the zero loci of polynomials in
			$\mathbb{K}\left[x_0, \ldots, x_n, y_0, \ldots, y_m \right]$ that are bihomogeneous in the $x$ and $y$ coordinates.
	\end{enumerate}
\end{lem} 		

\begin{rem}
	The ring $R := \mathbb{K}\left[x_0, \ldots, x_n, y_0, \ldots, y_m \right]$ is a bigraded ring, i.e.
	a graded ring over the monoid $\N \cross \N$.
	\begin{align}
		R &= \bigoplus_{(d,e) \in \N \cross \N} R_{d,e}\\
		R_{d,e} :&= \mathrm{span}_{\K}\, \left\{ x_0^{i_0} \ldots x_n^{i_n} 
		y_0^{j_0} \ldots y_m^{j_m} \ \middle|\ 
		i_0 + \ldots + i_n = d \text{ and } j_0 + \ldots + j_m = e \right\}
	.\end{align} 
	The elements of $R_{d,e}$ are called bihomogeneous polynomials of bidegree $(d,e)$.
	Moreover, as expected, when multiplying polynomials, the bidegrees are additive.
\end{rem}

\begin{ex}
	Let's compute the \textbf{Segre embedding} of $\mathbb{P}^{1} \cross \mathbb{P}^{1}$.
	\begin{equation}
	\mathbb{P}^{1} \cross \mathbb{P}^{1} \simeq \mathbb{V}_p\left( u_0 u_3 = u_1 u_2 \right) \subset \mathbb{P}^{3}
	.\end{equation} 
	Explicitly the embedding is given by
	\begin{equation}
	\left(\left[ x_0, x_1 \right], \left[ y_0, y_1 \right]\right) \mapsto \left[ x_0y_0, x_0y_1, x_1y_0, x_1y_1 \right]
	.\end{equation} 
	Any rank $4$ quadric in $\mathbb{P}^{3}$ is isomorphic to the Segre product $X$, after a linear change of coordinates.
	Hence it is covered by two families of (pairwise disjoint) lines.
	This is very different from $\mathbb{P}^{2}$, in which any two curves (and in particular any two lines) intersect.
	(In $\mathbb{P}^{1} \cross \mathbb{P}^{1}$ lines from the same family either coincide or do not intersect at all).
\end{ex} 

\begin{prop}[Corollary]
	Projective varieties are varieties (i.e. they are \textbf{separated} prevarieties).
\end{prop} 
\begin{rem}
	Since projective varieties are separated, also quasi-projective varieties
	(i.e. dense open subsets of projective varieties) are varieties.
\end{rem}

\section{Main theorem on projective varieties}

\begin{rem}
	Let's recall a few notions:
	\begin{itemize}
		\item $\mathbb{P}^{n}(\mathbb{C})$ is compact in the standard topology.
			This means that every projective variety is compact in the analytic topology.
		\item The image of a compact subset under a continuous map is compact.
	\end{itemize}
	Moreover, coming back to concepts concerning algebraic geometry:
	the image of an affine variety under a morphism may not be closed.
	An example is $\mathbb{A}^{2} \xrightarrow{\pi} \mathbb{A}^{1}$, then
	\begin{equation}
		\pi \left( \mathbb{V}\left( xy - 1 \right) \right) = \mathbb{A}^{1} \setminus \left\{ \mathbf{0} \right\}
	.\end{equation} 
	The first clearly is closed, whereas the latter is open.
\end{rem}

For the Zariski topology on projective varieties we will, instead, prove that the image of a projective variety under a morphism is always closed.
In other words the image of a projective variety under a morphism is always a projective variety.
(That's the reason why we needed to extend the definition of affine varieties to abstract varieties. In the case of projective ones we do not need to do so.)

\begin{thm}[Projection map is closed]
	Let $X \subset \mathbb{P}^{n} \cross \mathbb{P}^{m}$ be a closed subset.
	Let $\pi: \mathbb{P}^{n} \cross \mathbb{P}^{m} \to \mathbb{P}^{n}$ be the canonical projection.
	Then $\pi(X) \subset \mathbb{P}^{n}$ is closed.
	In other words the projection map is closed.
\end{thm}

\begin{rem}
	This theorem has application in computational algebra.
	It is sometimes called the \textit{main theorem of elimination theory}.
	In elimination theory, starting from a variety (i.e. the zero locus of a finite collection of bihomogeneous polynomials)
	\begin{equation}
		X: F_1(x,y) = F_2(x,y) = \ldots = F_r(x,y) = 0
	.\end{equation} 
	Then the projection sens $X$ to
	\begin{equation}
		\pi(X) = \left\{ x \in \mathbb{P}^{n} \ \middle|\ \exists\, y \in \mathbb{P}^{m} \text{ s.t. } F_i(x,y) = 0 \,\forall\, i \right\}
	.\end{equation} 
	We are trying to remove the $y$ variables from the first system.
	Since $\pi(X)$ is closed, then it is the solution set of some polynomial equations.
\end{rem}

\begin{cor}
	The projection morphism $\pi: \mathbb{P}^{n} \cross Y \to Y$ is closed for all varieties $Y$.
\end{cor} 

\begin{defn}[Complete variety]
	A variety $X$ is \textbf{complete} iff the projection $\pi: X \cross Y \to Y$ is closed for any variety $Y$.
\end{defn}

\begin{ex}\leavevmode\vspace{-.2\baselineskip}
	\begin{itemize}
		\item $\mathbb{P}^{n}$ is complete, by the above corollary.
		\item Any projective variety $X \subset \mathbb{P}^{n}$ is complete:
			Any closed subset of $X \cross Y$ is also closed in $\mathbb{P}^{n} \cross Y$.
			Since the image by the projection is the same, then also its
			projection to $Y$ is closed.
		\item There also exist complete varieties that are not projective, but it is hard to prove this last statement.
	\end{itemize}
\end{ex} 

\begin{cor}
	Let $f: X \to Y$ be a morphism of varieties.
	Let $X$ be a complete variety, then
	$f(X)$ is closed in $Y$.
\end{cor} 
\begin{proof}
	The idea is to study an arbitrary map as a projection.
\end{proof}

Let's now see a few corollaries to the above theorem.

\begin{cor}
	Let $X \subset \mathbb{P}^{n}$ be a projective variety that contains more than one point (equivalently $\mathrm{dim}\, X \geq 1$).
	Then, for any non constant homogeneous polynomial $f \in \mathbb{K}\left[x_0, \ldots, x_n \right]$, $X \cap \mathbb{V}_p\left( f \right) \neq \emptyset$.
\end{cor} 

\begin{cor}
	Any two curves in $\mathbb{P}^{2}$ intersect.
	In particular $\mathbb{P}^{2}$ is not isomorphic to $\mathbb{P}^{1} \cross \mathbb{P}^{1}$ 
	(recall that $\mathbb{P}^{1} \cross \mathbb{P}^{1}$ is covered by a family of lines that do not intersect each other).
\end{cor}

\begin{cor}
	Every regular fucntion on a complete variety is constant, i.e. for any $X$ complete
	\begin{equation}
	\mathcal{O}_{X} \left( X \right) = \K
	.\end{equation} 
\end{cor} 

\subsection{Veronese variety}

Let $n, d \geq 1$.
Consider $\mathbb{K}\left[x_0, \ldots, x_n \right]_d$, a $\K$-Vector Space of dimension $\binom{n+d}{n}$.
Let $F_0, \ldots, F_\nu$ be a basis for this space, hence
\begin{equation}
	\nu = \nu_{n,d} := \binom{n+d}{n} - 1
.\end{equation} 
One could choose the monomials of degree $d$ (ordered in some way, for example lexicographically), as a basis.
For instance
\begin{equation}
F_0 = x_0^d, \qquad
F_1 = x_0^{d-1} x_1,\qquad \ldots,\qquad
F_{\nu} = x_n^d
.\end{equation} 

The basis induces a morphism of varieties
\begin{align}
	v_d: \mathbb{P}^{n} &\to \mathbb{P}^{\nu} \\
	\left[ x_0 , \ldots , x_n \right] &\mapsto \left[ F_0(x) , \ldots , F_\nu(x) \right]
.\end{align} 
By the main theorem on projective varieties, the image of this morphism, $V_{n,d} := v_d \left( \mathbb{P}^{n} \right)$, is a projective variety.
We call it the Veronese variety of degree $d$.

\begin{prop}
	The morphism $v_d: \mathbb{P}^{n} \to V_{n,d}$ is an isomorphism.
\end{prop} 
\begin{rem}
	When we view $\mathbb{P}^{n}$ as $V_{n,d} \subset \mathbb{P}^{\nu}$, then the degree $d$ 
	homogeneous polynomials in $x_0, \ldots, x_n$ give \textbf{linear} forms in the coordinates of $\mathbb{P}^{\nu}$.
\end{rem} 

\begin{ex}[$n = 1$: the rational normal curve of degree $d$]
	$V_{1,d} \subset \mathbb{P}^{d}$ is called a \textit{rational normal curve of degree} $d$ in $\mathbb{P}^{d}$.
	It is given by
	\begin{align}
		v_d: \mathbb{P}^{1} &\to \mathbb{P}^{d} \\
		\left[ x_0 , x_1 \right] &\mapsto \left[ x_0^d, x_0^{d-1}x_1 , \ldots , x_1^d \right] = 
		\left[ u_0, u_1 , \ldots , u_d \right]
	.\end{align} 
	As a subvariety of $\mathbb{P}^{d}$, $V_{1,d}$ is defined by the condition
	\begin{equation}
	\mathrm{rk}\, 
	\begin{pmatrix}
		u_0 & u_1 & \ldots & u_{d-1}\\
		u_1 & u_2 & \ldots & u_d
	\end{pmatrix} 
	= 1
	.\end{equation} 
\end{ex} 

\begin{cor}
	Let $X \subset \mathbb{P}^{n}$ be a projective variety and $f \in \mathbb{K}\left[x_0, \ldots, x_n \right]$ any non constant homogeneous polynomial.
	Then $X \setminus \mathbb{V}_p\left( f \right)$ is an affine variety.
\end{cor} 

\section{Dimension}
Recall that the dimension of a \textit{Noetherian} topological space $X$ is the largest
index $d$ for which there is exists a chain
\begin{equation}
\emptyset \subsetneq X_0 \subsetneq X_1 \subsetneq \ldots \subsetneq X_d = X
\end{equation} 
of irreducible closed subsets $X_i \subset X$.
We will call such chain a \textit{longest chain} in $X$.

\begin{rem}
	What do we know about dimension as of now?
	\begin{itemize}
		\item $\dim \left\{ pt \right\} = 0$.
		\item $\dim \mathbb{A}^{1} = \dim \mathbb{P}^{1} = 1$, since the only irreducible closed subsets are the points.
		\item \textit{Bezout's theorem}: two curves $\mathbb{V}\left( f \right), \mathbb{V}\left( g \right) \subset \mathbb{A}^{2}$ with no common irreducible component, 
			intersect in $\left( \deg f \right) \left( \deg g \right)$ points (counted with multiplicity).
			This means that the irreducible closed subsets of $\mathbb{A}^{2}$ are:
			\begin{itemize}
				\item $\mathbb{A}^{2}$,
				\item irreducible curves $X = \mathbb{V}\left( f \right)$, for $f \in \mathbb{K}\left[x, y \right]$ irreducible,
				\item points.
			\end{itemize}
			Hence longest chains are of the form
			\begin{equation}
			\emptyset \subsetneq \left\{ pt \right\} \subsetneq \mathbb{V}\left( f \right) \subsetneq \mathbb{A}^{2}
			,\end{equation} 
			with $f(p) = 0$ and $f$ is irreducible.
			This implies that $\dim \mathbb{A}^{2} = 2$.
			Since, moreover, $\mathbb{P}^{2} = \mathbb{A}^{2} \cup \mathbb{V}_p\left( x_0 \right)$, we obtain that $\dim \mathbb{P}^{2} = 2$.
		\item We also noticed that $\dim \mathbb{A}^{n} \geq n$, in fact we have the \textit{longest chain}
			\begin{equation}
			\emptyset \subsetneq \mathbb{V}\left( x_1, \ldots, x_n \right) \subsetneq
			\mathbb{V}\left( x_2, \ldots, x_n \right) \subsetneq \ldots \subsetneq
			\mathbb{V}\left( x_n \right) \subsetneq \mathbb{A}^{n}
			.\end{equation} 
			Analogously we can show that $\dim \mathbb{P}^{n} \geq n$.
	\end{itemize}
\end{rem}

\begin{lem}\leavevmode\vspace{-.2\baselineskip}
	\begin{enumerate}
		\item If $X$ is a Noehterian topological space and we consider a longest chain
			\begin{equation}
			\emptyset \subsetneq X_0 \subsetneq X_1
			\subsetneq \ldots \subsetneq X_n = X
			,\end{equation} 
			then $\dim X_j = j$ for all $j = 0, \ldots, n$.
		\item If $X$ is irreducible and $Y \subsetneq X$ is closed, then $\dim Y < \dim X$.
	\end{enumerate}
\end{lem} 

A comment on what's to come:
\begin{itemize}
	\item We'll prove that dimension is a local property, i.e. that given a variety $X$ and an affine open subset $U \subset X$, 
		then $\dim U = \dim X$.
		Moreover, for any $U \subset \mathbb{A}^{n}$, then $\dim Y = \dim U$, for $Y:= \overline{U} \subset \mathbb{P}^{n}$.
		This means that we can concentrate our study only on projective varieties.
	\item Given a surjective morphism $f: X \to Y$ of projective varieties, then $\dim Y \leq \dim X$.
		From this idea we get the expectation that, given $X$ a projective variety, then $\dim X$ is the largest integer $d$ 
		s.t. there is a surjective morphism $f: X \to \mathbb{P}^{d}$.
\end{itemize}

\begin{lem}
	Let $f: X \to Y$ be a surjective morphism of projective varieties.
	Then every longest chain
	\begin{equation}
	\emptyset \subsetneq Y_0 \subsetneq Y_1 \subsetneq
	\ldots \subsetneq Y_n = Y
	\end{equation} 
	can be lifted to a chain
	\begin{equation}
	\emptyset \subsetneq X_0 \subsetneq X_1 \subsetneq \ldots
	\subsetneq X_n = X
	\end{equation} 
	of irreducible closed subsets $X_j \subset X$, s.t. $f(X_j) = Y_j$ for all $j = 0, \ldots, n$.
	In particular $\dim X \geq \dim Y$.
\end{lem} 

\begin{rem}[Next step]
	Our next aim is to show that every projective variety $X \subset \mathbb{P}^{n}$ admits a surjective morphism to $\mathbb{P}^{m}$, for some $m \leq n$.
	It will be constructed by taking a sequence of projections.
\end{rem}
\begin{proof}[Construction]
	Let $X \subsetneq \mathbb{P}^{n}$ a projective variety and $p \in \mathbb{P}^{n}$ s.t. $p \not\in X$.
	Let $H \subset \mathbb{P}^{n}$ be a linear subspace of codim $1$ no passing through $p$,
	i.e. $H$ is the zero locus of a degree $1$ homogeneous polynomial.
	For instance $H = \mathbb{V}_p\left( a_0 x_0 + \ldots + a_n x_n \right)$, with $a_i \neq 0$ for some $i$.
	WLOG $a_n \neq 0$.
	Then $H$ is isomorphic to $\mathbb{P}^{n-1}$, via
	\begin{align}
		\sigma: H = \mathbb{V}_p\left( a_0 x_0 + \ldots + a_n x_n \right) &\xrightarrow{\cong} \mathbb{P}^{n-1} \\
		\left[ x_0 , \ldots , x_n \right] &\mapsto \left[ x_0 , \ldots , x_{n-1} \right]
	,\end{align} 
	where $x_n = -\frac{a_0}{a_n} x_0 - \ldots - \frac{a_{n-1}}{a_n}$.
	We then define the projection $\pi: X \to \mathbb{P}^{n-1}$ as the map sending $q \in X$
	to the point of $H \cong \mathbb{P}^{n-1}$ corresponding to the intersection of $H$ with the line $pq$.
	Up to a linear change of coordinates we may assume $p = \left[ 0 , \ldots , 0, 1 \right]$,
	and $H = \mathbb{V}_p\left( x_n \right)$.
	Then
	\begin{align}
		\pi: X &\to \mathbb{P}^{n-1} \\
		\left[ x_0 , \ldots , x_n \right] &\mapsto \left[ x_0 , \ldots , x_{n-1} \right]
	.\end{align} 
	Let's now construct this change of coordinates.
	Recall that a linear change of coordinates $A \in \mathrm{PGL}(n+1)$ is determined by the image
	of $n+2$ points in general position (which means that no $n+1$ of them are contained in a hyperplane).
	In particular we choose $n$ points, $q_0, \ldots, q_{n-1}$ in general position s.t. $\mathrm{span}(q_0, \ldots, q_n) = H$,
	$p \not\in H$ and $r$ any point on $pq_{n-1}$ different both from $p$ and $q_{n-1}$, then $A$ is defined as follows
	\begin{equation}
	\begin{matrix}
		q_0& \mapsto &\left[ 1, 0 , \ldots , 0 \right]\\
		   &\vdots&\\
		q_{n-1}& \mapsto &\left[ 0 , \ldots , 1, 0 \right]\\
		p& \mapsto &\left[ 0 , \ldots , 0, 1 \right]\\
		r& \mapsto &\left[ 1, 1 , \ldots , 1, 1 \right]
	\end{matrix} 
	.\end{equation} 
\end{proof}

\begin{rem}[]\leavevmode\vspace{-.2\baselineskip}
	\begin{itemize}
		\item The map $\pi: X \to \mathbb{P}^{n-1}$ is a morphism, since $x_0 = \ldots = x_{n-1} = 0$
			has no common solution on $X$ ($p \not\in X$).
		\item All fibers of $\pi$ are finite sets:
			for all $a \in H$, $\pi^{-1}\left( \left\{ a \right\} \right) = pa \cap X$ is a closed subset of
			$pa$ (a line), different from the whole $pa$, since $p \not\in X$.
			The closed subsets of a line are finite sets by the classification of closed sybsets of $\mathbb{P}^{1}$.
		\item We can iterate
			\begin{equation}
			X \xrightarrow{\pi_{p_1}} \mathbb{P}^{n-1} \xrightarrow{\pi_{p_2}} 
			\ldots \xrightarrow{\pi_{p_{n-m}}} \mathbb{P}^{m}
			,\end{equation} 
			so long as $\exists\, p_i \not\in \mathbb{P}^{n-i}$, until
			$f = \pi_{p_{n-m}} \circ \ldots \circ\pi_{p_2} \circ \pi_{p_1}$ is surjective.
			We can interpret geometrically $f$ as a projection of $X$ onto a linear subspace
			$H' \cong \mathbb{P}^{m}$, with center a linear subspace of dimension $n-m-1$,
			spanned by $p_1, \ldots, p_{n-m}$
	\end{itemize}
\end{rem}

\begin{lem}\label{lem:Lem3_Dim}
	Let $X \subsetneq \mathbb{P}^{n}$ be a projective variety not containing $p = \left[ 0 , \ldots , 0, 1 \right]$.
	Then, for every $f \in S(X) = \mathbb{K}\left[x_0, \ldots, x_n \right]/\mathbb{I}_p(X)$
	there exists a degree $D \geq 1$, and homogeneous polynomials
	$a_0, \ldots, a_{D-1} \in \mathbb{K}\left[x_0, \ldots, x_{n-1} \right]$ s.t.
	\begin{equation}\label{eqn:Lem3_Dim}
		f^D + a_{D-1} f^{D-1} + \ldots + a_1 f + a_0 = 0 \in S(X)
	.\end{equation} 
\end{lem} 
\begin{rem}[]
	If we consider $f = \bar{x}_n$, we obtain again that the fibers of $\pi$ are finite.
	However it is a stronger condition: let's restrict to an affine open subset $p \not \in U \cong \mathbb{A}^{n}$ of $\mathbb{P}^{n}$.
	(Let, for simplicity, $U = U_0$), then we have the projection
	\begin{align}
		\left.\pi\right|_{Y}: X \cap U_0 \subset \mathbb{A}^{n} &\to \mathbb{A}^{n-1} \\
		\left( t_1, \ldots, t_n \right) &\mapsto \left( t_1, \ldots, t_{n-1} \right)
	.\end{align} 
	Moreover the formula \eqref{eqn:Lem3_Dim} in lemma \ref{lem:Lem3_Dim}
	says that $\left( \left.\pi\right|_{Y} \right)^*$ realizes $\mathbb{K}[Y]$
	as an integral ring extension of $\mathbb{K}\left[t_1, \ldots, t_{n-1} \right] = \mathbb{K}[\mathbb{A}^{n-1}]$.
	Morphisms satisfying this property (on each affine open set) are called \textbf{finite morphisms}.
\end{rem}

\begin{lem}
	Let $\pi: X \to \mathbb{P}^{n-1}$ be the projection of $X \subset \mathbb{P}^{n}$ with center $p \notin X$.
	If $Y \subset X$ is a closed subset and $\pi(X) = \pi(Y)$, then $Y = X$.
\end{lem} 
\begin{rem}[]
	Equivalently, for a closed $Y \subsetneq X$, then $\pi(Y) \subsetneq \pi(X)$.
\end{rem}

\begin{cor}
	Let $\pi: X \to \mathbb{P}^{n-1}$ be the projections from $p \notin X$.
	Then $\dim X = \dim \pi(X)$.
\end{cor} 
\begin{cor}
	$\dim \mathbb{P}^{n} = n$.
\end{cor} 

\subsection{Dimension of the intersection with a hypersurface}
\subsubsection{Projective case}
\begin{prop}
	Let $X \subset \mathbb{P}^{n}$ be a projective variety, and $f \in \mathbb{K}\left[x_0, \ldots, x_n \right]$
	be a non constant homogeneous polynomial, s.t. $f \notin \mathbb{I}_p(X)$.
	Then
	 \begin{equation}
		 \dim \left( X \cap \mathbb{V}_p\left( f \right) \right) = \dim X - 1
	.\end{equation} 
\end{prop} 
\begin{rem}[]
	If the intersection $X \cap \mathbb{V}_p\left( f \right)$ is reducible,
	we don't know if all the components have maximal dimension ($\dim X - 1$).
	However we will later prove that all components have the same dimension,
	i.e. $X \cap \mathbb{V}_p\left( f \right)$ is of pure dimension $\dim X - 1$.
\end{rem}

\begin{prop}
	Let $X$ be a variety and $\emptyset \neq U \subset X$ be an open subset.
	Then $\dim X = \dim U$.
\end{prop}  

\begin{rem}[Some consequences]\leavevmode\vspace{-.2\baselineskip}
	\begin{enumerate}
		\item $\mathbb{A}^{n} \subset \mathbb{P}^{n}$ is open, hence $\dim \mathbb{A}^{n} = \dim \mathbb{P}^{n}$.
		\item $\mathbb{A}^{n+m} = \mathbb{A}^{n} \cross \mathbb{A}^{m} \stackrel{\text{open}}{\subset} 
			\mathbb{P}^{n} \cross \mathbb{P}^{m}$, hence
			$\dim \left( \mathbb{P}^{n} \cross \mathbb{P}^{m} \right) = n + m$.
		\item Given $f \in \mathbb{K}\left[x_1, \ldots, x_n \right]$ a non-constant polynomial,
			then $\mathbb{V}\left( f \right) \subset \mathbb{A}^{n}$ has dimension $n-1$.
			This follows from $X \subset \overline{X} = \mathbb{V}_p\left( {}^hf \right) \subset \mathbb{P}^{n}$.
		\item Moreover every irreducible component of $\mathbb{V}\left( f \right)$,
			with $f$ the above polynomial, has dimension $n-1$.
			This follows from the factorization in prime components
			(possible, since $\mathbb{K}\left[x_1, \ldots, x_n \right]$ is a UFD)
			\begin{equation}
			f = f_1 \cdot \ldots \cdot f_r
			,\end{equation} 
			with $f_1, \ldots, f_r$ prime.
			Then the irreducible components of $\mathbb{V}\left( f \right)$ are exactly $\mathbb{V}\left( f_i \right)$,
			and, by the above remark, they are exactly of dimension $n-1$.
		\item If $f \in \mathbb{K}\left[x_0, \ldots, x_n \right]$ is a non-constant polynomial 
			and $X := \mathbb{V}_p\left( f \right) \subset \mathbb{P}^{n}$,
			then each irreducible component of $X$ has dimension $n-1$.

			This is true, since we can always find a hyperplane 
			$H \not\subset X ( \subsetneq \mathbb{P}^{n})$.
			Then $\mathbb{A}^{n}\supset X \cap \left( \mathbb{P}^{n} \setminus H \right) \simeq
			\mathbb{V}\left( g \right)$, for some $g \in \mathbb{K}\left[x_1, \ldots, x_n \right]$ non-constant.
			Then the irreducible components of $X$ are the projective closures of the irreducible components
			of $\mathbb{V}\left( g \right)$, which have dimension $n-1$ by the above remark.

			In particular, given a factorization
			\begin{equation}
			f = f_1 \cdot \ldots \cdot f_r
			\end{equation} 
			for $f \in \mathbb{K}\left[x_0, \ldots, x_n \right]$, in prime components
			(which can be obtained homogenizing the prime components of $g$).
			Then
			\begin{equation}
			X = \mathbb{V}_p\left( f \right) =
			\underbrace{\mathbb{V}_p\left( f_1 \right) \cup \ldots \cup \mathbb{V}_p\left( f_r \right)}_{\text{irreducible components of } X}
			.\end{equation} 
	\end{enumerate}
\end{rem}

\begin{defn}[Hypersurface]
	A \textbf{hypersurface} in $\mathbb{A}^{n}$, resp. $\mathbb{P}^{n}$, is a closed subset of dimension $n-1$.
	The minimal degree of a generator of its ideal is called the \textit{degree} of the hypersurface $X$.

	Hypersurfaces of degree $1$ are called \textit{hyperplanes}.
\end{defn}

\begin{rem}[]
	Up to scaling, we have the following correspondances
	\begin{equation}
	\begin{tikzcd}
		\left\{ 
		\begin{matrix}
		\text{irreducible degree } d\\
		\text{hypersurfaces in } \mathbb{A}^{n}
		\end{matrix} 
		\right\} \arrow[r, "", leftrightarrow] &
		\left\{ 
		\begin{matrix}
		\text{irreducible degree } d\\
		\text{polynomials }
		\end{matrix} 
		\right\}_{/ \K^*}
		\subset \mathbb{P} \left( \mathbb{K}\left[x_1, \ldots, x_n \right]_{\leq d} \right)
	\end{tikzcd}
	.\end{equation} 
	\begin{equation}
	\begin{tikzcd}
		\left\{ 
		\begin{matrix}
		\text{irreducible degree } d\\
		\text{hypersurfaces in } \mathbb{P}^{n}
		\end{matrix} 
		\right\} \arrow[r, "", leftrightarrow] &
		\left\{ 
		\begin{matrix}
		\text{irred. homogoneneous}\\
		\text{polynomials of deg. } d 
		\end{matrix} 
		\right\}_{/ \K^*}
		\subset \mathbb{P} \left( \mathbb{K}\left[x_0, \ldots, x_n \right]_{d} \right)
	\end{tikzcd}
	.\end{equation} 
	One can actually prove that the locus of irreducible polynomials is a dense open subset
	in the projective space of all homogenous polynomials of a fixed degree.
	This is indeed a consequence of Main theorem on projective varieties.
\end{rem}

\begin{defn}[Codimension]
	Let $Y \subset X$ be a closed subset.
	We define the \textbf{codimension} of $Y$ in $X$ as
	\begin{equation}
	\mathrm{codim}_X\, Y := \dim X - \dim Y
	.\end{equation} 
\end{defn}

\subsubsection{Affine case}
\begin{rem}[]
	The projective case does not imply the affine one.
	In fact, let $X := \mathbb{V}\left( x_2 - x_1^2 \right) \subset \mathbb{A}^{2}$ and $f = x_1$.
	Let $\overline{X} = \mathbb{V}_p\left( x_0x_2 - x_1^2 \right) \subset \mathbb{P}^{2}$.
	Then
	\begin{equation}
		\dim \left( \overline{X} \cap \mathbb{V}_p\left( x_1 \right) \right) =
		\dim \overline{X} - 1 = 0
	.\end{equation} 
	More explicitly 
	\begin{equation}
	\overline{X} \cap \mathbb{V}_p\left( x_1 \right) = 
	\big\{ \left[ 1, 0, 0 \right], 
	\underbrace{\left[ 0 , 0, 1 \right]}_{\begin{matrix}{\scriptstyle \text{contained in}}\\
	{\scriptstyle H = \mathbb{V}_p\left( x_0 \right)}\end{matrix}} \big\}
	.\end{equation} 
	Moreover factorizing $f$ does not help to construct all irreducible components.
\end{rem}

\begin{prop}
	Let $X \subset \mathbb{A}^{n}$ be an affine variety, and
	$f \in \mathbb{K}\left[x_1, \ldots, x_n \right]$ be a polynomial not vanishing identically on $X$.
	Then, if $X \cap \mathbb{V}\left( f \right) \neq \emptyset$, we have
	$\dim \left( X \cap \mathbb{V}\left( f \right) \right) = \dim X - 1$.
\end{prop} 

\begin{cor}
	If $X \subset \mathbb{A}^{n}$ is an affine variety, and
	$f \notin \mathbb{I}(X)$ is a non constant polynomial, then
	all irreducible components of $X \cap \mathbb{V}\left( f \right)$ have dimension equal to
	$\dim X - 1$.
\end{cor} 

\begin{thm}[Dimension of a fiber (weak version)]
	Let $f: X \to Y$ be a morphism of varieties and assume there is a nonempty
	open subset $U \subset Y$ s.t. $f^{-1}(p) \neq \emptyset$ for all $p \in U$
	and $\dim f^{-1}(p) = n$ for all $p \in U$, then
	\begin{equation}
	\dim X = \dim Y + n
	.\end{equation} 
\end{thm}

\begin{thm}[Dimension of a fiber (complete version)]
	Let $f: X \to Y$ be a surjective morphism of varieties, then
	$\dim X \geq \dim Y$ and
	\begin{enumerate}
		\item for every $y \in Y$ and every component $F$ of $f^{-1}(y)$,
			we have $\dim F \geq \dim X - \dim Y$,
		\item there exists a non-empty open subset $U \subset Y$
			s.t. $\dim f^{-1}(y) = \dim X - \dim Y$ for all $y \in U$.
	\end{enumerate}
\end{thm}

\begin{cor}
	Let $X \subset \mathbb{P}^{n}$ be Zariski closed and
	$f: X \to Y$ be a morphism to a variety $Y$ (in the sense that
	$f$ is the restriction of a morphism defined on some larger subvariety of $\mathbb{P}^{n}$).
	If all fibers $f^{-1}(y)$, for $y \in Y$, are irreducible and of the same dimension, 
	then $X$ is irreducible.
\end{cor} 

\begin{cor}
	For any two varieties $X$ and $Y$, we have
	\begin{equation}
	\dim X \cross Y = \dim X + \dim Y
	.\end{equation} 
	[Consider $\pi_Y: X \cross Y \to Y$, with fibers isomorphic to $X$].
\end{cor} 

\begin{cor}
	Let $X, Y \subset \mathbb{A}^{n}$ be affine varieties s.t. $X \cap Y \neq \emptyset$,
	then every irreducible component $Z$ of $X \cap Y$ has dimension
	\begin{equation}
	\dim Z \geq \dim X + \dim Y - n
	.\end{equation} 
\end{cor} 

\begin{cor}
	Let $X, Y \subset \mathbb{P}^{n}$ be projective varieties.
	Then every irreducible component of $X \cap Y$ has dimension 
	$\geq \dim X + \dim Y - n$ and, if $\dim X + \dim Y \geq n$, then
	$X \cap Y \neq 0$.
\end{cor} 

\subsection{Dimension - again}
\begin{rem}[]
	Let $X$ be a variety.
	Denote by $\mathbb{K}(X)$ the function field of $X$, i.e. the field
	of rational functions on $X$.

	If $X \subset \mathbb{A}^{n}$ is affine, then $\mathbb{K}(X) = Q(\mathbb{K}[X])$,
	e.g.
	\begin{equation}
		\mathbb{K}(\mathbb{A}^{n}) = \mathbb{K}(x_1, \ldots, x_n) \qquad \text{ the field of rational functions}
	.\end{equation} 
	If $X \subset \mathbb{P}^{n}$ is projective, then
	\begin{equation}
		\mathbb{K}(\mathbb{P}^{n}) = \left\{ 
		\frac{p(x_0, \ldots, x_n)}{q(x_0, \ldots, x_n)} \ \middle|\ p, q \in S(X)_d \text{ and } 
	q \neq 0\right\}
	.\end{equation} 
	If $X$ is an arbitrary variety, by ex. 6, ex-sh 3, we have
	\begin{equation}
		\K(X) = \left\{  \left(U, \phi\right) \ \middle|\ 
		\phi \in \mathcal{O}_{X} \left( U \right) \right\}/\sim
	,\end{equation} 
	where $\left(U, \phi\right) \sim \left(U', \phi'\right)$ iff $\left.\phi\right|_{V} = \left.\phi'\right|_{V}$ for some 
	$\emptyset \subsetneq V \subset U \cap U'$ open.
	The fundamental property is that $\K(X) = \K(U)$ for any (affine) open set $U \neq \emptyset$.
\end{rem}

\begin{defn}[Transcendence degree]
	Given $\K \subset E$ a field extension, we define the \textbf{transcendence degree}, 
	denoted by $\mathrm{tr.deg.}(E/\K)$, is the largest integer $l$ s.t.
	there exists $l$ algebraically independent elements $\zeta_1, \ldots, \zeta_l \in E$.
	In the above algebraically independent means that the valuation morphism
	\begin{align}
		\phi: \mathbb{K}\left[x_1, \ldots, x_l \right] &\to E \\
		F(x_1, \ldots, x_l) &\mapsto F(\zeta_1, \ldots, \zeta_l)
	\end{align} 
	has trivial kernel.
\end{defn}

\begin{thm}[]
	If $X$ is a variety, then $\dim X$ equals the 
	transendence degree of $\K(X)$ over $\K$.
\end{thm}

\begin{rem}[]
	If $X \subset \mathbb{A}^{m}$ is an affine variety, then there is a finite map
	$X \to \mathbb{A}^{n}$, with $n = \dim X$, given by a sequence of
	projections.
	Hence $\K(X)$ is an integral extension of $\mathbb{K}\left[x_1, \ldots, x_n \right]$, then
	$\mathrm{tr.deg.}\, \K(X) = n$.
\end{rem}


\section{Blow-ups}

\begin{defn}[Rational map]
	Let $X,Y$ be varieties.
	A \textbf{rational map} $f: X \dashrightarrow Y$ is a morphism $f: U \to Y$, for some
	$\emptyset \neq U \subset X$ open.
	(Notice that we denote rational maps via a dashed arrow, since it might not be defined on the whole variety $X$).
	If $f: U \to Y$ and $f': V \to Y$ are morphisms with
	$\emptyset \neq U \subset X \supset V \neq \emptyset$ both open, then
	$f$ and $g$ define the same rational map iff
	\begin{equation}
	\left.f\right|_{U \cap V} = \left.g\right|_{U \cap V} 
	.\end{equation} 
	(Recall that, since $X$ is irreducible, then $U \cap V \neq \emptyset$).
\end{defn}

\begin{defn}[]\leavevmode\vspace{-.2\baselineskip}
	\begin{itemize}
		\item A rational map $f: X \dashrightarrow Y$, given by a morphism $f: U \to Y$, 
			is dominant iff $f(U)$ contains a non-empty open subset $V \subset Y$
			(in particular its image is dense in $Y$).
		\item If $f: X \dashrightarrow Y$ is dominant, for every $g: Y \dashrightarrow Z$, 
			the composition $g \circ f: X \dashrightarrow Z$ is a rational map.
		\item A birational map is a rational map $f: X \dashrightarrow Y$ with a rational inverse,
			i.e. $f$ is a dominant rational map s.t. there exists a dominant $g: Y \dashrightarrow X$ satisfying
			\begin{equation}
			g \circ f = id_X \qquad \text{ and } \qquad
			f \circ g = id_Y \qquad
			\text{(as rational maps)}
			.\end{equation} 
		\item Two varieties $X$ and $Y$ are said to be birational,
			or birationally equivalent, iff there is a birational map 
			$f: X \dashrightarrow Y$.
	\end{itemize}
\end{defn}

\begin{thm}[]
$X$ and $Y$ are birational iff there exist non-empty open subsets
$U \subset X$ and $V \subset Y$ that are isomorphic to each other.
\end{thm}

Let's now construct the most important class of birational maps: the blow-ups.
Since this is a local construction, we'll start from affine varieties.
\begin{defn}[Construction of blow-up]
	Let $X \subset \mathbb{A}^{n}$ be an affine variety, and $f_0, \ldots, f_r \notin \mathbb{I}(X)$
	be polynomials in $x_1, \ldots, x_n$.
	Then $U := X \setminus \mathbb{V}\left( f_0, \ldots, f_r \right)$ is a non-empty open
	subset of $X$.
	We can consider the morphism
	\begin{align}
		f: U &\to \mathbb{P}^{r} \\
		x &\mapsto \left[ f_0(x) , \ldots , f_r(x) \right]
	.\end{align} 
	The graph $\Gamma := \left\{ \left(p, f(p)\right) \ \middle|\ p \in U \right\} \subset X \cross \mathbb{P}^{r}$
	is isomorphic to $U$ via the projection $\left(p, q\right) \mapsto p$.
	We define the \textbf{blow-up} of $X$ at $\left( f_0, \ldots, f_r \right)$ to be 
	the Zariski closure
	\begin{equation}
	\widetilde{X} := \overline{\Gamma} \subset X \cross \mathbb{P}^{r}
	.\end{equation} 
\end{defn}

\begin{rem}[]\leavevmode\vspace{-.2\baselineskip}
	\begin{itemize}
		\item By definition $\widetilde{X}$ is closed in $X \cross \mathbb{P}^{r}$, hence it
			is a quasi-projective variety.
		\item $\widetilde{X}$ is irreducible, since $U \cong \Gamma$ is irreducible.
		\item $\dim \widetilde{X} = \dim X$.
		\item The projections $p_X: X \cross \mathbb{P}^{r} \to X$ and
			$p_{\mathbb{P}^{r}}: X \cross \mathbb{P}^{r} \to \mathbb{P}^{r}$
			induce the projection morphisms
			\begin{equation}
			\pi: \widetilde{X} \to X \qquad \text{ and } \qquad
			p: \widetilde{X} \to \mathbb{P}^{r}
			.\end{equation} 
	\end{itemize}
\end{rem}

\begin{ex}
	Let $r = 0$, then the blow-up of $X$ at $f_0 \notin \mathbb{I}(X)$ is isomorphic to $X$.
	In fact $\widetilde{X} \subset X \cross \mathbb{P}^{0} \cong X$, since $\mathbb{P}^{0}$ is just a point.
	Then an irreducible subvariety of $X \cross \mathbb{P}^{0}$, with the same dimension as $X$,
	is isomorphic to $X$.	
\end{ex} 

\begin{ex}
	Let $X = \mathbb{A}^{2}$, with coordinates $x_0, x_1$.
	Set $f_0 := x_0$ and $f_1 := x_1$.
	Then $\widetilde{X} \subset \mathbb{A}^{2} \cross \mathbb{P}^{1}$.
	Then
	\begin{align}
		f: U = \mathbb{A}^{2}\setminus \left\{ (0, 0) \right\} &\to \mathbb{P}^{1} \\
		(x_0, x_1) &\mapsto \left[ x_0 , x_1 \right]
	.\end{align} 
	The graph of $f$ is
	\begin{equation}
		\Gamma = \left\{ \left((x_0, x_1), [y_0,y_1] \right) \ \middle|\ 
		x_0y_1 = x_1y_0 \right\} \subset U \cross \mathbb{P}^{1}
	.\end{equation} 
	Then, if we compute its closure, we obtain
	\begin{equation}
		\widetilde{X} = \left\{ \left((x_0, x_1), [y_0,y_1] \right) \ \middle|\ 
		x_0y_1 = x_1y_0 \right\} \subset \mathbb{A}^{2} \cross \mathbb{P}^{1}
	,\end{equation} 
	in which we allow $(x_0, x_1) = (0,0)$.
	Denoting $\pi: \widetilde{X} \to X$ the projection, we obtain
	\begin{align}
		\pi^{-1}\left( (x_0, x_1) \right) &= \left\{ \left((x_0, x_1), [x_0,x_1] \right) \right\}
		\qquad \text{ if } (x_0, x_1) \neq (0,0)\\
		\pi^{-1}\left( (x_0, x_1) \right) &= \left( (0,0) \right) \cross \mathbb{P}^{1} \cong \mathbb{P}^{1}
	.\end{align} 
\end{ex} 

\begin{defn}[Strict transform]
	Let $Y \subset X$ be a closed subset with $Y \cap U \neq \emptyset$,
	then the Zariski closure of $Y \cap U$ in $\widetilde{X}$ is called the
	\textbf{strict transform} $\widetilde{Y}$ of $Y$.
	This is the same as the blow-up of $Y$ at $(f_0, \ldots, f_r)$.
\end{defn}

\begin{ex}
	In the above example the strict transform is interesting for a curve $C$ in $\mathbb{A}^{2}$
	passing through $(0,0)$.
	Since $C$ is a curve, then $C = \mathbb{V}\left( g \right)$, for
	\begin{equation}
		g(x_0,x_1) = \sum_{i,j \geq 0}^{} a_{ij} x_0^i x_1^j = 
		a_{00} + a_{10} x_0 + a_{01} x_1 + h.o.t
	.\end{equation} 
	Let's assume that $a_{00} = 0$, i.e. $(0.0) \in C$, and $(a_{01}, a_{10}) \neq (0,0)$, so that
	\begin{equation}
	L := \mathbb{V}\left( a_{10} x_0 + a_{01} x_1 \right)
	\end{equation} 
	is a line, which we call the tangent line to $C$ at $(0,0)$.
	If $\left((x_0,x_1), [y_0,y_1]\right) \in \widetilde{D}$, then it satisfies
	\begin{equation}\label{eqn:strtrexcond}
	\begin{cases}
		a_{10}x_0 + a_{01}x_1 + a_{20}x_0^2 + \ldots = 0\\
		x_0y_1 - x_1y_0 = 0
	\end{cases} 
	.\end{equation} 
	But the solution set of \eqref{eqn:strtrexcond} is larger than $\widetilde{C}$, since
	it contains the whole $\pi^{-1} \left( (0,0) \right) \cong \mathbb{P}^{1}$.
	But $\widetilde{C}$ is irreducible, hence it cannot contain such additional component.

	To get the equations for $\widetilde{C}$, we work first on $\mathbb{A}^{2} \setminus \left\{ x_0x_1 = 0 \right\}$.
	There we can multiply the equation of $C$ by $\frac{y_0}{x_0}$, which is
	a regular function on $\mathbb{A}^{2} \setminus \left\{ x_0x_1 = 0 \right\} \cross \mathbb{P}^{1}$,
	and obtain
	\begin{equation}
	a_{10}y_0 + a_{01} \frac{y_0}{x_0} x_1 + a_{20} y_0 x_0 +
	a_{11} y_0 x_1 + \ldots
	.\end{equation} 
	Using $\frac{y_0}{x_0} = \frac{y_1}{x_1}$ this is equivalent to
	\begin{equation}
	\begin{cases}
		a_{10}y_0 + a_{01} y_1 + a_{20}y_0x_0 +
		a_{11}y_0x_1 + a_{02}y_1x_1 + \ldots = 0\\
		x_0y_1 - x_1y_0 = 0
	\end{cases} 
	.\end{equation} 
	This equation should hold on $\widetilde{C}$.
	For every $C \ni (x_0, x_1) \neq (0,0)$, the above equations give the
	single point $\left((x_0,x_1), [x_0,x_1]\right)$.
	Instead over $(0,0)$ we obtain the point $\left((0,0), [y_0,y_1]\right)$
	with $a_{10}y_0 + a_{01}y_1 = 0$, which is exactly the equation of the tangent line at $(0,0)$.
\end{ex} 

\begin{ex}[Blow-up of $\mathbb{A}^{2}$ at $(x_0,x_1)$]
	It is a subset
	\begin{equation}
	\widetilde{\mathbb{A}^{2}} \subset \mathbb{A}^{2} \cross \mathbb{P}^{1}
	,\end{equation} 
	whose points are 
	\begin{equation}
	\left( (x_0, x_1), [y_0,y_1] \right)\quad \text{ s.t. }\quad \mathrm{rk}\, 
	\begin{pmatrix}
		x_0 & x_1\\y_0 & y_1
	\end{pmatrix} = 1
	.\end{equation} 
	Moreover it comes endowed with two morphisms (since they are given by the
	restriction of the projection morphisms)
	\begin{itemize}
		\item $p: \widetilde{\mathbb{A}^{2}} \to \mathbb{P}^{1}$ that extends the map
			used to define the projective space $\mathbb{P}^{1}$
			\begin{align}
				\K^2\setminus\left\{ \textbf{0} \right\} &\to \mathbb{P}^{1} \\
				(y_0,y_1) &\mapsto [y_0,y_1]
			.\end{align} 
			Then the fibers of $p$ are isomorphic to the lines of $\mathbb{A}^{2}$
			through the origin via $\pi$.
		\item $\pi: \widetilde{\mathbb{A}^{2}} \to \mathbb{A}^{2}$, which is an isomorphism
			outside $(0,0) = \mathbf{0}$.
			In fact $\pi^{-1}(\left\{ \mathbf{0} \right\}) = \left\{ \mathbf{0} \right\} \cross \mathbb{P}^{1}$.
	\end{itemize}
	In particular we can say that the points of $\pi^{-1}\left( \left\{ \mathbf{0} \right\} \right)$ correspond
	to the directions of the lines through the origin $\mathbf{0}$.

	Looking at the strict transform, then we can identify $\pi^{-1}(\left\{ \mathbf{0} \right\})$
	with the set of tangent directions to $\mathbb{A}^{2}$ at the origin $\mathbf{0}$.
\end{ex} 

\begin{lem}
	The blow-up of an affine variety $X$ at $(f_0, \ldots, f_r)$
	only depends on the ideal generated by $(f_0, \ldots, f_r)$ in $\mathbb{K}[X]$.
\end{lem} 

\begin{defn}[Blow-up at an ideal]
	Given an affine variety $X$ and $\mathcal{I} \subset \mathbb{K}[X]$ an ideal,
	we define the blow-up of $X$ at $\mathcal{I}$ as the
	blow-up of $X$ at $(f_0, \ldots, f_r)$, for any $f_0, \ldots, f_r$ generators 
	of $\mathcal{I}$.

	In particular, for $Y \subset X$ a closed subset, we can define the blow-up of $X$
	along $Y$ as the blow-up of $X$ at $\mathbb{I}(Y)$.
\end{defn}

\begin{rem}[]
	The construction of the blow-up is local.
	In fact let $X$ be an arbitrary variety and $Y \subset X$ a closed subset.
	If $\left\{ U_i \right\}_{i \in I}$ is an affine open cover of $X$, 
	let us denote by $\widetilde{U}_i$ the blow-up of the affine variety $U_i$
	along $Y \cap U_i$.

	It is easy to check that we can glue together the $\widetilde{U}_i$ in 
	a natural way, to construct a variety $\widetilde{X}$, which we will call the
	blow-up of $X$ along $Y$.

	Actually we can blow-up a sheaf of ideals on $X$.
	In fact the construction of the blow-up of an ideal can be generalized to arbitrary varieties.
	In order to do so we need
	\begin{itemize}
		\item $I(U) \subset \mathcal{O}_{X} \left( U \right)$, for each open subset $U \subset X$,
		\item gluing conditions.
	\end{itemize}
\end{rem}
Let's specialize the above remark to the case of projective varieties.
\begin{defn}[Blow-up for projective varieties]
	Let $X$ be a projective variety, and let
	$f_0, \ldots, f_r$ homogeneous polynomials of degree $d$, not in $\mathbb{I}_p(X)$.
	Let $U := X \setminus \mathbb{V}_p\left( f_0, \ldots, f_r \right)$ and
	\begin{equation}
		\Gamma := \left\{ \left(x, [f_0(x), \ldots, f_r(x)]\right) \ \middle|\ 
		x \in U\right\} \subset X \cross \mathbb{P}^{r}
	.\end{equation} 
	Then we define $\widetilde{X} := \Gamma \cross \mathbb{P}^{r}$ to be the \textbf{blow-up}
	of $X$ at $(f_0, \ldots, f_r)$.
\end{defn}

\begin{rem}[]
	In particular, the blow-up of a projective variety is a projective variety.
\end{rem}

\begin{lem}\leavevmode\vspace{-.2\baselineskip}
	\begin{enumerate}
	\item $\widetilde{X} \subset \left\{ \left(x, [y_0, \ldots, y_r]\right) \ \middle|\ 
		y_i f_j(x) - y_j f_i(x) = 0, \text{ for all } i,j = 0, \ldots, r\right\}
		\subset X \cross \mathbb{P}^{r}$.
	\item The inverse image $E := \pi^{-1} \left( \mathbb{V}_p\left( f_0, \ldots, f_r \right) \right)$
		is of pure dimension $\dim X - 1$.
		$E$ is called the {\em exceptional hypersurface} of the blow-up.
	\end{enumerate}
\end{lem} 

\begin{rem}[]
	The equations $y_i f_j(x) = y_j f_i(x)$, in general, are not the only
	equations defining $\widetilde{X}$ inside $X \cross \mathbb{P}^{r}$.

	In case they actually are the only ones, then
	\begin{equation}
		E = \left( X \cap \mathbb{V}\left( f_0, \ldots, f_r \right) \right) \cross \mathbb{P}^{r}
	.\end{equation} 
	The dimensions are: $\dim E = \dim X -1$, whereas $X \cap \mathbb{V}\left( f_0, \ldots, f_r \right)$ might have
	dimension $> \dim X - r -1$, then the r.h.s. might have dimension $> \dim X -1$.
\end{rem}

\begin{defn}[Initial polynomial/ideal]
	Consider $0 \neq f \in \mathbb{K}\left[x_1, \ldots, x_n \right]$.
	\begin{itemize}
		\item We can decompose in homogeneous components $f = \sum_{i \in \N}^{} f_i$,
			for $f_i \in \mathbb{K}\left[x_1, \ldots, x_n \right]_i$
			homogeneous of degree $i$.
			The minimal $i \in \N$ s.t. $f_i \neq 0$ is called the \textbf{minimal degree},
			denoted with $\mathrm{mindeg}\, f$, of $f$.
			For $i = \mathrm{mindeg}\, f$, we call $f^{in}:= f_i$ the \textbf{initial polynomial}
			of $f$.
		\item For an ideal $(0) \neq \mathcal{I} \subset \mathbb{K}\left[x_1, \ldots, x_n \right]$,
			we define the \textbf{initial ideal} of $\mathcal{I}$ to be
			\begin{equation}
			\mathcal{I}^{in} :=
			\left( f^{in} \ \middle|\ f \in \mathcal{I} \right)
			,\end{equation} 
			the homogeneous ideal generated by the initial polynomials of the elements of $\mathcal{I}$.
	\end{itemize}
\end{defn}

\begin{rem}[]
	Notice that, given $\mathcal{I} = \left( g_1, \ldots, g_l \right)$, then it is not always
	true that $\mathcal{I}^{in} = \left( g_1^{in}, \ldots, g_l^{in} \right)$.
\end{rem}

\begin{rem}[]
	Let's now study the blow-up of an affine variety $X \subset \mathbb{A}^{n}$ at a point,
	wlog $\left( 0, \ldots, 0 \right)$.
	Let us denote by $\widetilde{\mathbb{A}^{n}}$ the blow up of $\mathbb{A}^{n}$
	at $\left( x_1, \ldots, x_n \right)$.
	We want to cover it with affine varieties:
	let's start covering $\mathbb{P}^{n-1}$ with the affine varieties
	$U_i := \left\{ \left[ y_1 , \ldots , y_n \right] \in \mathbb{P}^{n-1} \ \middle|\ y_i \neq 0 \right\}$.
	In particular, for $i = 1$, we have $y_1 \neq 0$.
	Then, by the condition on the rank we obtain $x_j = x_1 y_j$ for the elements in
	$\widetilde{\mathbb{A}^{n}}$, hence we can define a map $\varphi$, with which we can cover the blow-up
	\begin{align}
		\varphi: \mathbb{A}^{n} &\to \widetilde{\mathbb{A}^{n}} \subset \mathbb{A}^{n} \cross \mathbb{P}^{n-1}
		\supset \mathbb{A}^{n} \cross U_1\\
		\big( x_1, \underbrace{x_1y_2, \ldots, x_1y_n}_{\text{inhom. coord. on }U_1} \big) &\mapsto 
		\left(\left( x_1, x_1y_2, \ldots, x_1y_n \right), \left[ 1, y_2 , \ldots , y_n \right]\right)
	,\end{align} 
	Then, by choosing the various affine charts of $\mathbb{P}^{n-1}$, we can cover
	the blow-up $\widetilde{\mathbb{A}^{n}}$ by affine spaces.

	Then the blow-up of any $X := \mathbb{V}\left( \mathcal{I} \right) \subset \mathbb{A}^{n}$ containing $\mathbf{0}$
	is defined, on this coordinate patch, by the vanishing of
	\begin{equation}
		\frac{f(x_1, x_1y_2, \ldots, x_1y_n)}{x_1^{\mathrm{mindeg}\, f}} \qquad
		\text{ for all } f \in \mathcal{I}, f \neq 0
	.\end{equation} 
	The exceptional hypersurface $E \subset \widetilde{X}$ is
	\begin{equation}
	\left\{ \mathbf{0} \right\} \cross \mathbb{V}_p\left( \mathcal{I}^{in} \right)
	\subset \left\{ \mathbf{0} \right\} \cross \mathbb{P}^{n-1}
	.\end{equation} 
	The geometric interpretation of $E$ is similar to taking the lower terms of
	the Taylor expansion of a function.
	In fact $\mathbb{I}(X)^{in}$ approximates $\mathbb{I}(X)$ by keeping only the terms of smallest degree.
	Since $\mathbb{I}^{in}$ is a homogeneous ideal, then
	$\hat{Y} := \mathbb{V}\left( \mathbb{I}(X)^{in} \right) \subset \mathbb{A}^{n}$ is
	a cone with vertex in the origin.
	By construction, $\hat{Y}$ is the cone (of the same dimension as $X$) that approximates $X$ best around $\mathbf{0}$.
	In fact
	\begin{equation}
		\dim \hat{Y} = \dim \underbrace{\mathbb{V}_p\left( \mathbb{I}(X)^{in} \right)}_{E} + 1 =
		\dim X - 1 + 1 = \dim X
	.\end{equation} 
\end{rem}

\begin{defn}[Tangent cone]
	Let $p \in X \subset \mathbb{A}^{n}$ a point of the Zariski closed subset $X$
	(wlog $p = \left( 0, \ldots, 0 \right) \in \mathbb{A}^{n}$).
	Then the tangent cone to $X$ at the point $p$ is
	\begin{equation}
		C_{X,p} := \mathbb{V}\left( \mathbb{I}(X)^{in} \right) \subset \mathbb{A}^{n}
	.\end{equation} 
\end{defn}
\begin{rem}[]
	Then the exceptional hypersurface $E \subset \widetilde{X}$ is the projectivization of this cone
	and corresponds to the limit tangent directions to $X$ at $p$.
\end{rem}

\begin{ex}\leavevmode\vspace{-.2\baselineskip}
\begin{enumerate}
	\item Let $X:= \mathbb{V}\left( y - x^2 + x \right) \subset \mathbb{A}^{2}$.
		Let $f(x,y) = y-x^x + x$, then $\mathrm{mindeg}\, f = 1$.
		Then $C_{X,0} = \mathbb{V}\left( y + x \right) \subset \mathbb{A}^{2}$ is the tangent line to $X$
		at the origin $\mathbf{0}$.
	\item Let $X:= \mathbb{V}\left( x^2 + x^3 -y^2 \right) \subset \mathbb{A}^{2}$.
		Let $f(x,y) = x^2 + x^3 - y^2$, then $\mathrm{mindeg}\, f = 2$.
		Then $C_{X,0} = \mathbb{V}\left( x^2 - y^2 \right) = \mathbb{V}\left( x - y \right) \cup
		\mathbb{V}\left( x + y \right) \subset \mathbb{A}^{2}$ is the tangent line to $X$
		at the origin $\mathbf{0}$.
		If we blow up $X$ at $\mathbf{0}$, then $\widetilde{X}$ separates the two branches 
		of $X$ at the origin $\mathbf{0}$.
	\item Let $X:= \mathbb{V}\left( x^3 -y^2 \right) \subset \mathbb{A}^{2}$.
		Let $f(x,y) = x^3 - y^2$, then $\mathrm{mindeg}\, f = 2$.
		Then $C_{X,0} = \mathbb{V}\left( y^2 \right) = \mathbb{V}\left( y \right)
		\subset \mathbb{A}^{2}$ is the tangent line to $X$
		at the origin $\mathbf{0}$.
		But this line comes with some nontrivial multiplicity.
		In fact, if we blow-up the origin $\widetilde{X}$ is tangent
		to $\left\{ \mathbf{0} \right\} \cross \mathbb{P}^{1} = E \subset \widetilde{\mathbb{A}^{2}}$
		at its point corresponding to the line $y = 0$.
\end{enumerate}
\end{ex} 

\begin{rem}
	Given a rational map $f: X \dashrightarrow \mathbb{P}^{r}$, given by
	$x \mapsto \left[ f_0(x) , \ldots , f_f(x) \right]$ for $r+1$ polynomials of the same degree
	(not all of them in the homogeneous ideal of $X$), then
	$f$ is a well defined morphism only on an open subset
	$U := X \setminus \mathbb{V}\left( f_0, \ldots, f_r \right)$.
	Then we have a closed subset on which $f$ is not determined: $\mathbb{V}\left( f_0, \ldots, f_r \right)$.
	We will call it the \textit{indeterminacy locus} of $f$.

	However, we can always extend $f$ to a morphism $\tilde{f}: \widetilde{X} \to \mathbb{P}^{r}$, 
	where $\widetilde{X}$ is a the blow-up of $X$ at $\left( f_0, \ldots, f_r \right)$, and
	$f$ is defined by the composition $\widetilde{X} \hookrightarrow X \cross \mathbb{P}^{r} \xrightarrow{\pi} \mathbb{P}^{r}$.
	If we blow-up the indeterminacy locus, we can always extend $f: X \dashrightarrow \mathbb{P}^{r}$
	to a projective variety birational to $X$.
	
	If, moreover, we start from a birational map $f: X \dashrightarrow Y$ this gives rise to interesting situations.
	Let's look at how to extend such a map to isomorphisms $\widetilde{X} \to \widetilde{Y}$.
\end{rem} 

\begin{ex}[Cremona transformation]
	It is a birational map
	\begin{align}
		f: \mathbb{P}^{2} &\dashrightarrow \mathbb{P}^{2} \\
		\left[ x_0, x_1 , x_2 \right] &\mapsto [ x_1x_2, x_0x_2, x_0x_1]
	.\end{align} 
	Its indeterminacy locus is $\mathbb{V}_p\left( x_1x_2, x_0x_2, x_0x_1 \right) = \left\{ 
	p_0 := [1,0,0], p_1 := [0,1,0], p_3 := [0,0,1] \right\}$.
	Since there are no relations in $S(\mathbb{P}^{2}) = \mathbb{K}\left[x_0, x_1, x_2 \right]$,
	we cannot extend $f$ to a morphism outside $U:= \mathbb{P}^{2} \setminus \left\{ p_0, p_1, p_2 \right\}$.

	Let's show that $f$ is a birational map: in such case there are $U \subset \mathbb{P}^{2} \supset V$
	open subsets s.t. $\left.f\right|_{U}: U \to V$ is an isomorphism.
	In fact, on $V := \mathbb{V}_p\left( x_0x_1x_2 \right)$, we can divide by $x_0x_1x_2$, and obtain that
	\begin{equation}
		f([x_0, x_1, x_2]) = [x_1x_2, x_0x_2, x_0x_1] = 
		[ \frac{1}{x_0}, \frac{1}{x_1}, \frac{1}{x_2}]
	.\end{equation} 
	Clearly, when restricted to $V$, $f \circ f = id_V$, hence $\left.f\right|_{U}: V \to V$
	is an isomorphism (f is an involution of $V$).
	Moreover, denoting by $L_i := \mathbb{V}_p\left( x_i \right)$, we can notice that
	$f(L_i) = p_i$, where $p_i$ are the points defined above.
	In fact $f(L_0) = f([0,x_1,x_2]) = [x_1x_2, 0, 0] = p_0$.

	Finally we can extend $f$ to a morphism by using blow-ups.
	Let's take $\widetilde{\mathbb{P}^{2}}$ the blow-up of $\mathbb{P}^{2}$ at
	$(x_1x_2, x_0x_2, x_0x_1)$, i.e. the blow-up at $p_0, p_1, p_2$.
	Then we can extend $f$ to $\tilde{f}: \widetilde{\mathbb{P}^{2}} \to \widetilde{\mathbb{P}^{2}}$.
	In fact $\widetilde{\mathbb{P}^{2}} \subset \mathbb{P}^{2} \cross \mathbb{P}^{2}$, and
	\begin{equation}
		\left( [x], [u] \right) \in \widetilde{\mathbb{P}^{2}} \iff
		\mathrm{rk}
		\begin{pmatrix}
			x_1x_2 & x_0x_2 & x_0x_1\\
			u_0 & u_1 & u_2
		\end{pmatrix} = 1
	.\end{equation} 
	Notice that this is equivalent (in general), to asking that $u_0, u_1, u_2$ is proportional to $1/x_0, 1/x_1, 1/x_2$,
	hence the condition is symmetric in $[x]$ and $[u]$.
	The we can define
	\begin{equation}
		\tilde{f} \left( [x_0, x_1, x_2], [u_0, u_1, u_2] \right) =
		\left( [u_0, u_1, u_2], [x_0, x_1, x_2] \right)
	,\end{equation} 
	and $[u_0, u_1, u_2] = f([x_0, x_1, x_2])$, where defined.
	Then this is clearly an extension of $f$ and also clearly an isomorphism.
	We only need to check that its image is $\widetilde{\mathbb{P}^{2}}$.
	In order to do so we only need to check that the image of the three exceptional divisors
	is contained in $\widetilde{\mathbb{P}^{2}}$.
	This is true, since $f$ is a birational inverse to itself.

	More explicitly, in $\widetilde{\mathbb{P}^{2}}$ the points $p_i$ get mapped to the exceptional lines $E_i$, and $L_i$
	get mapped to their strict transform $\widetilde{L_i}$.
	Then $\tilde{f}$ sends $\widetilde{L_i}$ to $E_i$ and vice-versa.
\end{ex} 

\begin{ex}
	Another example of a couple of birational projective varieties is given by
	\begin{equation}
	f: \mathbb{P}^{2} \dashrightarrow \mathbb{P}^{1} \cross \mathbb{P}^{1}
	,\end{equation} 
	since they contain the isomorphic open subsets $\mathbb{A}^{2} \cong \mathbb{A}^{1} \cross \mathbb{A}^{1}$.
	In particular we can explicitly write $f$ as, in the affine chart $U_0 \subset \mathbb{P}^{2}$
	\begin{equation}
	[1,x,y] \xrightarrow{f} s_{1,1} \left( [1,x], [1,y] \right) = [1, x, y, xy]
	.\end{equation} 
	In homogeneous coordinates (homogenizing everything with respect to $x_0$) it becomes
	\begin{equation}
		f \left( [x_0, x_1, x_2] \right) = [x_0^2, x_0x_1, x_0x_2, x_1x_2]
	.\end{equation} 
	Its indeterminacy locus is exactly
	\begin{equation}
	\mathbb{V}_p\left( x_0^2, x_0x_1, x_0x_2, x_1x_2 \right) =
	\mathbb{V}_p\left( x_0, x_1x_2 \right) = 
	\left\{ q_1 := [0,1,0], q_2 := [0,0,1] \right\}
	.\end{equation} 
	To extend $f$ to a morphism (to resolve $f$) we need to blow-up $\mathbb{P}^{2}$ at
	$\mathcal{I} = \left( x_0^2, x_0x_1, x_0x_2, x_1x_2 \right) \subsetneq \mathbb{I}_p(\left\{ q_1, q_2 \right\})$
	(notice that they have the same zero locus: they are related by saturations).

	Let $\widetilde{\mathbb{P}^{2}} \subset \mathbb{P}^{2} \cross \mathbb{P}^{3}$ be the blow-up at $\mathcal{I}$.
	Then $\widetilde{\mathbb{P}^{2}}$ is isomorphic to the blow-up
	of $\mathbb{P}^{2}$ at $q_1$ and $q_2$.
	In $U_1 = \left\{ x_1 \neq 0 \right\} \ni q_1$, we can take $x_1 = 1$ and then $\mathcal{I}$
	restricts to 
	\begin{equation}
	\left( x_0^2, x_0, x_0x_2, x_2 \right) = \left( x_0, x_2 \right) =
	\mathbb{I}(q_2) \subset \mathbb{K}\left[x_0, x_2 \right] = \mathbb{K}[U_1]
	.\end{equation} 
	andalogously, on $U_2$, $\mathcal{I}$ restricts to 
	\begin{equation}
		\left( x_0^2, x_0x_1, x_0, x_1 \right) = \left( x_0, x_1 \right) =
		\mathbb{I}(q_1) \subset \mathbb{K}\left[x_0, x_1 \right] = \mathbb{K}[U_2]
	.\end{equation} 
	The birational inverse of $f$ is
	\begin{align}
		g: \mathbb{P}^{1} \cross \mathbb{P}^{1} &\to \mathbb{P}^{2} \\
		[u_0, u_1, u_2, u_3] &\mapsto [u_0, u_1, u_2]
	.\end{align} 
	This is the projection from $p = [0,0,0,1] = s_{1,1} \left( [0,1],[0,1] \right)$
	on $\mathbb{P}^{1} \cross \mathbb{P}^{1}$.
	Hence $p$ is the indeterminacy locus and we need
	to blow it up, to obtain
	\begin{equation}
	\widetilde{\mathbb{P}^{1} \cross \mathbb{P}^{1}} \subset \mathbb{P}^{1} \cross \mathbb{P}^{1} \cross \mathbb{P}^{2}
	\subset \mathbb{P}^{3} \cross \mathbb{P}^{2}
	.\end{equation} 
	Now we want to define an isomorphism between $\widetilde{\mathbb{P}^{2}}$ and $\widetilde{\mathbb{P}^{1} \cross \mathbb{P}^{1}}$.
	\begin{align}
		\tilde{f}: \widetilde{\mathbb{P}^{2}} &\to \widetilde{\mathbb{P}^{1} \cross \mathbb{P}^{1}} \\
		\left( [x_0, x_1, x_2], [y_0, y_1, y_2, y_3] \right) &\mapsto 
		\left( [y_0,y_1, y_2, y_3], [x_0, x_1, x_2] \right)
	,\end{align} 
	where the points $\left( [x_0, x_1, x_2], [y_0, y_1, y_2, y_3] \right) \in \widetilde{\mathbb{P}^{2}}$
	have to satisfy
	\begin{equation}
	1 = \mathrm{rk}\, 
	\begin{pmatrix}
		y_0 & y_1 & y_2 & y_3\\
		x_0^2 & x_0x_1 & x_0x_2 & x_1x_2
	\end{pmatrix}
	\Rightarrow \mathrm{rk}\, 
	\begin{pmatrix}
		y_0 & y_1 & y_2\\
		x_0^2 & x_0x_1 & x_0x_2
	\end{pmatrix} = 1
	\ \ \text{ for } x_0 \neq 0
	.\end{equation} 
	This is the restriction to $\widetilde{\mathbb{P}^{2}}$ of an isomorphism between
	$\mathbb{P}^{2} \cross \mathbb{P}^{3}$ and $\mathbb{P}^{3} \cross \mathbb{P}^{2}$.
	Hence we only have to check that
	\begin{equation}
		\tilde{f} \left( \widetilde{\mathbb{P}^{2}} \right) \subset \widetilde{\mathbb{P}^{1} \cross \mathbb{P}^{1}}
		\qquad \text{ and } \qquad
		\tilde{f}^{-1} \left( \widetilde{\mathbb{P}^{1} \cross \mathbb{P}^{1}} \right) \subset
		\widetilde{\mathbb{P}^{2}}
	.\end{equation} 
	Since $\widetilde{\mathbb{P}^{2}}$ is an irreducible projective variety, it is sufficient 
	to show the above inclusions for dense open subsets of the required varieties.
	Take $\mathbb{A}^{2} \cong \mathbb{P}^{2} \setminus \mathbb{V}_p\left( x_0 \right) \subset \widetilde{\mathbb{P}^{2}}$,
	where $\mathbb{V}_p\left( x_0 \right)$ is the line passing trhough $q_1$ and $q_2$.
	We know that
	\begin{equation}
		\tilde{f} \left(  \mathbb{A}^{2} \right) = f \left( \mathbb{A}^{2} \right) =
		\mathbb{A}^{1} \cross \mathbb{A}^{1} \subset
		\mathbb{P}^{1} \cross \mathbb{P}^{1} \setminus \mathbb{V}_p\left( x_0y_0 \right)
		\hookrightarrow \widetilde{\mathbb{P}^{1} \cross \mathbb{P}^{1}}
	.\end{equation} 
	The result for $\tilde{f}^{-1}$ follows analogously.
\end{ex} 

\begin{lem}
	$\mathbb{P}^{1} \cross \mathbb{P}^{1}$ blown-up at one point is isomorphic to the blow-up 
	of $\mathbb{P}^{2}$ at two points.
\end{lem} 
\begin{proof}
	The proof is a generalzation of the above example.
\end{proof}

\subsection{Smooth varieties}
\begin{defn}[Tangent space at a point]
	Let $X \subset \mathbb{A}^{n}$ be an affine variety and $p \in X$.
	After a linear change of coordinates
	\begin{equation}
	X_1 := x_1 - p_1, \qquad X_2 := x_2 - p_2, \qquad \ldots, \qquad
	X_n := x_n - p_n
	\end{equation} 
	we can assume that $p = \left( 0, \ldots, 0 \right)$.
	We define the \textbf{tangent space} to $X$ at $p$ to be
	\begin{equation}
	T_{X,p} := \mathbb{V}\left( \left\{ f_1 \in \mathbb{K}\left[x_1, \ldots, x_n \right] \ \middle|\ 
	f \in \mathbb{I}(X)\right\} \right)
	,\end{equation} 
	in which $f_1$ represents the linear part of $f = \sum_{j \in \N}^{} f_j$, with
	$f_j \in \mathbb{K}\left[x_1, \ldots, x_n \right]_j$ are the homogeneous components of $f$.
\end{defn}

\begin{rem}[]
	Consider $p = \left( p_1, \ldots, p_n \right)$.
	We can consider the Taylor expansion of $f \in \mathbb{K}\left[x_1, \ldots, x_n \right]$ at $p$
	(after all it is a polynomial):
	\begin{equation}
		f \left( x_1, \ldots, x_n \right) =
		f(p) + \sum_{i=1}^{n} \frac{\partial f}{\partial x_i} (p) \left( x_i - p_i \right) +
		h.o.t.
	.\end{equation} 
	Then $T_{X,p}$ is defined by the linear equations
	\begin{equation}
		\sum_{i=1}^{n} \frac{\partial f}{\partial x_i} (p) \left( x_i - p_i \right) = 0
	,\end{equation} 
	for all $f \in \mathbb{I}(X)$ (for which, hence, $f(p) = 0$).
\end{rem}

\begin{rem}[]
	The degree $1$ part of an ideal is always contained in the initial ideal,
	i.e. $\mathcal{I}_1 \subset \mathcal{I}^{in}$.
	It follows that $C_{X,p} \subset T_{X, p}$, i.e. the tangent cone is contained in the tangent space.
\end{rem}
 
Let's give a more intrinsic definition to the tangent space:
\begin{lem}
	Let $p \in X \subset \mathbb{A}^{n}$ a point of an affine variety.
	Then the Vector Space $T_{X,p} \subset \K^n$ is isomorphic to the space of
	linear forms on $\mathfrak{m}_{X,p}/\mathfrak{m}^2_{X,p}$, where
	\begin{equation}
		\mathfrak{m}_{X,p} = \left\{ \varphi \in \mathcal{O}_{X,p} \ \middle|\ \varphi(p) = 0 \right\}
	\subset \mathcal{O}_{X,p} \subset \K(X)
	,\end{equation} 
	which can be identified with the germs of functions vanishing at $p$.
	Analogously, for $\mathfrak{m}^2_{X,p}$, we have
	\begin{equation}
	\mathfrak{m}^2_{X,p} = \left\langle \varphi \cdot \psi \ \middle|\ 
\varphi, \psi \in \mathfrak{m}_{X,p} \right\rangle
	\subset \mathcal{O}_{X,p} \subset \K(X)
	,\end{equation} 
	i.e. it contains the germs of functions vanishing at order $\geq 2$ at $p$.
	Then
	\begin{equation}
		T_{X,p} \cong \left( \mathfrak{m}_{X,p}/\mathfrak{m}^2_{X,p} \right)^{\vee}
	.\end{equation} 
\end{lem} 
\begin{proof}
	We can assume, wlog, that $p = \left( 0, \ldots, 0 \right)$.
Then, the coordinates $x_1, \ldots, x_n$ of $\mathbb{A}^{n} \cong \K^n$ define linear forms in $V^{\vee}$.
In particular $x_1, \ldots, x_n$ is a basis of $V^{\vee} = \mathrm{Hom}_{\K}\left( V, \K \right)$.
Let, now, $\mathcal{I} := \mathbb{I}(X)$ and $\mathcal{I}_1 \subset \mathbb{K}\left[x_1, \ldots, x_n \right]_1 = V^{\vee}$
be a linear subspace.
\begin{description}
	\item[Step 1:] \textit{$V/\mathcal{I}_1$ is canonically isomorphic to
		$T^{\vee}_{X,p} = \mathrm{Hom}_{\K}\left( T_{X,p}, \K \right)$}.

		We define the restriction morphism $\varphi: V^{\vee} \to T_{X,p}^{\vee}$
		that sends any linear form $l \in V^{\vee}$ to itself, as a linear form in $T^{\vee}_{X,p}$.
		This is clearly a linear map.
		Moreover $T^{\vee}_{X,p} = \mathrm{span} \left( \varphi(x_1), \ldots, \varphi(x_n) \right)$, 
		hence $\varphi$ is surjective.
		Finally $T_{X,p} = \mathbb{V}\left( \mathcal{I}_1 \right)$, hence
		$\ker \varphi = \mathcal{I}_1$.

	\item[Step 2:] \textit{There is a natural isomorphism}
		\begin{align}
			\psi: V^{\vee} / \mathcal{I}_1 &\to \mathfrak{m}_{X,p}/\mathfrak{m}^2_{X,p} \\
			x_j &\mapsto [x_j]
		.\end{align} 
		Recall that $p = 0$, hence any $x_j$ identifies an equivalence class in $\mathfrak{m}_{X,p}/\mathfrak{m}^2_{X,p}$.
		\textit{Injectivity:} Let $0 \neq \alpha \in \mathbb{K}\left[x_1, \ldots, x_n \right]_1$ be 
		s.t. $[\alpha] = 0 \in \mathfrak{m}_{X,p}/\mathfrak{m}^2_{X,p}$.
		Then $\alpha \in \mathcal{O}_{X,p}$ belongs to $\mathfrak{m}^2_{X,p}$, i.e.
		\begin{equation}
			\alpha = \frac{F_1}{G_1} \cdot \frac{F_2}{G_2} \in \K \left( x_1, \ldots, x_n \right)
		,\end{equation} 
		for $F_i, G_i \in \mathbb{K}\left[x_1, \ldots, x_n \right]$,
		with $F_i(p) = 0$ and $G_i(p) \neq 0$ for $i = 1,2$.
		Then
		\begin{equation}
		G_1G_2 \alpha = F_1F_2 \in \mathbb{K}\left[x_1, \ldots, x_n \right]
		.\end{equation} 
		Then, since $\alpha$ is of degree $1$, it is irreducible (hence prime).
		Since $G_1$ cannot divide $F_i$ (they do not vanish at $p$, then $\alpha$ divides the product $F_1F_2$,
		then $\alpha$ divides one of the two, let's say $F_1 = \alpha \widetilde{F_1}$.
		Then $G_1G_2 = \widetilde{F_1}F_2$, but
		\begin{equation}
			0 \neq \left( G_1G_2 \right)(p) = \left( \widetilde{F_1}F_2 \right)(p) = 
			\widetilde{F_1}(p) F_2(p) = 0
		.\end{equation} 
		This is a contradiction, hence $\alpha = 0$.

		\textit{Surjectivity:} Let $\phi = \frac{f}{g} \in \mathfrak{m}_{X,p}$, wlog $g(p) = 1$.
		We want to prove that
		\begin{equation}
			\phi' := \sum_{i}^{} \frac{\partial \phi}{\partial x_i} (p) x_i
			\qquad \text{ where } \qquad
			\frac{\partial \phi}{\partial x_i} =
			\frac{\frac{\partial f}{\partial x_i} \cdot g - \frac{\partial g}{\partial x_i} \cdot f }{g^2}
		\end{equation} 
		is the inverse image of $\phi$, i.e. $\psi(\phi') - \phi \in \mathfrak{m}^2_{X,p}$.
		Let's multiply by $g$, to obtain
		\begin{align}
			g \left( \phi' - \phi  \right) &=
			f - \frac{g}{g^2} \sum_{i = 1}^{n} \left( \frac{\partial f}{\partial x_i} (p) g(p) -
			\frac{\partial g}{\partial x_i} (p) f(p) \right)x_i\\
			&\equiv f - \frac{g(p)}{g^2(p)} \sum_{i=1}^{n} \frac{\partial f}{\partial x_i} (p) x_i
			\mod \mathfrak{m}^2_{X,p}\\
			&\equiv f - \sum_{i=1}^{n} \frac{\partial f}{\partial x_i} (p) x_i \mod \mathfrak{m}^2_{X,p}
		.\end{align} 
\end{description} 
\end{proof}

\begin{rem}[]
	The same proof yields
	\begin{equation}
	T_{X,p}^{\vee} \cong M/M^2 \qquad \text{ for } \qquad
	M = \mathbb{I}(p)/\mathbb{I}(X) \subset \K[X]
	.\end{equation} 
\end{rem}
\begin{rem}[]
	The identification $T_{X,p}^{\vee} \cong \mathfrak{m}_{X,p}/\mathfrak{m}^2_{X,p}$ is
	intrinsic, i.e. independent of the embedding in $\mathbb{A}^{n}$.
	In particular we can use it to define $T_{X,p}$ for $X$ any
	variety and $p \in X$ any point.
	In this case $T_{X,p}$ is an abstract Vector Space (not
	embedded in $\mathbb{A}^{n}$ anymore).
\end{rem}
 
\begin{rem}[Comparison with tangent cones]
	Let $X \subset \mathbb{A}^{n}$ be an affine variety, and $p \in X$.
	WLOG we assume that $p = \left( 0, \ldots, 0 \right)$.
	Then, for $f_0 + f_1 + \ldots + f_d = f \in \mathbb{K}\left[x_1, \ldots, x_n \right]$,
	with $f(p) = 0$ (i.e. $f_0 = 0$), we have
	$f_1 \neq 0 \implies \mathrm{mindeg}\, f = 1$ and $f_1 = f^{in}$.
	In particular $\mathbb{I}\left( X \right)_{in} = \left\{ f^{in} \ \middle|\ 
	f \in \mathbb{I}\left( X \right) \right\} \supset \mathbb{I}\left( X \right)_1$.
	Then $C_{X,p} = \mathbb{V}\left( \mathbb{I}\left( X \right)_{in} \right) \subset
	\mathbb{V}\left( \mathbb{I}\left( X \right)_1 \right) = T_{X,p}$ in $\mathbb{A}^{n}$.

	In particular we obtain that $\dim X = \dim C_{X,p} \leq \dim T_{X,p}$.
	The dimension of the tangent (linear space) might be bigger than the dimension of the cone.
\end{rem}

\begin{defn}[Regular and singular points/varieties]
	Given a variety $X$ we say that a point $p \in X$ is a \textbf{regular point} of $X$
	iff $\dim T_{X,p} = \dim X$.
	(Alternatively it can be called \textbf{simple}, \textbf{smooth}, \textbf{non-singular} point).
	If, instead, $\dim T_{X,p} > \dim X$, then $p$ is called a \textbf{singular point}.

	A variety $X$ is called \textbf{singular} iff it has at least one singular point.
	Non-singular varieties are allso called \textbf{regular} or \textbf{smooth}.
	
	Finally, we define the regular locus of $X$ to be
	\begin{equation}
	X_{\text{reg}} := \left\{ p \in X \ \middle|\ X \text{ is regular at } p \right\}
	\end{equation} 
	and the singular locus of $X$ as
	\begin{equation}
	X_{\text{sing}} := X \setminus X_{\text{reg}}
	.\end{equation} 
\end{defn}

\begin{ex}[Curves in $\mathbb{A}^{2}$]
	Consider the parabola $X_1 := \mathbb{V}\left( x (x-1) - y \right)$,
	the nodal cubic $X_2 := \mathbb{V}\left( x^2 + x^3 - y^2 \right)$
	and the cuspidal cubic $X_3 := \mathbb{V}\left( x^3 - y^2 \right)$.
	Then $T_{X_1, \mathbf{0}} = C_{X_1, \mathbf{0}}$, hence the parabola is regular at $\mathbf{0}$.
	Instead $T_{X_2, \mathbf{0}} = T_{X_3, \mathbf{0}} = \mathbb{A}^{2}$, 
	hence $X_2$ and $X_3$ are both singular at $\mathbf{0}$.

	In fact the blow-ups at $\mathbf{0}$ are as follows.
	For $\widetilde{X}_1 \to X_1$ is an isomorphism, in fact it coincides with the strict transform.
	For $\widetilde{X}_2 \to X_2$ it is not even injective: the two tangent directions corresponing
	to the two components of the tangent cone $C_{X_2, \mathbf{0}}$ get both mapped to $\mathbf{0}$.
	$\widetilde{X}_3 \to X_3$, instead, is injective.
	Indeed both $\widetilde{X}_2$ and $\widetilde{X}_3$ are smooth, whereas $X_2$ and $X_3$ are singular
	(and this gives the fact that the above maps cannot be iso).
	Moreover we call the above maps from the smooth varieties, obtained by
	blowing up the singular locus, a resolution of singularities.
	(Those are birational map inducing isomorphisms $p^{-1} \left( X_{i, \text{reg}} \right) \to X_{i, \text{reg}}$
	s.t. $\widetilde{X}_i$ is smooth).
	In particular the maps $\widetilde{X}_i \xrightarrow{p} X_i$ (for $i = 2,3$)
	are resolutions of the singularities at $\mathbf{0}$.

	Let's check, for instance, that $\widetilde{X}_3$ is non-singular.
	We only need to check what happens at the preimage of $\mathbf{0}$.
	\begin{equation}
		\widetilde{\mathbb{A}^{2}} = \left\{ \left((x,y), [\zeta,\eta]\right) \ \middle|\ 
		x \eta = y \zeta\right\} \subset \mathbb{A}^{2} \cross \mathbb{P}^{1}
	.\end{equation} 
	The local coordinates at $\left( p, [1,0] \right)$ are $(u,v) \in \mathbb{A}^{2}$.
	Recall that the chart we are using is induced by
	\begin{align}
		\mathbb{A}^{2} &\to \widetilde{\mathbb{A}^{2}} = \mathbb{A}^{2} \cross \mathbb{P}^{1} \\
		\left(u, v\right) &\mapsto \left((v, uv), [1, u] \right)
	\end{align} 
	for $u = \eta/\zeta$ (for $\zeta \neq 0$) and $v = x$.
	The equation for $X_3$ is $y^2 - x^3 = 0$.
	We substitute $x = v$ and $y = uv$ and obtain $(uv)^2 - v^3 = v^2 \left( u^2 - v \right) = 0$.
	This has two components:
	$v = 0$, which defines the exceptional line of $\widetilde{\mathbb{A}^{2}} \to \mathbb{A}^{2}$,
	which is not contained in the stric transform $\widetilde{X}_3$, by definition.
	And $u^2 - v = 0$, which gives the local equation for the strict transform $\widetilde{X}_3$.
	It has a nontrivial linear part, hence $\widetilde{X}_3$ is smooth at $\left(p, [1,0]\right)$.

	With similar computations one can check that $\widetilde{X}_2$ is smooth
	at $(u,v) = (\pm1, 0)$, since the local equations are $v = 1 - u^2 = (1-u)(1+u)$.
\end{ex} 

In general, one needs to be careful, since one needs more than
one blow-up to obtain a resolution of singularities.
\begin{defn}[Jacobi matrix]
	Given $F := \left( f_1, \ldots, f_r \right)$ an $r$-tuple of polynomials in $x_1, \ldots, x_n$,
	we define the \textbf{Jacobi matrix} of $F$ at $p$ as
	\begin{equation}
		J_{F,p} := \left( \frac{\partial f_i}{\partial x_j}  \right)_{i,j}
		\qquad \text{ for } i =1, \ldots, r \text{ and } j = 1, \ldots, n
	.\end{equation} 
\end{defn}

\begin{prop}[Jacobi criterion]\leavevmode\vspace{-.2\baselineskip}
	\begin{description}
		\item[Affine case:] Let $X \subset \mathbb{A}^{n}$ be an affine variety, and
			$f_1, \ldots, f_r$ be generators of $\mathbb{I}\left( C \right)$.
			Then $p \in X$ is regular iff
			\begin{equation}
			\mathrm{rk}\, J_{\left( f_1, \ldots, f_r \right), p} \geq n - \dim X
			.\end{equation} 
		\item[Projective case:] Let $X \subset \mathbb{P}^{n}$ be a projective variety
			and $f_1, \ldots, f_r$ homogeneous generators of $\mathbb{I}_p\left( X \right)$.
			Then $p \in X$ is regular iff
			\begin{equation}
			\mathrm{rk}\, J_{\left( f_1, \ldots, f_r \right), p} \geq n - \dim X
			.\end{equation} 
			Notice that, even though the Jacobi matrix itself depends on the choice of
			homogeneous coordinates for $p$, its rank does not change (each row gets multiplied by
			a nonzero constant).
	\end{description}
	Moreover, in both cases, if $\mathrm{rk}\, J_{F,p} = r$,
	then $X$ is smooth of dimension $n - r$ at $p$.
\end{prop} 

\begin{rem}\leavevmode\vspace{-.2\baselineskip}
	\begin{itemize}
		\item Set $d := \dim X$, then in the above notation
			\begin{align}
				X_{\text{sing}} &= \left\{ p \in X \ \middle|\ 
			\mathrm{rk}\, J_{\left( f_1, \ldots, f_r \right), p} \leq n-d-1\right\}\\
				&= \mathbb{V}\left( \left\{ (n-d) \cross (n-d) \text{-minors of }
				J_{\left( f_1, \ldots, f_r \right), p}\right\} \right)		
			,\end{align} 
			hence $X_{\text{sing}} \subset X$ is a closed subset.
		\item $X_{\text{reg}} \subset X$ is always a nonempty open subset.
			In particular it is always dense.
			Let's give a sketch of the argument:
			Since $d := \dim X$, there is a surjective projection on $\mathbb{A}^{d}$.
			Though let's stop a step earlier and consider
			$\pi: X \to X' \subset \mathbb{A}^{d+1}$. Clearly $X'$ is a
			hypersurface in $\mathbb{A}^{d+1}$.
			Making a good choice of the projection there is a non empty open
			subset of $X$, on which $\pi$ is an isomorphism.
			Hence $\pi$ will be a birational map (it is a consequence of the
			primitive element theorem).
			It is easy to check that $X'_{\text{reg}} \neq \emptyset$, for hypersurfaces.
			Then (one should check that the inclusion holds)
			$X_{\text{reg}} \supset \pi^{-1} \left( X'_{\text{reg}} \cap V \right)$
			(for $V = \pi(U)$) is non empty.
	\end{itemize}
\end{rem} 

\begin{rem}[]
	If $X$ is a hypersurface, and
	\begin{itemize}
		\item $X = \mathbb{V}\left( f \right) \subset \mathbb{A}^{n}$, for a square-free polynomial $f$
			(not necessairily irreducible).
			A point $p \in \mathbb{A}^{n}$ is a singular point of $X$ iff
			\begin{equation}
				f(p) = \frac{\partial f}{\partial x_1} (p) = \ldots
				= \frac{\partial f}{\partial x_n} (p) = 0
			.\end{equation} 
		\item $X = \mathbb{V}_p\left( f \right) \subset \mathbb{P}^{n}$, for a suqare-free 
			and homogeneous polynomial $f$. Then
			\begin{equation}
			p \in X_{\text{sing}} \iff
			\frac{\partial f}{\partial x_0} (p) = \frac{\partial f}{\partial x_1} (p) =
			\ldots = \frac{\partial f}{\partial x_n} (p) = f(p) = 0
			.\end{equation} 
			Moreover one can easily check that, set $d := \deg f$,
			multiplying each partial derivative by the corresponding linear form
			\begin{equation}
				d \cdot f (x) = 
				x_0 \frac{\partial f}{\partial x_0} (x) + \ldots +
				x_n \frac{\partial f}{\partial x_n}
			.\end{equation} 
			(Clearly this only works for homogeneous polynomials).
			Moreover, if $d$ is invertible in the ground field (e.g.
			if the field is of char $0$), then one can omit the condition $f(p) = 0$.
			Then we have
			\begin{equation}
			X_{\text{sing}} = \mathbb{V}_p\left( \frac{\partial f}{\partial x_0}, \ldots,
			\frac{\partial f}{\partial x_n} \right)
			.\end{equation} 
	\end{itemize}
\end{rem}

\begin{ex}[Fermat hypersurfaces]
	Consider $n,d \geq 1$. The Fermat hypersurfaces of degree $d$ in $\mathbb{P}^{n}$ are defined by
	\begin{equation}
	X_d := \mathbb{V}_p\left( x_0^d + \ldots + x_n^d \right) \subset \mathbb{P}^{n}
	.\end{equation} 
	This is a degree $d$ hypersurface if $\mathrm{char}\, \K = 0, p$ for $p \nmid d$
	(otherwise we'd have $x_0^d + \ldots x_n^d = \left( x_0 + \ldots + x_n \right)^d$).
	Then the Jacobi matrix is
	\begin{equation}
		J_f = \left( d x_0^{d-1}, d x_1^{d-1}, \ldots, d x_n^{d-1} \right)
	.\end{equation} 
	Hence $X_d$ is a smooth hypersurface by the Jacobi criterion.
\end{ex}  

\begin{ex}[Twisted cubic curve]
	\begin{equation}
		X = \left\{ [s^3, s^2t, st^2, t^3] \ \middle|\ [s,t] \in \mathbb{P}^{1} \right\}
	.\end{equation} 
	Since $X \cong \mathbb{P}^{1}$, we already know that $X$ is regular.
	Let's check it with the Jacobi criterions:
	\begin{equation}
	X = \mathbb{V}_p\left( x_0x_2 - x_1^2, x_0x_3 - x_1x_2, x_1x_3 - x_2^2 \right)
	\end{equation} 
	(the above is determined by the determinant of the minors of the matrix 
	$\begin{pmatrix} x_0 & x_1 & x_2\\ x_1 & x_2 & x_3 \end{pmatrix}$).
	The Jacobi matrix is
	\begin{equation}
	J = 
	\begin{pmatrix}
		x_2 & -2 x_1 & x_0 & 0\\
		x_3 & -x_2 & -x_1 & x_0\\
		0 & x_3 & -2 x_2 & x_1\\
	\end{pmatrix} 
	.\end{equation} 
	We need to check that, at every point $\mathrm{rk}\, J \geq \dim \mathbb{P}^{3} - \dim X = 3 - 1 = 2$.
	We need to find $2 \cross 2$ minors that do not vanish simultaneosly:
	\begin{equation}
	\begin{vmatrix}
		x_0 & 0\\
		-x_1 & x_0
	\end{vmatrix} = x_0^2
	\qquad \text{ and } \qquad
	\begin{vmatrix}
		x_3 & -x_2 \\
		0 & x_3
	\end{vmatrix} = x_3
	.\end{equation} 
	The first vanishes only at $[0,0,0,1] \in X$, whereas the second at $[1,0,0,0] \in X$, 
	i.e. they have no common zeros on $X$, hence $\mathrm{rk}\, J \geq 2$ at all $p \in X$.
	In particular $X$ is regular.
\end{ex} 	

\section{Affine schemes}
The idea is to extend the category of prevarieties in order to
\begin{itemize}
	\item include also reducible objects,
	\item expand the (Nullstellensatz) one to one correspondance also to non-radical ideals:
	\begin{equation}
	\begin{tikzcd}[row sep=tiny]
			\left\{\begin{matrix}
				\text{ affive viarieties }\\
				\text{ in } \mathbb{A}^{n}(\K)
			\end{matrix}\right\} \arrow[rr, "", leftrightarrow] & &
			\left\{  \begin{matrix}
				\text{ radical ideals }\\
				\text{ in } \mathbb{K}\left[x_1, \ldots, x_n \right]\\
			\end{matrix}\right\}
	\end{tikzcd}
	.\end{equation} 
\end{itemize}		
In fact, for $X_1, X_2 \subset \mathbb{A}^{n}$, the ideal $\mathbb{I}\left( X_1 \right) + \mathbb{I}\left( X_2 \right)$
contains more refined information then the associated ideal to the intersection $X_1 \cap X_2$
and, in general, it is not radical.
\begin{equation}
\mathbb{I}\left( X_1 \cap X_2 \right) =
\sqrt{\mathbb{I}\left( X_1 \right) + \mathbb{I}\left( X_2 \right)}
.\end{equation} 
For example, given $X_1' := \mathbb{V}\left( x^3 - y \right)$ and $X_2 := \mathbb{V}\left( x \right)$,
then $\mathbb{I}\left( X_1' \right) + \mathbb{I}\left( X_2 \right) = \left( x,y \right)$.
If, instead, we consider the curve $X_1 := \mathbb{V}\left( y^2 - x \right)$,
then $\mathbb{I}\left( X_1 \right) + \mathbb{I}\left( X_2 \right) = \left( x, y^2 \right)$.

We have two different geometrical situations (transverse intersection, against non-transverse intersection),
signified by different sums of associated ideals, but clearly with the same radical ideal.

For instance, in the second case,
if we blow up $\mathbb{A}^{2}$ at $\mathbb{I}\left( X_1 \right) + \mathbb{I}\left( X_2 \right)$,
then the strict transforms $\widetilde{X}_1$ and $\widetilde{X}_2$ are disjoint.
If, instead, we blow up $\mathbb{I}\left( X_1 \cap X_2 \right) = \sqrt{\mathbb{I}\left( X_1 \right) + \mathbb{I}\left( X_2 \right)}$,
then $\widetilde{X}_1 \cap \widetilde{X}_2 \neq 0$.
Another reason why the radical ideal is not always the right choice.

Moreover, from
	\begin{equation}
	\begin{tikzcd}[row sep=tiny]
			\left\{\begin{matrix}
				\text{ Affine }\\
				\text{ varieties }
			\end{matrix}\right\}_{/ \text{isom}}
			\arrow[rr, "", leftrightarrow] & &
			\left\{  \begin{matrix}
				\text{ finitely generated } \K\text{-algebras }\\
				\text{ that are integral domains}
			\end{matrix}\right\}_{/ \text{isom}}
	\end{tikzcd}
	,\end{equation} 
we will drop all assumptions on the right hand side
and associate a ringed space (i.e. a space on which we can do geometry)
to any commutative ring with unity.
(The idea is that, when defining the Zariski topology, we didn't use the $\K$-algebra
structure, but only ideals, hence the ring structure).

\begin{defn}[Spectrum of a ring]
	Let $R$ be a ring.
	We define the \textbf{spectrum} of $R$ to be
	\begin{equation}
	\mathrm{Spec}\, R := \left\{ \mathfrak{p} \triangleleft R \ \middle|\ 
	\mathfrak{p} \text{ is a prime ideal}\, \right\}
	.\end{equation} 
	$\mathrm{Spec}\, R$ is also called the \textbf{affine scheme} associated with $R$.
	Moreover, for every $\mathfrak{p} \in \mathrm{Spec}\, R$, we define the residue field
	of $\mathrm{Spec}\, R$ at $\mathfrak{p}$ to be
	\begin{equation}
	k(\mathfrak{p}) := Q \left( R/\mathfrak{p} \right)
	.\end{equation} 
	(Notice that we will use the convention that $(1) = R$ is not a prime ideal).

	Moreover we can define $\mathrm{MaxSpec}\, R := \left\{ \mathfrak{m} \triangleleft R \ \middle|\ 
	\mathfrak{m} \text{ is a maximal ideal}\, \right\}$ the maximal specturm of $R$.
\end{defn}

\begin{rem}[]
	$R$ is the analogue of the coordinate ring:
	we can view any element $f \in R$ as a function on $\mathrm{Spec}\, R$ by
	\begin{equation}
	\begin{tikzcd}[row sep=tiny, column sep=small]
		R \arrow[r, "", twoheadrightarrow] &
		R/\mathfrak{p} \arrow[r, "", hookrightarrow] &
		k(\mathfrak{p})\\
		f \arrow[rr, "", mapsto] && f(\mathfrak{p})
	\end{tikzcd}
	.\end{equation} 
	Hence $f$ can be regarded as a function, although $f(\mathfrak{p})$ might lie
	in different fields, depending on $\mathfrak{p}$.

	If, in particular, $R = \mathbb{K}[X]$, for some affine variety $X$,
	we can use point $x \in X$ to construct prime ideals, namely $\mathfrak{p} = \mathfrak{m}_x := \mathbb{I}\left( x \right) \in \mathrm{Spec}\, R$.
	In particular $\mathbb{K}[X]/\mathfrak{m}_x \cong \K$.
	Then, for all $f \in \mathbb{K}[X]$, we have $f(\mathfrak{p}) = f(x) \in \K$
	in which the left hand side intereprets $f$ as a function on $\mathrm{Spec}\, R$,
	whereas the right hand side interepresentationets it as a function on $X$.

	If, instead, $\mathfrak{p} \subset \mathbb{K}[X]$ is prime, but not maximal, then
	$\mathfrak{p} = \mathbb{I}\left( Y \right) \subset \mathbb{K}[X]$, for
	some irreducible closed subset $Y \subset X$.
	Then $f(\mathfrak{p}) \in k(\mathfrak{p}) = \K(Y)$ is the restriction of
	$f: X \to \K$ to $Y$.
\end{rem}

\begin{ex}\leavevmode\vspace{-.2\baselineskip}
\begin{enumerate}
\item If $\K$ is a field (not necessairily algebraically closed),
	then $\mathrm{Spec}\, \K = \left\{ (0) \right\}$ is a single point.
\item $\mathrm{Spec}\, \mathbb{C}[x] = \left\{ (x-a) \ \middle|\ a \in \mathbb{C} \right\} \cup \left\{ (0) \right\}$.
	The same description holds for any algebraically closed field $\mathbb{K}$ instead of $\mathbb{C}$.
\item Let $X$ be an affine variety over the algebraically closed field $\K$.
	Consider $R := \mathbb{K}[X]$. Then we have the bijection
	\begin{equation}
	\begin{tikzcd}[row sep=tiny]
		\mathrm{Spec}\, R \arrow[r, "", leftrightarrow] &
		\left\{ \text{subvarieties of } X \right\}\\
		\mathbb{I}\left( Y \right) &
		Y \arrow[l, "", rightarrow] 
	\end{tikzcd}
	\end{equation} 
	in which subvarieties are exactly the irreducible closed subsets of $X$.
	Then, restricting to the maximal spectrum of $R$, we obtain the correspondance
	\begin{equation}
	\begin{tikzcd}[row sep=tiny]
		\mathrm{MaxSpec}\, R \arrow[r, "", leftrightarrow] &
		X\\
		\mathfrak{m}_x &
		x \arrow[l, "", rightarrow] 
	\end{tikzcd}
	\end{equation} 
	The element $\mathbb{I}\left( Y \right) \in \mathrm{Spec}\, R$ is called
	\textbf{generic point} of $Y$.
	If we don't want to consider the additional points of $\mathrm{Spec}\, R$ coming
	from $\mathbb{I}\left( Y \right)$, for $\dim Y \geq 1$,
	we may want to restrict to the maximal spectrum of $R$.

\item Let's consider the special case in which the field is not algebraically closed, i.e. $\R$:
	\begin{equation}
	\mathrm{Spec}\, \R =
	\left\{ (x-a) \ \middle|\ a \in \R \right\} \cup
	\left\{ \left( (x-a)^2 + b^2 \right) \ \middle|\ a,b \in \R \text{ and } b > 0 \right\} \cup
	\left\{ (0) \right\}
	.\end{equation} 
	Then the residue fields are:
	\begin{enumerate}
		\item $k \left( (x-a) \right) \cong \R$,
		\item $k \left( \left( (x-a)^2 + b^2 \right) \right) \cong
			k \left( (x^2 + 1) \right) \cong \mathbb{C}$ (by scaling things),
		\item $k \left( (0) \right) \cong \R(x)$.
	\end{enumerate}
\item $\mathrm{Spec}\, \Z = \left\{ (2), (3), \ldots, (p), \ldots \right\} \cup \left\{ (0) \right\}$
	for $p$ prime.
	$(0)$ is the generic point of $\Z$.
	Then $k \left( (p) \right) \cong Q \left( \mathbb{Z}/p\mathbb{Z} \right) = \mathbb{F}_p$
	and, clearly, $k \left( (0) \right) = \mathbb{Q}$.

	FInally, by definition, given $f \in \Z$, $f \left( (p) \right) = f \mod p \in \mathbb{F}_p$.
\end{enumerate}
\end{ex} 

We want to defien a ringed space for $\mathrm{Spec}\, R$, hence we need to
define a topology on it. Ideally we should mimic the Zariski topology on affine varieties.

\begin{defn}[Vanishing locus]
	Let $R$ be a ring and $S \subset R$.
	We define the \textbf{vanishing locus} of $S$ as
	\begin{align}
		\mathbb{V}\left( S \right) :&= \left\{ \mathfrak{p} \in \mathrm{Spec}\, R \ \middle|\ 
	f(\mathfrak{p}) = 0 \text{ for all } f \in S \right\}\\
		&= \left\{ \mathfrak{p} \in \mathrm{Spec}\, R \ \middle|\ 
		S \subset \mathfrak{p} \right\} \subset \mathrm{Spec}\, R
	.\end{align} 
\end{defn}

\begin{rem}[]
	We clearly have $\mathbb{V}\left( S \right) = \mathbb{V}\left( (S) \right)$,
	where we denote the ideal generated by $S$ by $(S)$.
\end{rem}

\begin{lem}\leavevmode\vspace{-.2\baselineskip}
\begin{enumerate}
	\item $\mathbb{V}\left( 0 \right) = \mathrm{Spec}\, R$ and $\mathbb{V}\left( 1 \right) = \emptyset$.
	\item If $\left\{ \mathcal{I}_j \right\}_{j \in J}$ is an arbitrary family
		of ideals, then
		\begin{equation}
		\bigcap_{j \in J} \mathbb{V}\left( \mathcal{I}_j \right) =
		\mathbb{V}\left( \sum_{j \in J}^{} \mathcal{I}_j \right) \subset \mathrm{Spec}\, R
		.\end{equation} 
	\item Given $\mathcal{I}_1, \mathcal{I}_2 \triangleleft R$, then
		\begin{equation}
		\mathbb{V}\left( \mathcal{I}_1 \right) \cup \mathbb{V}\left( \mathcal{I}_2 \right) =
		\mathbb{V}\left( \mathcal{I}_1 \cdot \mathcal{I}_2 \right) \subset \mathrm{Spec}\, R
		.\end{equation} 
	\item Given $\mathcal{I}_1, \mathcal{I}_2 \triangleleft R$, we have
		\begin{equation}
		\mathbb{V}\left( \mathcal{I}_1 \right) \subset \mathbb{V}\left( \mathcal{I}_2 \right) \iff
		\sqrt{\mathcal{I}_2} \subset \sqrt{\mathcal{I}_1}
		.\end{equation} 
\end{enumerate}
\end{lem}

\begin{defn}[Zariski topology]
	The topology on $\mathrm{Spec}\, R$, whose closed subsets are of the form
	$\mathbb{V}\left( S \right)$, for $S \subset R$, is called the \textbf{Zariski topology}.
\end{defn}
\begin{rem}[]
	We can use the Zariski topology to study the irreducibility of $\mathrm{Spec}\, R$
	and to define its dimension.
\end{rem}

\begin{rem}[]
	Not all point of $\mathrm{Spec}\, R$ are closed.
	In fact we have
	\begin{equation}
	\overline{\left\{ \mathfrak{p} \right\}} = \mathbb{V}\left( \mathfrak{p} \right) =
	\left\{ \mathfrak{q} \in \mathrm{Spec}\, R \ \middle|\ \mathfrak{q} \supset \mathfrak{p} \right\}
	.\end{equation} 
	Then $\left\{ \mathfrak{p} \right\}$ is closed iff $\mathfrak{p}$ is a maximal ideal.
	In particular, in $\mathrm{Spec}\, R$, the closed points are maximal ideals (i.e. the elements
	that correspond to points in the ring of functions on an affine variety).
	In other words $\left\{ \text{closed points} \right\} = \mathrm{MaxSpec}\, R$.
	Whereas non-closed points are generic points of $\mathbb{V}\left( \mathfrak{p} \right)$.
\end{rem}

\begin{ex}[Generic points]
	Let $\K$ be an algebraically closed field.
	Consider $X := \mathrm{Spec}\, \mathbb{K}\left[x_1, x_2 \right]$
	the spectrum of the coordinate ring of the affine plane.
	Consider $L := \mathbb{V}\left( x_2 \right) \subset X$ the $x_1$-axis.
	What is $X \setminus L$?
	It contains
	\begin{itemize}
		\item all closed points not lying in $L$,
		\item generic points (e.g. $\mathfrak{p} = x_1$, the $x_2$-axis).
	\end{itemize}
	The $x_1$-axis and the $x_2$-axis have common points, but
	the generic points of the $x_2$-axis does not lie on the $x_1$-axis,
	so that $(x_1) \in X \setminus L$.
\end{ex} 

\begin{defn}[Distinguished open subsets]
	Let $R$ be a ring, and $X := \mathrm{Spec}\, R$.
	The open subsets of the form $X_f := X \setminus \mathbb{V}\left( f \right)$, for some $f \in R$
	are called the \textbf{distinguished open subsets} of $X$.
\end{defn}

\begin{rem}[]
	As for varieties, the distinguished open subsets will form a basis for the Zariski topology.
	Though, since we don't know whether this Zariski topology makes the spectrum of the ring $R$
	a Noetherian space, we cannot say that every open subset can be obtained by
	finite union of distinguished open subsets.
\end{rem}

\begin{defn}[Regular functions on an open subset]
	Let $R$ be a ring, $X = \mathrm{Spec}\, R$ and $U \subset X$ an open subset.
	We define the ring of regular functions on $U$, denoted by $\mathcal{O}_{X} \left( U \right)$,
	as
	\begin{align}
		\mathcal{O}_{X} \left( U \right) := \bigg\{ &\varphi = \left( \varphi_{\mathfrak{p}} \right)_{\mathfrak{p} \in U}
		\in \prod_{\mathfrak{p} \in U} R_{\mathfrak{p}} \ \bigg|\ \text{for every }
		\mathfrak{p} \in U \text{ there is } V \subset U\\
		&\text{an open neighbourhood of } \mathfrak{p} \text{ and } f,g \in R, 
		\text{ with } g \notin Q,\\
		&\varphi_Q = \frac{f}{g} \in R_Q \text{ for all } Q \in V \bigg\}
	.\end{align} 
	The idea is thet a regular function on $U$ should locally be a quotient $f/g$,
	with $f,g \in R$. In particular $g$ should be ``ivertible" at some $Q \in U$, i.e.
	$Q \notin \mathbb{V}\left( g \right)$.
	Equivalently $g \notin Q$, for some $Q \in U$.
\end{defn}

\begin{rem}[Structure sheaf]
	It is easy to see that $\mathcal{O}_{X}$ is a sheaf on $X$ (the defining condition
	on $\varphi$ is local).
	Moreover we call it the {\em structure sheaf} on $X$.
\end{rem}

\begin{prop}
	Let $R$ be a ring and $X = \mathrm{Spec}\, R$.
	\begin{enumerate}
		\item For all $\mathfrak{p} \in X$ we have $\mathcal{O}_{X, \mathfrak{p}} \simeq R_{\mathfrak{p}}$
			(the stalk of $\mathcal{O}_X$ at $\mathfrak{p}$ is isomorphic to the
			localization of $R$ at $\mathfrak{p}$),
		\item For any $f \in R$, $\mathcal{O}_{X} \left( X_f \right) \simeq R_f$.
	\end{enumerate}
	In particular, for $f = 1$, we obtain $\mathcal{O}_{X} \left( X \right) \simeq R$.
\end{prop} 

\begin{rem}[]
	A regular function is not determined by its value at each point of $X := \mathrm{Spec}\, R.$
\end{rem}

\begin{ex}
	Let $R := \K[x]/ (x^2)$ and $X := \mathrm{Spec}\, R$.
	Clearly $R$ is not a domain and $X$ contains only the point $(x)$,
	with residue field $k \left( (x) \right) \cong \mathbb{K}[x]/ (x) \cong \K$.
	Since $R \cong \mathcal{O}_{X} \left( X \right)$ we have that $0$ and $x \in R$ define different functions on $X$.
	However both functions have value $0$ at the only point of $X$.
\end{ex} 

\subsection{Locally ringed spaces}
\begin{rem}[]
To define morphisms between affine schemes one needs to keep in mind that
the structure sheaf is not necessairily a sheaf of $\K$-valued functions.
This means that, if we want to define a morphism $f: X \to Y$ between affine schemes,
we need to also define the pull-back maps
\begin{equation}
	f^*: \mathcal{O}_{Y} \left( U \right) \to \mathcal{O}_{X} \left( f^{-1}(U) \right)
\end{equation} 
in the data required to define $f$.

Recall, moreover, that the pull-back should:
\begin{itemize}
	\item be compatible with restriction morphisms, i.e. for all $V \stackrel{\text{open}}{\subset} U$ in $Y$,
		the following diagram should commute
		\begin{equation}
			\begin{tikzcd}[column sep=large]
			\mathcal{O}_{Y} \left( U \right) \arrow[r, "\rho_{U,V}", rightarrow] \arrow[d, "f^*_U"', rightarrow] &
			\mathcal{O}_{Y} \left( V \right) \arrow[d, "f^*_{V}", rightarrow] \\
			\mathcal{O}_{X} \left( f^{-1}(U) \right) \arrow[r, "\rho_{f^{-1}(U), f^{-1}(V)}"', rightarrow] &
			\mathcal{O}_{X} \left( f^{-1}(V) \right)
		\end{tikzcd}
		;\end{equation} 
	\item preserve the structure of the stalks as local rings:
		$\mathcal{O}_{X, p}$ has a unique macimal ideal
		\begin{equation}
		\mathfrak{m}_{X, p} := \left\{ \phi \in \mathcal{O}_{X, p}
		\ \middle|\ \phi(p) = 0 \right\}
		\end{equation} 
		and $\mathcal{O}_{Y, f(p)}$ has maximal ideal $\mathfrak{m}_{Y, f(p)}$.
		Then we require that the induced map
		\begin{align}
			f^*_{p}: \mathcal{O}_{Y, f(p)} &\to \mathcal{O}_{X, p} \\
			\left(U, \phi\right) &\mapsto \left(f^{-1}(U), f^* \phi\right)
		\end{align} 
		satisfies $\left( f^*_{p} \right)^{-1}(\mathfrak{m}_{X, p}) =
		\mathfrak{m}_{Y, f(p)}$.
\end{itemize}
\end{rem}

\begin{defn}[Locally ringed space]
	A \textbf{locally ringed space} is a ringed space $\left( X, \mathcal{O}_{ X } \right)$ s.t.
	at all $p \in X$ the stalk $\mathfrak{O}_{X, p}$ is a local ring,
	wth maximal ideal denoted by $\mathfrak{m}_{X,p}$
	and residue field $k(p) := \mathcal{O}_{X,p}/\mathfrak{m}_{X,p}$.
\end{defn}

\begin{defn}[Morphism of locally ringed spaces]
	Consider two locally ringed spaces $\left( X, \mathcal{O}_{ X } \right)$ and $\left( Y, \mathcal{O}_{ Y } \right)$.
	A morphism of locally ringed spaces is given by 
	a continuous map $f: X \to Y$ and by a family, for all $U \subset Y$ open, of ring homomorphisms
	\begin{equation}
		f^*_U: \mathcal{O}_{Y} \left( U \right) \to \mathcal{O}_{X} \left( f^{-1}(U) \right)
	\end{equation} 
	satisfying
	\begin{enumerate}
		\item compatibility with restrictions, i.e.
			for all $U \subset V \subset Y$ open, the diagram commutes
			\begin{equation}
			\begin{tikzcd}
				\mathcal{O}_{Y} \left( V \right) \arrow[r, "f^*_V", rightarrow] 
				\arrow[d, "\rho_{V,U}"', rightarrow] &
				\mathcal{O}_{X} \left( f^{-1}(V) \right)
				\arrow[d, "\rho_{f^{-1}(V), f^{-1}(U)}", rightarrow] \\
				\mathcal{O}_{Y} \left( U \right) \arrow[r, "f^*_U"', rightarrow] &
				\mathcal{O}_{X} \left( f^{-1}(U) \right)
			\end{tikzcd}
			,\end{equation} 
		\item the induced maps $f^*_p : \mathcal{O}_{Y, f(p)} \to \mathcal{O}_{X,p}$
			(whose existance is granted by the previous property)
			satisfy $\left( f^*_p \right)^{-1}(\mathfrak{m}_{X,p}) = \mathfrak{m}_{Y, f(p)}$
			for all $p \in X$.
	\end{enumerate}
	Notice that, in general, the subset $U \subset Y$ will be clear from the context, hence
	one will write just $f^*$, instead of $f^*_U$.
\end{defn}

\begin{defn}[Morphism of affine schemes]
	A morphism of affine schemes is a morphism of locally
	ringed spaces, between affine schemes.
\end{defn}

\begin{prop}
	Let $R,S$ be rings, $X := \mathrm{Spec}\, R$ and $Y := \mathrm{Spec}\, S$.
	Then there is a one-to-one correspondance
	\begin{equation}
		\begin{tikzcd}[row sep=tiny]
		\left\{ \text{morphisms } X \to Y \right\} \arrow[r, "", leftrightarrow] &
		\left\{ \text{ring homomorphisms } S \to R \right\}\\
		f \arrow[r, "", mapsto] & f^*\\
		g^\# & g \arrow[l, "", mapsto]
	\end{tikzcd}
	,\end{equation} 
	in which, for a ring homomorphism $g: S \to R$ we define
	\begin{align}
		g^{\#}: X = \mathrm{Spec}\, R &\to Y = \mathrm{Spec}\, S \\
		\mathfrak{p} &\mapsto g^{-1}(\mathfrak{p})
	.\end{align} 
\end{prop} 

\begin{ex}
	Let $X := \mathrm{Spec}\, R$ and $\mathcal{I} \triangleleft R$ an ideal.
	Define $Y := \mathrm{Spec}\, R/\mathcal{I}$.
	The ring homomorphism $q: R \to R/\mathcal{I}$ gives rise to a morphism
	\begin{align}
		i: Y &\to X \\
		\mathfrak{p} &\mapsto \mathfrak{p} + \mathcal{I} = q^{-1}(\mathfrak{p})
	.\end{align} 
	Let us observe that $i = q^{\#}$ gives a one-to-one correspondance
	between prime ideals of $R/\mathcal{I}$ and prime ideals of $R$ containing $\mathcal{I}$.
	Thus $i$ is injective and its image is $\mathbb{V}\left( \mathcal{I} \right) \subset X$.
	In other words: ideals of $R$ give rise to affine closed subschemes of $X$.
\end{ex} 

\begin{defn}[Intersection scheme]
	Let $Y_1$ and $Y_2$ be affine subschemes of $X$, then the intersection scheme
	of $Y_1$ and $Y_2$ in $X$ is
	\begin{equation}
	Y_1 \cap Y_2 := \mathrm{Spec}\, \frac{R}{\mathcal{I}_1 + \mathcal{I}_2}
	.\end{equation} 
\end{defn}

\begin{defn}[Union scheme]
	Let $Y_1$ and $Y_2$ be affine subschemes of $X$, then the union scheme
	of $Y_1$ and $Y_2$ in $X$ is
	\begin{equation}
	Y_1 \cup Y_2 := \mathrm{Spec}\, \frac{R}{\mathcal{I}_1 \cap \mathcal{I}_2}
	.\end{equation} 
\end{defn}

\begin{ex}[Affine plane]
	Consider the affine plane $X := \mathrm{Spec}\, \mathbb{C}[x_1,x_2]$
	and $Y_1 := \mathrm{Spec}\, \mathbb{C}[x_1,x_2]/ (x_2)$ (the $x$ axis),
	$Y_2 := \mathrm{Spec}\, \mathbb{C}[x_1,x_2]/ (x_2 - x_1^2+a^2)$ for a fixed $a \in \mathbb{C}$ (a parabola).
	Then
	\begin{equation}
	Y_1 \cap Y_2 =
	\mathrm{Spec}\, \mathbb{C}[x_1,x_2]/ (x_2, x_1^2-a^2) =
	\mathrm{Spec}\, \mathbb{C}[x_1]/ ( (x_1-a)(x_1+a))
	.\end{equation} 
	If, in particular $a=0$, $Y_1 \cap Y_2 \cong \mathrm{Spec}\, \mathbb{C}[x]/ (x^2)$:
	it is a double point, it encodes the point $(0,0)$ togethere with a tangent direction
	(along the $x_1$-axis).
\end{ex} 

\begin{lem}
	Let $R$ be a ring, $X := \mathrm{Spec}\, R$, and $f \in R$.
	Then
	\begin{equation}
	X_f \cong \mathrm{Spec}\, R_f
	.\end{equation} 
\end{lem} 

\section{Schemes}
When gluing together affine varieties one obtaines prevarieties.
Let's define, in an analogous way, schemes, as locally ringed spaces which can be covered by
affine schemes.

\begin{defn}[Scheme]
	A \textbf{scheme} is a locally ringed space $\left( X, \mathcal{O}_{ X } \right)$
	that can be covered by open subsets $U_i \subset X$ s.t.
	\begin{equation}
	\left( U_i, \left.\mathcal{O}_X\right|_{U_i} \right) \cong
		\left( \mathrm{Spec}\, R, \mathcal{O}_{ \mathrm{Spec}\, R } \right)
	,\end{equation} 
	for some ring $R_i$ and for all $i$.

	Moreover a morphism of schemes is just a morphism of locally ringed
	spaces between schemes.
\end{defn}

\begin{rem}[]
	Each affine variety $X$ over a field $\K$ has an associated affine scheme
	\begin{equation}
	X_{\text{sch}} := \mathrm{Spec}\, \mathbb{K}[X]
	.\end{equation} 
\end{rem}

\begin{defn}[Scheme associated to a prevariety]
	Let $\K$ be a algebraically closed field and
	$X$ a prevariety over $\K$.
	Then the scheme associated with $X$ is
	\begin{equation}
	X_{\text{sch}} :=
	\left\{ Z \subset X \ \middle|\ \emptyset \neq Z \text{ is irreducible} \right\}
	\end{equation} 
	with the topology defined by $\left\{ U_{\text{sch}} \ \middle|\ U \stackrel{\text{open}}{\subset} X \right\}$
	with
	\begin{align}
	U_{\text{sch}} := \left\{ V \subset U \ \middle|\ V \text{ is an 
	irreducible closed subset of } U \right\} &\hookrightarrow X_{\text{sch}}\\
	U \supset Z &\mapsto \overline{Z} \subset X	
	,\end{align} 
	and structure sheaf $\mathcal{O}_{X_{\text{sch}}} \left( U_{\text{sch}} \right) := \mathcal{O}_{X} \left( U \right)$,
	with natural restrictions.
\end{defn}

\begin{prop}
	$X_{\text{sch}}$ is a scheme for every prevariety $X$.
	Moreover any morphism of prevarieties $f: X \to Y$
	naturally extends to a morphism of schemes
	\begin{align}
		f: X_{\text{sch}} &\to Y_{\text{sch}} \\
		Z &\mapsto \overline{f(Z)}
	\end{align} 
	for $Z \subset X$ an irreducible closed subset.
\end{prop} 

Moreover a scheme $X$ comes from a prevariety over $\K$ if
locally it is given by $\mathrm{Spec}\, R_i$, for $R_i = \mathbb{K}[Y_i]$
for $Y_i$ an affine variety over $\K$.
(More explicitly $R_i$ are all finitely generated $\K$-algebras which are also domains).

\begin{defn}[]
	Let $Y$ be a scheme.
	A scheme over $Y$ is a scheme $X$, together with a morphism $f: X \to Y$.
	A morphism of schemes $X_1 \to Y$, $X_2 \to Y$ over $Y$
	is a morphism of schemes s.t. the following diagram commutes
	\begin{equation}
	\begin{tikzcd}
		X_1 \arrow[r, "f", rightarrow] \arrow[d, "", rightarrow] &
		X_2 \arrow[d, "", rightarrow] \\
		Y \arrow[r, "", equal] &
		Y
	\end{tikzcd}
	.\end{equation} 
	If $Y = \mathrm{Spec}\, R$, then we call schemes over $Y$, schemes over $R$.
\end{defn}

\begin{defn}[Scheme over $Y$ of finite type]
	A scheme $X$ over $Y$ is of finite type iff there is a covering of $Y$
	by affine open subsets $V_i := \mathrm{Spec}\, B_i$
	and a covering of $f^{-1}(V_i)$ by affine open subsets
	\begin{equation}
	U_{ij} := \mathrm{Spec}\, A_{ij} \qquad \text{ for all indeces } i
	\end{equation} 
	with the property that each $A_{ij}$ is a finitely generated $B_i$-algebra
	(induced by $U_{ij} \to V_i$).
\end{defn}

\begin{rem}[]
	Recall that, given a ring $B$, then a $B$-algebra is a ring $A$ together with
	an injective ring homomorphism $B \hookrightarrow  A$.
	The $B$-algebra $A$ is finitely generated iff there are finitely many
	$\alpha_1, \ldots, \alpha_k \in A$ s.t.
	\begin{align}
		B [t_1, \ldots, t_k] &\to A \\
		f &\mapsto f(\alpha_1, \ldots, \alpha_k)
	\end{align} 
	is surjective.

	Moreover recall that, if $X \subset \mathbb{A}^{n}$ is a Zariski closed subset,
	then $\mathbb{I}\left( X \right) \subset \mathbb{K}\left[x_1, \ldots, x_n \right]$
	is a radical ideal
	and
	\begin{equation}
	\mathbb{K}[X] = \frac{\mathbb{K}\left[x_1, \ldots, x_n \right]}{\mathbb{I}\left( X \right)}
	\end{equation} 
	does not contain any nilpotent element.
\end{rem}

\begin{defn}[Reduced ring]
	A ring $R$ is reduced iff, for every $f \in R$,
	\begin{equation}
	f^r = 0 \implies f = 0
	.\end{equation} 
\end{defn}

\begin{defn}[Reduced scheme]
	A scheme $X$ is called reduced iff $\mathcal{O}_{X} \left( U \right)$ is
	a reduced ring for all $U \stackrel{\text{open}}{\subset} X$.
\end{defn}

\begin{prop}
	Let $\K$ be an algebraically closed field, then
	\begin{itemize}
		\item For any affine variety $X$ over $\K$,
			the associated scheme $X_{\text{sch}} = \mathrm{Spec}\, \mathbb{K}[X]$
			is a reduced and irreducible affine scheme
			of finite type over $\K$.
			Moreover any reduced and irreducible affine scheme of finite type over
			$\K$ is of this form.
		\item Let $X,Y$ be affine schemes over $\K$, then there
			are one to one correspondences
			\begin{equation*}
			\begin{tikzcd}
				\left\{ 
				\begin{matrix}
					\text{morphisms}\\
					X \to Y\\
					\text{as varieties}
				\end{matrix} 
				\right\} \arrow[r, "1:1", leftrightarrow] &
				\left\{ 
				\begin{matrix}
					\K\text{-algebra}\\
					\text{homomorphisms}\\
					\mathbb{K}[X] \to \K[Y]
				\end{matrix} 
				\right\} \arrow[r, "1:1", leftrightarrow] &
				\left\{ 
				\begin{matrix}
					\text{morphisms as}\\
					\text{schemes over }\K\\
					X_{\text{sch}} \to Y_{\text{sch}}\\
				\end{matrix} 
				\right\} 
			\end{tikzcd}
			.\end{equation*} 
		In particular the category of affine $\K$-varieties is equivalent to
		the category of reduced and irreducible affine schemes of
		finite type over $\K$.
		Moreover both are equivalent to the opposite of the category of finitely 
		generated reduced $\K$-algebras.
	\end{itemize}
\end{prop} 

\begin{prop}
	Let $\K$ be an algebraically closed field.
	Then there is an equivalence of categories between the category of {\em prevarieties over $\K$}
	and the category of {\em reduced and irreducible schemes of finite type over $\K$}.
	
	In particualr all such schemes arise from prevarieties over $\K$.
	Moreover, for any two prevarieties $X$ and $Y$, we have a
	one to one correspondance
	\begin{equation*}
	\begin{tikzcd}
		\left\{ 
		\begin{matrix}
			\text{morphisms}\\
			X \to Y\\
			\text{as prevarieties}
		\end{matrix} 
		\right\} \arrow[r, "1:1", leftrightarrow] &
		\left\{ 
		\begin{matrix}
			\text{morphisms as}\\
			\text{schemes over }\K\\
			X_{\text{sch}} \to Y_{\text{sch}}\\
		\end{matrix} 
		\right\} 
	\end{tikzcd}
	.\end{equation*} 
\end{prop} 

\begin{rem}
	Schemes can be constructed by gluing other schemes.
	In fact we need
	\begin{itemize}
		\item a collection $\left\{ X_i \right\}_{i \in I}$ of shcemes ($I$
			may even be infinite)
		\item a collection $\left\{ U_{ij}, f_{ij} \right\}_{i,j \in I}$ of open subsets
			$U_{ij} \subset X$ and isomorphisms
			\begin{equation}
			f_{ij}: U_{ij} \to U_{ji}
			.\end{equation} 
	\end{itemize}
	All of these satifying the conditions:
	\begin{enumerate}
		\item $U_{ii} = X_i$ and $f_{ii} = id_{X_i}$ for all $i \in I$,
		\item $f_{ij} \left( U_{ij} \cap U_{ik} \right) \subset U_{jk}$ and
			$f_{ik} = f_{jk} \circ f_{ij}$
			for all $i,j,k \in I$.
	\end{enumerate}
	The case where $U_{ij}$ is empty gives rise to a reducible scheme.
	Then a morphism from the scheme $X$, obtained by gluing the $X_i$,
	to a scheme $Y$ is a collection of morphisms of scheme
	\begin{equation}
	\left\{ X_i \to Y \right\}_{i \in I} 
	\end{equation} 
	compatible with the overla maps $f_{ij}: U_{ij} \to U_{ji} \subset X_j$.
\end{rem} 

\begin{prop}
	Let $X$ be a scheme and $Y := \mathrm{Spec}\, R$ an affine scheme.
	Thene there is a one to one correspondence
	\begin{equation*}
	\begin{tikzcd}
		\left\{ 
		\begin{matrix}
			\text{morphisms}\\
			X \to Y\\
			\text{as schemes}
		\end{matrix} 
		\right\} \arrow[r, "1:1", leftrightarrow] &
		\left\{ 
		\begin{matrix}
			\text{ring homomorphisms}\\
			 \mathcal{O}_{Y} \left( Y \right) = R \to \mathcal{O}_{X} \left( X \right)\\
		\end{matrix} 
		\right\} 
	\end{tikzcd}
	.\end{equation*} 
\end{prop} 

\begin{rem}[]
	Every ring $R$ has a natural ring homomorphism
	\begin{align}
		\Z:  &\to R \\
		0 \leq m &\mapsto \underbrace{1 + 1 + \ldots + 1}_{m \text{ times}}\\
		0 > m &\mapsto \underbrace{- 1 - 1 - \ldots - 1}_{-m \text{ times}}
	.\end{align} 
	In view of the proposition, this yields that every scheme $X$ is a
	scheme over $\Z$, by using
	\begin{align}
		\Z &\to \mathcal{O}_{X} \left( X \right) \\
		X &\mapsto \mathrm{Spec}\, \Z
	.\end{align} 
	Is $X$ is a scheme over $\mathbb{C}$, then the image of
	this morphism is the point
	$(0) \in \mathrm{Spec}\, \Z$.
\end{rem}

\subsection{Fiber products}
\begin{defn}[Fiber product of schemes]
	Let $X \xrightarrow{f} S$ and $Y \xrightarrow{g} S$ be two schemes
	over a fixed scheme $S$.
	The fiber product $\left(X \cross_S Y, \pi_X, \pi_Y\right)$ is a scheme
	endowed with two morphisms s.t. the following diagram commutes
	\begin{equation}
	\begin{tikzcd}
		X \cross_S Y \arrow[r, "\pi_X", rightarrow] \arrow[d, "\pi_Y"', rightarrow] &
		X \arrow[d, "f", rightarrow] \\
		Y \arrow[r, "g"', rightarrow] &
		S
	\end{tikzcd}
	\end{equation} 
	and satisfies the following universal property.
	For every couple of morphism $\phi_X$ and $\phi_Y$ s.t.
	$f \circ \phi_X = g \circ \phi_Y$, i.e. the following diagram commutes
	\begin{equation}
	\begin{tikzcd}[column sep=small]
		Z \arrow[rd, "\exists\, !\, \phi", dashrightarrow] \arrow[rrd, "\phi_X", rightarrow, bend left] 
		\arrow[rdd, "\phi_Y"', rightarrow, bend right] & & \\
		&
		X \cross_S Y \arrow[r, "\pi_X", rightarrow] \arrow[d, "\pi_Y"', rightarrow] &
		X \arrow[d, "f", rightarrow] \\
		&
		Y \arrow[r, "g"', rightarrow] &
		S\\
	\end{tikzcd}
	\end{equation} 
	there exists a unique morphism $\phi: Z \to X \cross_S Y$ s.t. the whole diagram commutes,
	i.e. $\phi_X = \pi_X \circ \phi$ and $\phi_Y = \pi_Y \circ \phi$.
\end{defn}

\begin{rem}[]
	As with the usaul argument with universal properties, if $X \cross_S Y$ exists,
	then it is unique up to isomorphism.
	
	Moreover one can construct $X \cross_S Y$ by a gluing construction.
	The key point is that, called $X := \mathrm{Spec}\, M$, $Y := \mathrm{Spec}\, N$
	and $S := \mathrm{Spec}\, R$
	(in which $M$ and $N$ are $R$-algebras, since both $X$ and $Y$ are schemes over $S$)
	then
	\begin{equation}
		\mathrm{Spec} \left( M \otimes_R N \right) = X \cross_S Y
	.\end{equation} 
\end{rem}

\section{Projective schemes}
\begin{defn}[Irrelevant ideal of a graded ring]
	Let $R := \bigoplus_{d \geq 0} R_d$ be a graded ring.
	The irrelevant ideal of $R$ is
	\begin{equation}
	R_+ := \bigoplus_{d > 0} R_d
	.\end{equation} 
\end{defn}

\begin{defn}[Projective scheme]
	Let $R$ be a graded ring.
	The projective scheme associated with $R$ is
	\begin{equation}
	\mathrm{Proj}\, R := \left\{ \mathfrak{p} \triangleleft R \ \middle|\ 
	\mathfrak{p} \text{ is a homogeneous prime ideal, and } R_+ \not\subset \mathfrak{p} \right\}
	,\end{equation} 
	with the topology defined by the closed subsets
	\begin{equation}
	\mathbb{V}\left( \mathcal{I} \right) =
	\left\{ \mathfrak{p} \in \mathrm{Proj}\, R \ \middle|\ \mathfrak{p} \supset \mathcal{I} \right\}
	\end{equation} 
	for all homogeneous ideals $\mathcal{I} \triangleleft R$.
\end{defn}

\begin{rem}[]
	The Zariski topology on $\mathrm{Proj}\, R$ is well defines, since
	\begin{align}
		\bigcap_{j \in \mathcal{J}} \mathbb{V}\left( \mathcal{I}_j \right) &=
	\mathbb{V}\left( \sum_{j \in \mathcal{J}}^{} \mathcal{I}_j \right) \subset \mathrm{Proj}\, R\\
	\mathbb{V}\left( \mathcal{I}_1 \right) \cup \mathbb{V}\left( \mathcal{I}_2 \right) =
	\mathbb{V}\left( \mathcal{I}_1 \cdot \mathcal{I}_2 \right) \subset \mathrm{Proj}\, R
	\end{align} 
	also hold in this new setting.
\end{rem}

\begin{defn}[Structure sheaf]
	Let $R$ be a graded ring and $X := \mathrm{Proj}\, R$.
	\begin{enumerate}
		\item For all $\mathfrak{p} \in \mathrm{Proj}\, R$ one sets
			\begin{equation}
				R_{(\mathfrak{p})} :=
				\left\{ \frac{f}{g} \in R_{\mathfrak{p}} \ \middle|\ 
				g \notin \mathfrak{p},\, f,g \in R_d 
			\text{ for some } d \in \N \right\}
			.\end{equation} 
		\item For all open subsets $U \subset X$ we define
			\begin{align*}
				\mathcal{O}_{X} \left( U \right) := \bigg\{ &
			\phi := \left( \phi_{\mathfrak{p}} \right)_{\mathfrak{p} \in U} \in 
		\prod_{\mathfrak{p} \in U} R_{(\mathfrak{p})} \ \bigg|\ 
	\text{for every } \mathfrak{p} \in U \text{ there is an open}\\
	&\text{neighbourhood } V \text{ of } \mathfrak{p} \text{ and } f,g \in R_d \text{ for some } d \in \N\\
	&\text{ s.t. } g \notin Q \text{ and } \phi_Q = \frac{f}{g} \,\forall\, Q \in V \bigg\}
			.\end{align*} 
	\end{enumerate}
	$\mathcal{O}_X$ is called the {\em structure sheaf} of $X = \mathrm{Proj}\, R$.
\end{defn}

\begin{prop}
	Let $R$ be a graded ring and $X := \mathrm{Proj}\, R$.
	Then $\left( O, \mathcal{O}_{ O } \right)$ is a scheme with
	\begin{enumerate}
		\item $\mathcal{O}_{X, \mathfrak{p}} \cong R_{(\mathfrak{p})}$
			for all $\mathfrak{p} \in X$,
		\item for each homogeneous element $f \in R_d$, for $d > 0$,
			we define the distinguished open subset
			\begin{equation}
			X_f := X \setminus \mathbb{V}\left( f \right) =
			\left\{ \mathfrak{p} \in \mathrm{Proj}\, R \ \middle|\ f \notin \mathfrak{p} \right\}
			.\end{equation} 
			The distinguishe open subsets $X_f$ cover $X$ and satisfy
			\begin{equation}
			\left( X_f, \left.\mathcal{O}_{ X }\right|_{X_f} \right) \cong
				\mathrm{Spec}\, R_{(f)}
			,\end{equation} 
			(i.e. distinguished open subsets are affine schemes),
			with
			\begin{equation}
				R_{(f)} := \left\{ 
				\frac{g}{f^r} \in R_f \ \middle|\ g \in R_{rd},\, d = \deg f \right\}
			.\end{equation} 
	\end{enumerate}
\end{prop}
\begin{proof}
	The proof is analogous to the affine case, apart
	from the fact that $X_f$ cover $X$.
\end{proof}


\begin{ex}
	Consider $\left( \mathbb{P}^{n}_{\K} \right)_{\text{sch}} := \mathrm{Proj}\, \mathbb{K}\left[x_0, \ldots, x_n \right]$.
	If $X \subset \mathbb{P}^{n}_{\K}$ is a projective variety, then
	\begin{equation}
		X_{\text{sch}} = \mathrm{Proj} \left( \mathbb{K}\left[x_0, \ldots, x_n \right]/
		\mathbb{I}\left( X \right) \right)
	,\end{equation} 
	where $S(X) := \mathbb{K}\left[x_0, \ldots, x_n \right] / \mathbb{I}\left( X \right)$ is
	the homogeneous coordinate ring of $X$.
\end{ex} 
 \begin{defn}[Projective subscheme]
	 Let $\K$ be an algebraically closed field.
	 A projective subscheme of $\mathbb{P}^{n}_{\K}$ is a scheme of the form
	 \begin{equation}
		 \mathrm{Proj} \left( \mathbb{K}\left[x_0, \ldots, x_n \right]/ \mathcal{I} \right)
	 \end{equation} 
	 for $\mathcal{I} \triangleleft \mathbb{K}\left[x_0, \ldots, x_n \right]$
	 a homogeneous ideal.
 \end{defn}

 \begin{rem}[]
 	Projective subschemes can be reducible, e.g.
	\begin{equation}
		\mathrm{Proj}\, \mathbb{K}\left[x_0, x_1 \right]/ (x_1^2)
	\end{equation} 
	or even non-reduced, e.g.
	\begin{equation}
		\mathrm{Proj}\, \mathbb{K}\left[x_0, x_1, x_2 \right] / (x_1x_2)
	.\end{equation} 
 \end{rem}
 
 \begin{rem}[]
 	Different homogeneous ideals may define the same projective subscheme in $\mathbb{P}^{n}$.
	For example
	\begin{equation}
		\mathrm{Proj}\, \mathbb{K}\left[x_0, \ldots, x_n \right] / (f) =
		\mathrm{Proj}\, \mathbb{K}\left[x_0, \ldots, x_n \right] /
		(fx_0, fx_1, \ldots, fx_n)
	.\end{equation} 
 \end{rem}
 
 \begin{defn}[Saturation of an ideal]
 	Let $\mathcal{I} \triangleleft R := \mathbb{K}\left[x_0, \ldots, x_n \right]$
	be a homogeneous ideal.
	The saturation of $\mathcal{I}$ is the homogeneous ideal
	\begin{equation}
	\overline{\mathcal{I}} := \left\{ f \in R \ \middle|\ 
	x_0^mf, \ldots, x_n^mf \in \mathcal{I} \text{ for some } m \in \mathcal{I} \right\}
	.\end{equation} 
	A homogeneous ideal $\mathcal{I}$ is called saturated iff $\mathcal{I} = \overline{\mathcal{I}}$.
 \end{defn}
 
 \begin{lem}
 	\begin{equation}
 	\mathrm{Proj}\, R/I = \mathrm{Proj}\, R/J \iff
	\overline{I} = \overline{J}
 	.\end{equation} 
 \end{lem} 

 \begin{thm}[]
 	There is a one to one correspondence
	\begin{equation}
	\begin{tikzcd}[row sep=tiny]
		\left\{ 
		\begin{matrix}
			\text{projective}\\
			\text{subschemes}\\
			\text{of } \mathbb{P}^{n}_{\K}
		\end{matrix} 
		\right\} \arrow[r, "1:1", leftrightarrow] &
		\left\{ 
		\begin{matrix}
			\text{saturated}\\
			\text{homogeneous ideals}\\
			\mathcal{I} \triangleleft \mathbb{K}\left[x_0, \ldots, x_n \right]
		\end{matrix} 
		\right\}\\
		X \arrow[r, "", mapsto] &
		\mathbb{I}\left( X \right) = \overline{\mathcal{I}}
	\end{tikzcd}
	,\end{equation} 
	where $\mathcal{I}$ is any ideal defining $X$.
 \end{thm}

 \begin{defn}[Homogeneous coordinate ring of a projective subscheme]
 	Let $X$ be a projective subscheme in $\mathbb{P}^{n}_{\K}$.
	The {\em homogeneous coordinate ring} of $X$ is
	\begin{equation}
		S(X) := \mathbb{K}\left[x_0, \ldots, x_n \right] / \mathbb{I}\left( X \right)
	.\end{equation} 
 \end{defn}
 


\end{document}
