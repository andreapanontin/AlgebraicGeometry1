\section{Main theorem on projective varieties}

\begin{rem}
	Let's recall a few notions:
	\begin{itemize}
		\item $\mathbb{P}^{n}(\mathbb{C})$ is compact in the standard topology.
			This means that every projective variety is compact in the analytic topology.
		\item The image of a compact subset under a continuous map is compact.
	\end{itemize}
	Moreover, coming back to concepts concerning algebraic geometry:
	the image of an affine variety under a morphism may not be closed.
	An example is $\mathbb{A}^{2} \xrightarrow{\pi} \mathbb{A}^{1}$, then
	\begin{equation}
		\pi \left( \mathbb{V}\left( xy - 1 \right) \right) = \mathbb{A}^{1} \setminus \left\{ \mathbf{0} \right\}
	.\end{equation} 
	The first clearly is closed, whereas the latter is open.
\end{rem}

For the Zariski topology on projective varieties we will, instead, prove that the image of a projective variety under a morphism is always closed.
In other words the image of a projective variety under a morphism is always a projective variety.
(That's the reason why we needed to extend the definition of affine varieties to abstract varieties. In the case of projective ones we do not need to do so.)

\begin{thm}[Projection map is closed]
	Let $X \subset \mathbb{P}^{n} \cross \mathbb{P}^{m}$ be a closed subset.
	Let $\pi: \mathbb{P}^{n} \cross \mathbb{P}^{m} \to \mathbb{P}^{n}$ be the canonical projection.
	Then $\pi(X) \subset \mathbb{P}^{n}$ is closed.
	In other words the projection map is closed.
\end{thm}

\begin{rem}
	This theorem has application in computational algebra.
	It is sometimes called the \textit{main theorem of elimination theory}.
	In elimination theory, starting from a variety (i.e. the zero locus of a finite collection of bihomogeneous polynomials)
	\begin{equation}
		X: F_1(x,y) = F_2(x,y) = \ldots = F_r(x,y) = 0
	.\end{equation} 
	Then the projection sens $X$ to
	\begin{equation}
		\pi(X) = \left\{ x \in \mathbb{P}^{n} \ \middle|\ \exists\, y \in \mathbb{P}^{m} \text{ s.t. } F_i(x,y) = 0 \,\forall\, i \right\}
	.\end{equation} 
	We are trying to remove the $y$ variables from the first system.
	Since $\pi(X)$ is closed, then it is the solution set of some polynomial equations.
\end{rem}

\begin{cor}
	The projection morphism $\pi: \mathbb{P}^{n} \cross Y \to Y$ is closed for all varieties $Y$.
\end{cor} 

\begin{defn}[Complete variety]
	A variety $X$ is \textbf{complete} iff the projection $\pi: X \cross Y \to Y$ is closed for any variety $Y$.
\end{defn}

\begin{ex}\leavevmode\vspace{-.2\baselineskip}
	\begin{itemize}
		\item $\mathbb{P}^{n}$ is complete, by the above corollary.
		\item Any projective variety $X \subset \mathbb{P}^{n}$ is complete:
			Any closed subset of $X \cross Y$ is also closed in $\mathbb{P}^{n} \cross Y$.
			Since the image by the projection is the same, then also its
			projection to $Y$ is closed.
		\item There also exist complete varieties that are not projective, but it is hard to prove this last statement.
	\end{itemize}
\end{ex} 

\begin{cor}
	Let $f: X \to Y$ be a morphism of varieties.
	Let $X$ be a complete variety, then
	$f(X)$ is closed in $Y$.
\end{cor} 
\begin{proof}
	The idea is to study an arbitrary map as a projection.
\end{proof}

Let's now see a few corollaries to the above theorem.

\begin{cor}
	Let $X \subset \mathbb{P}^{n}$ be a projective variety that contains more than one point (equivalently $\mathrm{dim}\, X \geq 1$).
	Then, for any non constant homogeneous polynomial $f \in \mathbb{K}\left[x_0, \ldots, x_n \right]$, $X \cap \mathbb{V}_p\left( f \right) \neq \emptyset$.
\end{cor} 

\begin{cor}
	Any two curves in $\mathbb{P}^{2}$ intersect.
	In particular $\mathbb{P}^{2}$ is not isomorphic to $\mathbb{P}^{1} \cross \mathbb{P}^{1}$ 
	(recall that $\mathbb{P}^{1} \cross \mathbb{P}^{1}$ is covered by a family of lines that do not intersect each other).
\end{cor}

\begin{cor}
	Every regular fucntion on a complete variety is constant, i.e. for any $X$ complete
	\begin{equation}
	\mathcal{O}_{X} \left( X \right) = \K
	.\end{equation} 
\end{cor} 

\subsection{Veronese variety}

Let $n, d \geq 1$.
Consider $\mathbb{K}\left[x_0, \ldots, x_n \right]_d$, a $\K$-Vector Space of dimension $\binom{n+d}{n}$.
Let $F_0, \ldots, F_\nu$ be a basis for this space, hence
\begin{equation}
	\nu = \nu_{n,d} := \binom{n+d}{n} - 1
.\end{equation} 
One could choose the monomials of degree $d$ (ordered in some way, for example lexicographically), as a basis.
For instance
\begin{equation}
F_0 = x_0^d, \qquad
F_1 = x_0^{d-1} x_1,\qquad \ldots,\qquad
F_{\nu} = x_n^d
.\end{equation} 

The basis induces a morphism of varieties
\begin{align}
	v_d: \mathbb{P}^{n} &\to \mathbb{P}^{\nu} \\
	\left[ x_0 , \ldots , x_n \right] &\mapsto \left[ F_0(x) , \ldots , F_\nu(x) \right]
.\end{align} 
By the main theorem on projective varieties, the image of this morphism, $V_{n,d} := v_d \left( \mathbb{P}^{n} \right)$, is a projective variety.
We call it the Veronese variety of degree $d$.

\begin{prop}
	The morphism $v_d: \mathbb{P}^{n} \to V_{n,d}$ is an isomorphism.
\end{prop} 
\begin{rem}
	When we view $\mathbb{P}^{n}$ as $V_{n,d} \subset \mathbb{P}^{\nu}$, then the degree $d$ 
	homogeneous polynomials in $x_0, \ldots, x_n$ give \textbf{linear} forms in the coordinates of $\mathbb{P}^{\nu}$.
\end{rem} 

\begin{ex}[$n = 1$: the rational normal curve of degree $d$]
	$V_{1,d} \subset \mathbb{P}^{d}$ is called a \textit{rational normal curve of degree} $d$ in $\mathbb{P}^{d}$.
	It is given by
	\begin{align}
		v_d: \mathbb{P}^{1} &\to \mathbb{P}^{d} \\
		\left[ x_0 , x_1 \right] &\mapsto \left[ x_0^d, x_0^{d-1}x_1 , \ldots , x_1^d \right] = 
		\left[ u_0, u_1 , \ldots , u_d \right]
	.\end{align} 
	As a subvariety of $\mathbb{P}^{d}$, $V_{1,d}$ is defined by the condition
	\begin{equation}
	\mathrm{rk}\, 
	\begin{pmatrix}
		u_0 & u_1 & \ldots & u_{d-1}\\
		u_1 & u_2 & \ldots & u_d
	\end{pmatrix} 
	= 1
	.\end{equation} 
\end{ex} 

\begin{cor}
	Let $X \subset \mathbb{P}^{n}$ be a projective variety and $f \in \mathbb{K}\left[x_0, \ldots, x_n \right]$ any non constant homogeneous polynomial.
	Then $X \setminus \mathbb{V}_p\left( f \right)$ is an affine variety.
\end{cor} 

\section{Dimension}
Recall that the dimension of a \textit{Noetherian} topological space $X$ is the largest
index $d$ for which there is exists a chain
\begin{equation}
\emptyset \subsetneq X_0 \subsetneq X_1 \subsetneq \ldots \subsetneq X_d = X
\end{equation} 
of irreducible closed subsets $X_i \subset X$.
We will call such chain a \textit{longest chain} in $X$.

\begin{rem}
	What do we know about dimension as of now?
	\begin{itemize}
		\item $\dim \left\{ pt \right\} = 0$.
		\item $\dim \mathbb{A}^{1} = \dim \mathbb{P}^{1} = 1$, since the only irreducible closed subsets are the points.
		\item \textit{Bezout's theorem}: two curves $\mathbb{V}\left( f \right), \mathbb{V}\left( g \right) \subset \mathbb{A}^{2}$ with no common irreducible component, 
			intersect in $\left( \deg f \right) \left( \deg g \right)$ points (counted with multiplicity).
			This means that the irreducible closed subsets of $\mathbb{A}^{2}$ are:
			\begin{itemize}
				\item $\mathbb{A}^{2}$,
				\item irreducible curves $X = \mathbb{V}\left( f \right)$, for $f \in \mathbb{K}\left[x, y \right]$ irreducible,
				\item points.
			\end{itemize}
			Hence longest chains are of the form
			\begin{equation}
			\emptyset \subsetneq \left\{ pt \right\} \subsetneq \mathbb{V}\left( f \right) \subsetneq \mathbb{A}^{2}
			,\end{equation} 
			with $f(p) = 0$ and $f$ is irreducible.
			This implies that $\dim \mathbb{A}^{2} = 2$.
			Since, moreover, $\mathbb{P}^{2} = \mathbb{A}^{2} \cup \mathbb{V}_p\left( x_0 \right)$, we obtain that $\dim \mathbb{P}^{2} = 2$.
		\item We also noticed that $\dim \mathbb{A}^{n} \geq n$, in fact we have the \textit{longest chain}
			\begin{equation}
			\emptyset \subsetneq \mathbb{V}\left( x_1, \ldots, x_n \right) \subsetneq
			\mathbb{V}\left( x_2, \ldots, x_n \right) \subsetneq \ldots \subsetneq
			\mathbb{V}\left( x_n \right) \subsetneq \mathbb{A}^{n}
			.\end{equation} 
			Analogously we can show that $\dim \mathbb{P}^{n} \geq n$.
	\end{itemize}
\end{rem}

\begin{lem}\leavevmode\vspace{-.2\baselineskip}
	\begin{enumerate}
		\item If $X$ is a Noehterian topological space and we consider a longest chain
			\begin{equation}
			\emptyset \subsetneq X_0 \subsetneq X_1
			\subsetneq \ldots \subsetneq X_n = X
			,\end{equation} 
			then $\dim X_j = j$ for all $j = 0, \ldots, n$.
		\item If $X$ is irreducible and $Y \subsetneq X$ is closed, then $\dim Y < \dim X$.
	\end{enumerate}
\end{lem} 

A comment on what's to come:
\begin{itemize}
	\item We'll prove that dimension is a local property, i.e. that given a variety $X$ and an affine open subset $U \subset X$, 
		then $\dim U = \dim X$.
		Moreover, for any $U \subset \mathbb{A}^{n}$, then $\dim Y = \dim U$, for $Y:= \overline{U} \subset \mathbb{P}^{n}$.
		This means that we can concentrate our study only on projective varieties.
	\item Given a surjective morphism $f: X \to Y$ of projective varieties, then $\dim Y \leq \dim X$.
		From this idea we get the expectation that, given $X$ a projective variety, then $\dim X$ is the largest integer $d$ 
		s.t. there is a surjective morphism $f: X \to \mathbb{P}^{d}$.
\end{itemize}

\begin{lem}
	Let $f: X \to Y$ be a surjective morphism of projective varieties.
	Then every longest chain
	\begin{equation}
	\emptyset \subsetneq Y_0 \subsetneq Y_1 \subsetneq
	\ldots \subsetneq Y_n = Y
	\end{equation} 
	can be lifted to a chain
	\begin{equation}
	\emptyset \subsetneq X_0 \subsetneq X_1 \subsetneq \ldots
	\subsetneq X_n = X
	\end{equation} 
	of irreducible closed subsets $X_j \subset X$, s.t. $f(X_j) = Y_j$ for all $j = 0, \ldots, n$.
	In particular $\dim X \geq \dim Y$.
\end{lem} 

\begin{rem}[Next step]
	Our next aim is to show that every projective variety $X \subset \mathbb{P}^{n}$ admits a surjective morphism to $\mathbb{P}^{m}$, for some $m \leq n$.
	It will be constructed by taking a sequence of projections.
\end{rem}
\begin{proof}[Construction]
	Let $X \subsetneq \mathbb{P}^{n}$ a projective variety and $p \in \mathbb{P}^{n}$ s.t. $p \not\in X$.
	Let $H \subset \mathbb{P}^{n}$ be a linear subspace of codim $1$ no passing through $p$,
	i.e. $H$ is the zero locus of a degree $1$ homogeneous polynomial.
	For instance $H = \mathbb{V}_p\left( a_0 x_0 + \ldots + a_n x_n \right)$, with $a_i \neq 0$ for some $i$.
	WLOG $a_n \neq 0$.
	Then $H$ is isomorphic to $\mathbb{P}^{n-1}$, via
	\begin{align}
		\sigma: H = \mathbb{V}_p\left( a_0 x_0 + \ldots + a_n x_n \right) &\xrightarrow{\cong} \mathbb{P}^{n-1} \\
		\left[ x_0 , \ldots , x_n \right] &\mapsto \left[ x_0 , \ldots , x_{n-1} \right]
	,\end{align} 
	where $x_n = -\frac{a_0}{a_n} x_0 - \ldots - \frac{a_{n-1}}{a_n}$.
	We then define the projection $\pi: X \to \mathbb{P}^{n-1}$ as the map sending $q \in X$
	to the point of $H \cong \mathbb{P}^{n-1}$ corresponding to the intersection of $H$ with the line $pq$.
	Up to a linear change of coordinates we may assume $p = \left[ 0 , \ldots , 0, 1 \right]$,
	and $H = \mathbb{V}_p\left( x_n \right)$.
	Then
	\begin{align}
		\pi: X &\to \mathbb{P}^{n-1} \\
		\left[ x_0 , \ldots , x_n \right] &\mapsto \left[ x_0 , \ldots , x_{n-1} \right]
	.\end{align} 
	Let's now construct this change of coordinates.
	Recall that a linear change of coordinates $A \in \mathrm{PGL}(n+1)$ is determined by the image
	of $n+2$ points in general position (which means that no $n+1$ of them are contained in a hyperplane).
	In particular we choose $n$ points, $q_0, \ldots, q_{n-1}$ in general position s.t. $\mathrm{span}(q_0, \ldots, q_n) = H$,
	$p \not\in H$ and $r$ any point on $pq_{n-1}$ different both from $p$ and $q_{n-1}$, then $A$ is defined as follows
	\begin{equation}
	\begin{matrix}
		q_0& \mapsto &\left[ 1, 0 , \ldots , 0 \right]\\
		   &\vdots&\\
		q_{n-1}& \mapsto &\left[ 0 , \ldots , 1, 0 \right]\\
		p& \mapsto &\left[ 0 , \ldots , 0, 1 \right]\\
		r& \mapsto &\left[ 1, 1 , \ldots , 1, 1 \right]
	\end{matrix} 
	.\end{equation} 
\end{proof}

\begin{rem}[]\leavevmode\vspace{-.2\baselineskip}
	\begin{itemize}
		\item The map $\pi: X \to \mathbb{P}^{n-1}$ is a morphism, since $x_0 = \ldots = x_{n-1} = 0$
			has no common solution on $X$ ($p \not\in X$).
		\item All fibers of $\pi$ are finite sets:
			for all $a \in H$, $\pi^{-1}\left( \left\{ a \right\} \right) = pa \cap X$ is a closed subset of
			$pa$ (a line), different from the whole $pa$, since $p \not\in X$.
			The closed subsets of a line are finite sets by the classification of closed sybsets of $\mathbb{P}^{1}$.
		\item We can iterate
			\begin{equation}
			X \xrightarrow{\pi_{p_1}} \mathbb{P}^{n-1} \xrightarrow{\pi_{p_2}} 
			\ldots \xrightarrow{\pi_{p_{n-m}}} \mathbb{P}^{m}
			,\end{equation} 
			so long as $\exists\, p_i \not\in \mathbb{P}^{n-i}$, until
			$f = \pi_{p_{n-m}} \circ \ldots \circ\pi_{p_2} \circ \pi_{p_1}$ is surjective.
			We can interpret geometrically $f$ as a projection of $X$ onto a linear subspace
			$H' \cong \mathbb{P}^{m}$, with center a linear subspace of dimension $n-m-1$,
			spanned by $p_1, \ldots, p_{n-m}$
	\end{itemize}
\end{rem}

\begin{lem}\label{lem:Lem3_Dim}
	Let $X \subsetneq \mathbb{P}^{n}$ be a projective variety not containing $p = \left[ 0 , \ldots , 0, 1 \right]$.
	Then, for every $f \in S(X) = \mathbb{K}\left[x_0, \ldots, x_n \right]/\mathbb{I}_p(X)$
	there exists a degree $D \geq 1$, and homogeneous polynomials
	$a_0, \ldots, a_{D-1} \in \mathbb{K}\left[x_0, \ldots, x_{n-1} \right]$ s.t.
	\begin{equation}\label{eqn:Lem3_Dim}
		f^D + a_{D-1} f^{D-1} + \ldots + a_1 f + a_0 = 0 \in S(X)
	.\end{equation} 
\end{lem} 
\begin{rem}[]
	If we consider $f = \bar{x}_n$, we obtain again that the fibers of $\pi$ are finite.
	However it is a stronger condition: let's restrict to an affine open subset $p \not \in U \cong \mathbb{A}^{n}$ of $\mathbb{P}^{n}$.
	(Let, for simplicity, $U = U_0$), then we have the projection
	\begin{align}
		\left.\pi\right|_{Y}: X \cap U_0 \subset \mathbb{A}^{n} &\to \mathbb{A}^{n-1} \\
		\left( t_1, \ldots, t_n \right) &\mapsto \left( t_1, \ldots, t_{n-1} \right)
	.\end{align} 
	Moreover the formula \eqref{eqn:Lem3_Dim} in lemma \ref{lem:Lem3_Dim}
	says that $\left( \left.\pi\right|_{Y} \right)^*$ realizes $\mathbb{K}[Y]$
	as an integral ring extension of $\mathbb{K}\left[t_1, \ldots, t_{n-1} \right] = \mathbb{K}[\mathbb{A}^{n-1}]$.
	Morphisms satisfying this property (on each affine open set) are called \textbf{finite morphisms}.
\end{rem}

\begin{lem}
	Let $\pi: X \to \mathbb{P}^{n-1}$ be the projection of $X \subset \mathbb{P}^{n}$ with center $p \notin X$.
	If $Y \subset X$ is a closed subset and $\pi(X) = \pi(Y)$, then $Y = X$.
\end{lem} 
\begin{rem}[]
	Equivalently, for a closed $Y \subsetneq X$, then $\pi(Y) \subsetneq \pi(X)$.
\end{rem}

\begin{cor}
	Let $\pi: X \to \mathbb{P}^{n-1}$ be the projections from $p \notin X$.
	Then $\dim X = \dim \pi(X)$.
\end{cor} 
\begin{cor}
	$\dim \mathbb{P}^{n} = n$.
\end{cor} 

\subsection{Dimension of the intersection with a hypersurface}
\subsubsection{Projective case}
\begin{prop}
	Let $X \subset \mathbb{P}^{n}$ be a projective variety, and $f \in \mathbb{K}\left[x_0, \ldots, x_n \right]$
	be a non constant homogeneous polynomial, s.t. $f \notin \mathbb{I}_p(X)$.
	Then
	 \begin{equation}
		 \dim \left( X \cap \mathbb{V}_p\left( f \right) \right) = \dim X - 1
	.\end{equation} 
\end{prop} 
\begin{rem}[]
	If the intersection $X \cap \mathbb{V}_p\left( f \right)$ is reducible,
	we don't know if all the components have maximal dimension ($\dim X - 1$).
	However we will later prove that all components have the same dimension,
	i.e. $X \cap \mathbb{V}_p\left( f \right)$ is of pure dimension $\dim X - 1$.
\end{rem}

\begin{prop}
	Let $X$ be a variety and $\emptyset \neq U \subset X$ be an open subset.
	Then $\dim X = \dim U$.
\end{prop}  

\begin{rem}[Some consequences]\leavevmode\vspace{-.2\baselineskip}
	\begin{enumerate}
		\item $\mathbb{A}^{n} \subset \mathbb{P}^{n}$ is open, hence $\dim \mathbb{A}^{n} = \dim \mathbb{P}^{n}$.
		\item $\mathbb{A}^{n+m} = \mathbb{A}^{n} \cross \mathbb{A}^{m} \stackrel{\text{open}}{\subset} 
			\mathbb{P}^{n} \cross \mathbb{P}^{m}$, hence
			$\dim \left( \mathbb{P}^{n} \cross \mathbb{P}^{m} \right) = n + m$.
		\item Given $f \in \mathbb{K}\left[x_1, \ldots, x_n \right]$ a non-constant polynomial,
			then $\mathbb{V}\left( f \right) \subset \mathbb{A}^{n}$ has dimension $n-1$.
			This follows from $X \subset \overline{X} = \mathbb{V}_p\left( {}^hf \right) \subset \mathbb{P}^{n}$.
		\item Moreover every irreducible component of $\mathbb{V}\left( f \right)$,
			with $f$ the above polynomial, has dimension $n-1$.
			This follows from the factorization in prime components
			(possible, since $\mathbb{K}\left[x_1, \ldots, x_n \right]$ is a UFD)
			\begin{equation}
			f = f_1 \cdot \ldots \cdot f_r
			,\end{equation} 
			with $f_1, \ldots, f_r$ prime.
			Then the irreducible components of $\mathbb{V}\left( f \right)$ are exactly $\mathbb{V}\left( f_i \right)$,
			and, by the above remark, they are exactly of dimension $n-1$.
		\item If $f \in \mathbb{K}\left[x_0, \ldots, x_n \right]$ is a non-constant polynomial 
			and $X := \mathbb{V}_p\left( f \right) \subset \mathbb{P}^{n}$,
			then each irreducible component of $X$ has dimension $n-1$.

			This is true, since we can always find a hyperplane 
			$H \not\subset X ( \subsetneq \mathbb{P}^{n})$.
			Then $\mathbb{A}^{n}\supset X \cap \left( \mathbb{P}^{n} \setminus H \right) \simeq
			\mathbb{V}\left( g \right)$, for some $g \in \mathbb{K}\left[x_1, \ldots, x_n \right]$ non-constant.
			Then the irreducible components of $X$ are the projective closures of the irreducible components
			of $\mathbb{V}\left( g \right)$, which have dimension $n-1$ by the above remark.

			In particular, given a factorization
			\begin{equation}
			f = f_1 \cdot \ldots \cdot f_r
			\end{equation} 
			for $f \in \mathbb{K}\left[x_0, \ldots, x_n \right]$, in prime components
			(which can be obtained homogenizing the prime components of $g$).
			Then
			\begin{equation}
			X = \mathbb{V}_p\left( f \right) =
			\underbrace{\mathbb{V}_p\left( f_1 \right) \cup \ldots \cup \mathbb{V}_p\left( f_r \right)}_{\text{irreducible components of } X}
			.\end{equation} 
	\end{enumerate}
\end{rem}

\begin{defn}[Hypersurface]
	A \textbf{hypersurface} in $\mathbb{A}^{n}$, resp. $\mathbb{P}^{n}$, is a closed subset of dimension $n-1$.
	The minimal degree of a generator of its ideal is called the \textit{degree} of the hypersurface $X$.

	Hypersurfaces of degree $1$ are called \textit{hyperplanes}.
\end{defn}

\begin{rem}[]
	Up to scaling, we have the following correspondances
	\begin{equation}
	\begin{tikzcd}
		\left\{ 
		\begin{matrix}
		\text{irreducible degree } d\\
		\text{hypersurfaces in } \mathbb{A}^{n}
		\end{matrix} 
		\right\} \arrow[r, "", leftrightarrow] &
		\left\{ 
		\begin{matrix}
		\text{irreducible degree } d\\
		\text{polynomials }
		\end{matrix} 
		\right\}_{/ \K^*}
		\subset \mathbb{P} \left( \mathbb{K}\left[x_1, \ldots, x_n \right]_{\leq d} \right)
	\end{tikzcd}
	.\end{equation} 
	\begin{equation}
	\begin{tikzcd}
		\left\{ 
		\begin{matrix}
		\text{irreducible degree } d\\
		\text{hypersurfaces in } \mathbb{P}^{n}
		\end{matrix} 
		\right\} \arrow[r, "", leftrightarrow] &
		\left\{ 
		\begin{matrix}
		\text{irred. homogoneneous}\\
		\text{polynomials of deg. } d 
		\end{matrix} 
		\right\}_{/ \K^*}
		\subset \mathbb{P} \left( \mathbb{K}\left[x_0, \ldots, x_n \right]_{d} \right)
	\end{tikzcd}
	.\end{equation} 
	One can actually prove that the locus of irreducible polynomials is a dense open subset
	in the projective space of all homogenous polynomials of a fixed degree.
	This is indeed a consequence of Main theorem on projective varieties.
\end{rem}

\begin{defn}[Codimension]
	Let $Y \subset X$ be a closed subset.
	We define the \textbf{codimension} of $Y$ in $X$ as
	\begin{equation}
	\mathrm{codim}_X\, Y := \dim X - \dim Y
	.\end{equation} 
\end{defn}

\subsubsection{Affine case}
\begin{rem}[]
	The projective case does not imply the affine one.
	In fact, let $X := \mathbb{V}\left( x_2 - x_1^2 \right) \subset \mathbb{A}^{2}$ and $f = x_1$.
	Let $\overline{X} = \mathbb{V}_p\left( x_0x_2 - x_1^2 \right) \subset \mathbb{P}^{2}$.
	Then
	\begin{equation}
		\dim \left( \overline{X} \cap \mathbb{V}_p\left( x_1 \right) \right) =
		\dim \overline{X} - 1 = 0
	.\end{equation} 
	More explicitly 
	\begin{equation}
	\overline{X} \cap \mathbb{V}_p\left( x_1 \right) = 
	\big\{ \left[ 1, 0, 0 \right], 
	\underbrace{\left[ 0 , 0, 1 \right]}_{\begin{matrix}{\scriptstyle \text{contained in}}\\
	{\scriptstyle H = \mathbb{V}_p\left( x_0 \right)}\end{matrix}} \big\}
	.\end{equation} 
	Moreover factorizing $f$ does not help to construct all irreducible components.
\end{rem}

\begin{prop}
	Let $X \subset \mathbb{A}^{n}$ be an affine variety, and
	$f \in \mathbb{K}\left[x_1, \ldots, x_n \right]$ be a polynomial not vanishing identically on $X$.
	Then, if $X \cap \mathbb{V}\left( f \right) \neq \emptyset$, we have
	$\dim \left( X \cap \mathbb{V}\left( f \right) \right) = \dim X - 1$.
\end{prop} 

\begin{cor}
	If $X \subset \mathbb{A}^{n}$ is an affine variety, and
	$f \notin \mathbb{I}(X)$ is a non constant polynomial, then
	all irreducible components of $X \cap \mathbb{V}\left( f \right)$ have dimension equal to
	$\dim X - 1$.
\end{cor} 

\begin{thm}[Dimension of a fiber (weak version)]
	Let $f: X \to Y$ be a morphism of varieties and assume there is a nonempty
	open subset $U \subset Y$ s.t. $f^{-1}(p) \neq \emptyset$ for all $p \in U$
	and $\dim f^{-1}(p) = n$ for all $p \in U$, then
	\begin{equation}
	\dim X = \dim Y + n
	.\end{equation} 
\end{thm}

\begin{thm}[Dimension of a fiber (complete version)]
	Let $f: X \to Y$ be a surjective morphism of varieties, then
	$\dim X \geq \dim Y$ and
	\begin{enumerate}
		\item for every $y \in Y$ and every component $F$ of $f^{-1}(y)$,
			we have $\dim F \geq \dim X - \dim Y$,
		\item there exists a non-empty open subset $U \subset Y$
			s.t. $\dim f^{-1}(y) = \dim X - \dim Y$ for all $y \in U$.
	\end{enumerate}
\end{thm}

\begin{cor}
	Let $X \subset \mathbb{P}^{n}$ be Zariski closed and
	$f: X \to Y$ be a morphism to a variety $Y$ (in the sense that
	$f$ is the restriction of a morphism defined on some larger subvariety of $\mathbb{P}^{n}$).
	If all fibers $f^{-1}(y)$, for $y \in Y$, are irreducible and of the same dimension, 
	then $X$ is irreducible.
\end{cor} 

\begin{cor}
	For any two varieties $X$ and $Y$, we have
	\begin{equation}
	\dim X \cross Y = \dim X + \dim Y
	.\end{equation} 
	[Consider $\pi_Y: X \cross Y \to Y$, with fibers isomorphic to $X$].
\end{cor} 

\begin{cor}
	Let $X, Y \subset \mathbb{A}^{n}$ be affine varieties s.t. $X \cap Y \neq \emptyset$,
	then every irreducible component $Z$ of $X \cap Y$ has dimension
	\begin{equation}
	\dim Z \geq \dim X + \dim Y - n
	.\end{equation} 
\end{cor} 

\begin{cor}
	Let $X, Y \subset \mathbb{P}^{n}$ be projective varieties.
	Then every irreducible component of $X \cap Y$ has dimension 
	$\geq \dim X + \dim Y - n$ and, if $\dim X + \dim Y \geq n$, then
	$X \cap Y \neq 0$.
\end{cor} 

\subsection{Dimension - again}
\begin{rem}[]
	Let $X$ be a variety.
	Denote by $\mathbb{K}(X)$ the function field of $X$, i.e. the field
	of rational functions on $X$.

	If $X \subset \mathbb{A}^{n}$ is affine, then $\mathbb{K}(X) = Q(\mathbb{K}[X])$,
	e.g.
	\begin{equation}
		\mathbb{K}(\mathbb{A}^{n}) = \mathbb{K}(x_1, \ldots, x_n) \qquad \text{ the field of rational functions}
	.\end{equation} 
	If $X \subset \mathbb{P}^{n}$ is projective, then
	\begin{equation}
		\mathbb{K}(\mathbb{P}^{n}) = \left\{ 
		\frac{p(x_0, \ldots, x_n)}{q(x_0, \ldots, x_n)} \ \middle|\ p, q \in S(X)_d \text{ and } 
	q \neq 0\right\}
	.\end{equation} 
	If $X$ is an arbitrary variety, by ex. 6, ex-sh 3, we have
	\begin{equation}
		\K(X) = \left\{  \left(U, \phi\right) \ \middle|\ 
		\phi \in \mathcal{O}_{X} \left( U \right) \right\}/\sim
	,\end{equation} 
	where $\left(U, \phi\right) \sim \left(U', \phi'\right)$ iff $\left.\phi\right|_{V} = \left.\phi'\right|_{V}$ for some 
	$\emptyset \subsetneq V \subset U \cap U'$ open.
	The fundamental property is that $\K(X) = \K(U)$ for any (affine) open set $U \neq \emptyset$.
\end{rem}

\begin{defn}[Transcendence degree]
	Given $\K \subset E$ a field extension, we define the \textbf{transcendence degree}, 
	denoted by $\mathrm{tr.deg.}(E/\K)$, is the largest integer $l$ s.t.
	there exists $l$ algebraically independent elements $\zeta_1, \ldots, \zeta_l \in E$.
	In the above algebraically independent means that the valuation morphism
	\begin{align}
		\phi: \mathbb{K}\left[x_1, \ldots, x_l \right] &\to E \\
		F(x_1, \ldots, x_l) &\mapsto F(\zeta_1, \ldots, \zeta_l)
	\end{align} 
	has trivial kernel.
\end{defn}

\begin{thm}[]
	If $X$ is a variety, then $\dim X$ equals the 
	transendence degree of $\K(X)$ over $\K$.
\end{thm}

\begin{rem}[]
	If $X \subset \mathbb{A}^{m}$ is an affine variety, then there is a finite map
	$X \to \mathbb{A}^{n}$, with $n = \dim X$, given by a sequence of
	projections.
	Hence $\K(X)$ is an integral extension of $\mathbb{K}\left[x_1, \ldots, x_n \right]$, then
	$\mathrm{tr.deg.}\, \K(X) = n$.
\end{rem}

