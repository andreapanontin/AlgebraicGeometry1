\section{Morphisms and products of projective varieties}

\begin{defn}[Quasi-projective variety]
	Let $X \subset \mathbb{P}^{n}$.
	$X$ is a \textbf{quasi-projective} variety iff it is a dense open subset of a projective variety.
\end{defn}

\begin{rem}
	All quasi-projective varieties are prevarieties.
\end{rem}

\begin{lem}
	Let $X \subset \mathbb{P}^{n}$ be a quasi-projective variety.
	Let $F_0, \ldots, F_m \in \mathbb{K}\left[x_0, \ldots, x_n \right]_d$ s.t., for every $x \in X$, there is at least one $F_i$ with $F_i(x) \neq 0$.
	Then the $F_i$s define a map
	\begin{align}
		F: X &\to \mathbb{P}^{m} \\
		x &\mapsto \left[ F_0(x) , \ldots , F_m(x) \right]
	\end{align} 
	which is always a morphism of quasi-projective varieties.
\end{lem} 

\begin{rem}
	Not all morphisms $X \to \mathbb{P}^{m}$ can be obtained in this way.
\end{rem}

\begin{ex}
	The map
	\begin{align}
		f: \mathbb{P}^{1} &\to \mathbb{P}^{2} \\
		\left[ s , t \right] &\mapsto \left[ s^2, st, t^2 \right]
	\end{align} 
	is a morphism with image $X = \mathbb{V}_p\left( xz - y^2 \right)$.
	In fact it is an isomorphism on its image.
	This means that its inverse is also a morphism
	\begin{align}
		f^{-1}: X &\to \mathbb{P}^{1} \\
		\left[ x, y, z \right] &\mapsto 
		\begin{cases}
			\left[ x, y \right] & \text{ for } \left( x, y \right) \neq 0 \\
			\left[ y, z \right] & \text{ for } \left( y, z \right) \neq 0 
		\end{cases} 
	.\end{align} 
	However $f^{-1}$ cannot be defined globally by a pair of homogeneous polynomials in $x, y$ and $z$.
\end{ex} 

\begin{rem}
	As a corollary, the example shows that all irreducible conics in $\mathbb{P}^{2}$ are isomorphic to $\mathbb{P}^{1}$.
\end{rem}

In the following we'll denote $N_{n,m} := \left( n + 1 \right) \left( m+1 \right) - 1$.
\begin{defn}[Segre embedding]
	The \textbf{Segre embedding} is the morphism
	\begin{align}
		s_{n,m}: \mathbb{P}^{n} \cross \mathbb{P}^{m} &\to \mathbb{P}^{N_{n,m}} \\
		\left( \left[ x_0 , \ldots , x_n \right], \left[ y_0 , \ldots , y_m \right] \right) &\mapsto 
		\left[ z_{ij} \ \middle|\ 0 \leq i \leq n,\ 0 \leq j \leq m \right]
	,\end{align} 
	with $z_{ij} := x_iy_j$.
\end{defn}

\begin{lem}\leavevmode\vspace{-.2\baselineskip}
	\begin{enumerate}
		\item The image of $s_{n,m}$ is the projective variety $X \subset \mathbb{P}^{n}$ whose
	ideal is generated by
	\begin{equation}
	z_{ij} z_{kl} - z_{il} z_{kj} \text{ for all } 0 \leq i,k \leq n \text{ and } 0 \leq j, l \leq m
	.\end{equation} 
	$X$ is called the \textbf{Segre} variety.
		\item $s_{n,m}: \mathbb{P}^{n} \cross \mathbb{P}^{m} \to X$ is an isomorphism of prevarieties.
			In particular $\mathbb{P}^{n} \cross \mathbb{P}^{m}$ is a projective variety.
		\item The closed subsets of $\mathbb{P}^{n} \cross \mathbb{P}^{m}$ are the zero loci of polynomials in
			$\mathbb{K}\left[x_0, \ldots, x_n, y_0, \ldots, y_m \right]$ that are bihomogeneous in the $x$ and $y$ coordinates.
	\end{enumerate}
\end{lem} 		

\begin{rem}
	The ring $R := \mathbb{K}\left[x_0, \ldots, x_n, y_0, \ldots, y_m \right]$ is a bigraded ring, i.e.
	a graded ring over the monoid $\N \cross \N$.
	\begin{align}
		R &= \bigoplus_{(d,e) \in \N \cross \N} R_{d,e}\\
		R_{d,e} :&= \mathrm{span}_{\K}\, \left\{ x_0^{i_0} \ldots x_n^{i_n} 
		y_0^{j_0} \ldots y_m^{j_m} \ \middle|\ 
		i_0 + \ldots + i_n = d \text{ and } j_0 + \ldots + j_m = e \right\}
	.\end{align} 
	The elements of $R_{d,e}$ are called bihomogeneous polynomials of bidegree $(d,e)$.
	Moreover, as expected, when multiplying polynomials, the bidegrees are additive.
\end{rem}

\begin{ex}
	Let's compute the \textbf{Segre embedding} of $\mathbb{P}^{1} \cross \mathbb{P}^{1}$.
	\begin{equation}
	\mathbb{P}^{1} \cross \mathbb{P}^{1} \simeq \mathbb{V}_p\left( u_0 u_3 = u_1 u_2 \right) \subset \mathbb{P}^{3}
	.\end{equation} 
	Explicitly the embedding is given by
	\begin{equation}
	\left(\left[ x_0, x_1 \right], \left[ y_0, y_1 \right]\right) \mapsto \left[ x_0y_0, x_0y_1, x_1y_0, x_1y_1 \right]
	.\end{equation} 
	Any rank $4$ quadric in $\mathbb{P}^{3}$ is isomorphic to the Segre product $X$, after a linear change of coordinates.
	Hence it is covered by two families of (pairwise disjoint) lines.
	This is very different from $\mathbb{P}^{2}$, in which any two curves (and in particular any two lines) intersect.
	(In $\mathbb{P}^{1} \cross \mathbb{P}^{1}$ lines from the same family either coincide or do not intersect at all).
\end{ex} 

\begin{prop}[Corollary]
	Projective varieties are varieties (i.e. they are \textbf{separated} prevarieties).
\end{prop} 
\begin{rem}
	Since projective varieties are separated, also quasi-projective varieties
	(i.e. dense open subsets of projective varieties) are varieties.
\end{rem}
