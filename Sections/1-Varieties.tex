\section{Affine Varieties}
\begin{defn}[Affine variety]
	We define an \textbf{affine variety} to be the solution set of systems of polynomial equations.
\end{defn}
\begin{rem}
	We will work over a fixed algebraically closed field.
	For example we will work over $\C{}$ or over the algebraic closure of the finite fields $\mathbb{F}_q$, denoted with $\overline{\mathbb{F}}_q$.
\end{rem}

\begin{defn}[Affine space]
	We denote with $\mathbb{A}^n(\mathbb{K}) = \mathbb{A}^n$ the \textbf{affine} $n$-space:
	\begin{equation}
		\mathbb{A}^n(\mathbb{K}) := \left\{ \left(a_1, a_2, \ldots, a_n\right) \in \mathbb{K}^n \right\}
	.\end{equation} 
\end{defn}

\begin{defn}[Polynomial in n indeterminates]
	A \textbf{polynomial} with coefficients in $\mathbb{K}$ in indeterminates $x_1, x_2, \ldots, x_n$ is an expression of the form
	\begin{equation}
		f \left( x_1, x_2, \ldots, x_n \right) = \sum_{I = \left( i_1, \ldots, i_n \right) \in \mathbb{N}^n}^{} c_I\, x_i^{i_1}\cdot \ldots \cdot x_n^{i_n}
	,\end{equation} 
	where $c_I \in \mathbb{K}$ for all $I$ and it is zero for all but finitely many choices of $I$.
\end{defn}

\begin{defn}[Polynomial ring]
	We define the set
	\begin{equation}
		\mathbb{K}[x_1, \ldots, x_n] := \left\{ \text{polynomials in } x_1, \ldots, x_n \text{ with coefficients in } \mathbb{K} \right\}
	.\end{equation} 
\end{defn}

\begin{defn}[(affine) Algebraic set]
	Let $S \subset \mathbb{K}[x_1, \ldots, x_n]$ a set of polynomials.
	We define the \textbf{zero set} (or \textbf{vanishing locus}) of $S$ as
	\begin{equation}
		\mathbb{V}(S) = Z(S) := \left\{ p \in \mathbb{A}^n \ \middle|\ f(p) = 0 \,\forall\, f \in S \right\} \subset \mathbb{A}^n
	.\end{equation} 
	Subsets of $\mathbb{A}^n$ of the form of $\mathbb{V}(S)$ are called \textbf{(affine) algebraic sets}.

	For finitely many polinomials $S = \left\{ f_1, \ldots, f_n \right\}$ we denote
	 \begin{equation}
		 \mathbb{V}(S) =: \mathbb{V}(f_1, \ldots, f_n)
	.\end{equation} 
\end{defn}

\begin{rem}
	The set $S$ defining the algebraic susbset $X = \mathbb{V}(S)$ is not unique. For instance:
	\begin{itemize}
		\item if $f$ and $g$ both vanish on $X$, then $f + g$ does so,
		\item if $f$ vanishes on $X$, then for all $h \in \mathbb{K}[x_1, \ldots, x_n]$, also $hf$ vanishes on $X$.
	\end{itemize}
	In particular we have, denoting by $\left( S \right)$ the ideal generated by $S$, the following equality
	\begin{equation}
		\mathbb{V}(S) = \mathbb{V}\left(\, (S)  \,\right)
	.\end{equation} 
\end{rem}

\begin{rem}
	In this course a ring $R$ will always be a commutative ring with unity.
\end{rem}

\subsection{Set-theoretical properties of algebraic sets}

\begin{lem}
	If $S_1$ and $S_2$ are sets of polynomials, then
	\begin{equation}
		\mathbb{V}(S_1) \cup \mathbb{V}(S_2) = \mathbb{V}(S_1 \cdot S_2)
	,\end{equation} 
	where $S_1 \cdot S_2 := \left\{ fg \ \middle|\ f \in S_1, g \in S_2 \right\}$.

	Moreover, for a family $\left\{ S_i \right\}_{i \in I}$, with $S_i \in \mathbb{K} \left[x_1, \ldots, x_n \right]$, we have
	\begin{equation}
		\bigcap_{i \in I}\mathbb{V}(S_i) = \mathbb{V} \big( \bigcup_{i \in I}S_i \big)
	.\end{equation} 
\end{lem} 
\begin{cor}
	Finite unions of algebraic sets and arbitrary intersections of algebraic sets are, again, algebraic sets.
\end{cor} 

\begin{defn}[Zariski topology]
	The \textbf{algebraic} subsets of $\mathbb{A}^n$ satisfy the properties for \textbf{closed} subsets in a topology: 
	\begin{itemize}
		\item $\emptyset, \mathbb{A}^n$ are algebraic,
		\item finite unions and arbitrary intersections of algebraic subsets are algebraic.
	\end{itemize}
	This means they induce a topology, which is called the \textbf{Zarisky} topology.
\end{defn}

From the above definition, in general, we will not refer to algebraic subsets of $\mathbb{A}^n$ as such, but as \textbf{closed} subsets in the \textbf{Zariski} topology.

\begin{rem}
	The \textbf{Zariski} topology is not \textbf{Hausdorff}.
	For example in $\mathbb{A}^1$ the \textbf{Zariski} topology is exactly the cofinite topology.
\end{rem}

\subsection{Hilbert's Nullstellensatz}
\begin{defn}[Associated ideal]
	Let $X \subset \mathbb{A}^n$, the ideal associated to $X$ is
	\begin{equation}
		\mathbb{I}(X) := \left\{ f \in \mathbb{K}\left[x_1, \ldots, x_n \right] \ \middle|\ f(p) = 0\ \,\forall\, p \in X \right\}
	.\end{equation} 
\end{defn}

\begin{lem}
	Let $S, T \subset \mathbb{K}\left[x_1, \ldots, x_n \right]$ and $X, Y \subset \mathbb{A}^n$, then
	\begin{enumerate}
		\item $\mathbb{V}$ and $\mathbb{I}$ reverse inclusions, i.e.
			\begin{align}
				S \subset T &\implies \mathbb{V}(T) \subset \mathbb{V}(S)\\
				X \subset Y &\implies \mathbb{I}(Y) \subset \mathbb{I}(X)
			,\end{align} 
		\item $X \subset \mathbb{V}(\mathbb{I}(X))$ and $S \subset \mathbb{I}\left( \mathbb{V}(S) \right)$,
		\item if $X$ is algebraic, then $X = \mathbb{V}\left(\mathbb{I}(X)\right)$.
	\end{enumerate}
\end{lem} 

\begin{rem}
	In general, given an ideal $\mathcal{I} \subset \mathbb{K}\left[x_1, \ldots, x_n \right]$, the identity
	\begin{equation}
		\mathcal{I} = \mathbb{I}\left( \mathbb{V}\left(\mathcal{I}\right) \right)
	\end{equation} 
	is false.
\end{rem}

\begin{lem}
	Let $\left( a_1, \ldots, a_n \right) \in \mathbb{A}^n$, then
	\begin{equation}
		\mathbb{I}\left( \left\{ a \right\} \right) = \left( x_1 - a_1, \ldots, x_n - a_n \right)
	.\end{equation} 
\end{lem} 

\begin{thm}[Hilbert's Nullstellensatz]
	Let $\mathbb{K}$ be an algebraically closed field and $f_1, \ldots, f_n \in \mathbb{K}\left[x_1, \ldots, x_n \right]$.
	If $\left( f_1, \ldots, f_n \right) \neq \left( 1 \right)$ (i.e. it is a proper ideal), then $\mathbb{V}\left(f_1, \ldots, f_n\right) \neq \emptyset$ in $\mathbb{A}^n(\mathbb{K})$.
\end{thm}

\begin{rem}
	This theorem is the reason why we work with algebraically closed fields, moreover, if $\mathbb{K}$ is not algebraically closed, then a common zero of $f_1, \ldots, f_n$ will exist over a finite extension of $\mathbb{K}$.
\end{rem}

\begin{cor}
	If $\mathbb{K}$ is algebraically closed, then $\mathbb{I}$ and $\mathbb{V}$ are bijections between the set of points in $\mathbb{A}^n$ and the set of maximal ideals in $\mathbb{K}\left[x_1, \ldots, x_n \right]$.
\end{cor} 

\begin{defn}[Radical of an ideal]
	Let $R$ be a ring and $\mathcal{I} \subset R$ be an ideal of $R$.
	We define the radical of $\mathcal{I}$, denoted with $\sqrt{\mathcal{I}}$, to be
	\begin{equation}
		\sqrt{\mathcal{I}} := \left\{ x \in R \ \middle|\ \exists\, n \in \N_+, \text{ s.t. } x^n \in \mathcal{I} \right\}
	.\end{equation} 
	If $\mathcal{I} = \sqrt{\mathcal{I}}$ we say that the ideal $\mathcal{I}$ is radical.
\end{defn}

\begin{prop}[Nullstellensatz]
	Let $\mathbb{K}$ be algebraically closed and $\mathcal{I} \subset \mathbb{K}\left[x_1, \ldots, x_n \right]$ be an ideal, then
	\begin{equation}
		\mathbb{I}\left( \mathbb{V}\left(\mathcal{I}\right) \right) = \sqrt{\mathcal{I}}
	.\end{equation} 
\end{prop} 

\begin{cor}
	The maps $\mathbb{I}$ and $\mathbb{V}$, from algebraic sets in $\mathbb{A}^n(\mathbb{K})$ to radical ideals in  $\mathbb{K}\left[x_1, \ldots, x_n \right]$ are inclusion-reversing correspondences that are inverse to one another.
\end{cor} 

Let's now state some basic operations on algebraic sets:
\begin{lem}
	Let $\mathcal{I}_1, \mathcal{I}_2 \subset \mathbb{K}\left[x_1, \ldots, x_n \right]$ ideals, then
	\begin{enumerate}
		\item $\displaystyle{\mathbb{V}\left(\mathcal{I}_1\right) \cap_{} \mathbb{V}\left(\mathcal{I}_2\right) = \mathbb{V}\left(\mathcal{I}_1 + \mathcal{I}_2\right)}$,
		\item $\displaystyle{\mathbb{V}\left(\mathcal{I}_1\right) \cup_{} \mathbb{V}\left(\mathcal{I}_2\right) = \mathbb{V}\left(\mathcal{I}_1 \mathcal{I}_2\right) = \mathbb{V}\left(\mathcal{I}_1 \cap_{} \mathcal{I}_2\right)}$.
	\end{enumerate}
\end{lem} 

\begin{lem}
	Let $X, Y \subset \mathbb{A}^n$ algebraic subsets, then
	\begin{enumerate}
		\item $\displaystyle{\mathbb{I}\left( X \cup_{} Y \right) = \mathbb{I}\left( X \right) \cap_{} \mathbb{I}\left( Y \right)}$,
		\item $\displaystyle{\mathbb{I}\left( X \cap_{} Y \right) = \sqrt{\mathbb{I}\left( X \right) + \mathbb{I}\left( Y \right)}}$.
	\end{enumerate}
\end{lem} 

\subsection{Reducibility}
\begin{defn}[Reducibile subsets]
	A topological space $X$ is \textbf{reducible} iff there exist proper closed subsets $X_1, X_2 \subsetneq X$ s.t.
	\begin{equation}
	X = X_1 \cup_{} X_2
	.\end{equation} 
	If $X$ is not reducible, it is said to be \textbf{irreducible}.
\end{defn}
\begin{defn}[Connected space]
	If $X$ is reducible and it is a disjoint union of closed subsets
	\begin{equation}
	X = X_1 \cup_{} X_2
	,\end{equation} 
	then we say that $X$ is \textbf{disconnected}.
	If $X$ is not disconnected it is said to be \textbf{connected}.
\end{defn}
\begin{rem}
	In order to be \textbf{disconnected}, a space $X$ has to be \textbf{reducible}, i.e. there exist no irreducible disconnected spaces.
\end{rem}

\begin{rem}
	If a space is \textbf{Hausdorff}, then only the singletons are \textbf{irreducible}.
\end{rem}

\begin{defn}[Affine variety]
	An irreducible algebraic set in $\mathbb{A}^n$, with the Zariski topology, is called an \textbf{affine variety}.
\end{defn}

\begin{lem}[Algebraic characterization of irreducibility]
	An algebraic set $X \subset \mathbb{A}^n$ is an affine variety iff $\mathbb{I}(X) \subset \mathbb{K}\left[x_1, \ldots, x_n \right]$ is a prime ideal.
\end{lem} 
\begin{rem}
	Recall: $\mathcal{I} \subset R$ is a prime ideal iff $R/\mathcal{I}$ is an integral domain.
\end{rem}

\begin{defn}[Noetherian topological space]
	A topological space $X$ is called \textbf{Noetherian} iff every descending chain of closed subset
	\begin{equation}
	X \supset X_1 \supset X_2 \supset X_3 \supset \ldots \supset X_n \supset X_{n+1} \supset \ldots
	\end{equation} 
	is stationary.
\end{defn}

\begin{rem}
	Each algebraic set, in the Zariski topology, is \textbf{Noetherian}.
	This follows from the fact that $\mathbb{K}\left[x_1, \ldots, x_n \right]$ (hence every quotient of this ring) is a Noetherian ring, therefore, by Nullstellensatz,
	\begin{equation}
		\mathbb{I}(X) \subset \mathbb{I}(X_1) \subset \mathbb{I}(X_2) \subset \ldots \subset \mathbb{I}(X_n) \subset \mathbb{I}(X_{n+1})
	.\end{equation} 
	Since the maps $\mathbb{V}$ and $\mathbb{I}$ reverse inclusions we have our claim. 
\end{rem}

\begin{prop}
	Every \textbf{Noetherian} topological space $C$ can be written as
	\begin{equation}
	X = X_1 \cup_{} X_2 \cup_{} \ldots \cup_{} X_r
	,\end{equation} 
	a finite union of disjoint irreducible closed subsets, with $X_1 \not\subset X_j$ for $j \neq i$.
	Moreover this decomposition is unique up to order of $X_1, \ldots, X_r$.
\end{prop} 

\begin{defn}[Irreducible components]
	The irreducible subsets $X_1, \ldots, X_r$ are called the \textbf{irreducible components} of $X$.
\end{defn}

\begin{cor}
	Every affine algebraic set $X$ is the finite union of affine varieties $X_1, \ldots, X_r$, i.e.
	\begin{equation}
	X = X_1 \cup_{} \ldots \cup_{} X_r
	.\end{equation} 
	From the algebraic point of view this can be viewed using Hilbert Nullstellesatz:
	let $\mathbb{I}(X)$ be the radical ideal associated to $X$ and $\mathbb{I}(X_i)$ the prime ideals associated to $X_i$, then
	\begin{equation}
		\mathbb{I}(X) = \mathbb{I}\left( X_1 \cup_{} \ldots \cup_{} X_r \right) = \mathbb{I}\left( X_1 \right) \cap_{} \ldots \cap_{} \mathbb{I}\left( X_r \right)
	,\end{equation} 
	i.e. every radical ideal in $\mathbb{K}\left[x_1, \ldots, x_n \right]$ is the finite intersection of prime ideals.
\end{cor} 

\begin{rem}
	If one reads (and adapts) the proof of the above proposition, one can prove that each algebraic set is the disjoit union of finitely many connected components.
\end{rem}

\begin{defn}[Dimension of a Noetherian topological space]
	Let $X$ be a nonempty irreducible topological space.
	We define the dimension of $X$ to be the largest index $n$ s.t. there exist a chain of length $n+1$ 
	\begin{equation}
	\emptyset \neq X_0 \subsetneq X_0 \subsetneq X_1 \subsetneq \ldots \subsetneq X_n = X
	\end{equation} 
	of irreducible closed subsets of $X$.

	If $X$ is an arbitrary Noetherian topological space (it might be reducible), the dimension of $X$ is defined to be the supremum (hence the maximum) of the dimensions of its irreducible components.
\end{defn}
\begin{rem}
	If all components of $X$ have the same dimension we say that $X$ is \textbf{pure-dimensional}.
	One gives special names to small pure-dimensional topological spaces:
	\begin{itemize}
		\item if $X$ is of pure dimension $1$, we call $X$ a \textbf{curve},
		\item if $X$ is of pure dimension $2$, we call $X$ a \textbf{surface.}
	\end{itemize}
\end{rem}

\begin{rem}[Characterization of irreducibility]
	The following properties are equivalent:
	\begin{itemize}
		\item $X$ is irreducible,
		\item any two nonempty open subsets of $X$ have a nonempty intersection,
		\item every nonempty open subset $U$ is dense in $X$, i.e. $\overline{U} = X$,
		\item every nonempty open subset of $X$ is connected.
	\end{itemize}
	Moreover, the image of an irreducible subset under a continuous map is again irreducible.
\end{rem}

\begin{defn}[Hypersurface]
	$X \subset \mathbb{A}^n$ is an \textbf{hypersurface} iff $X = \mathbb{V}\left( f \right)$ for a single polynomial $f$ of positive degree.
\end{defn}

\begin{ex}[Irreducible varieties]\leavevmode\vspace{-.2\baselineskip}
	\begin{itemize}
		\item Every linear subspace of $\mathbb{A}^n$ is irreducible (i.e. it is an affine variety).
			As a consequence we obtain that $\mathrm{dim}\mathbb{A}^n \geq n$.
		\item An hypersurface $X = \mathbb{V}\left( f \right)$ is irreducible iff $f$ is a power of an irreducible polynomial.
	\end{itemize}
\end{ex} 

\subsection{Functions and Morphisms}
\begin{defn}[Regular function]
	Given $X \subset \mathbb{A}^n$ a \textbf{Zariski} closed subset, a function
	\begin{equation}
	f: X \to \K
	\end{equation} 
	is said to be \textbf{regular} iff there exist $F \in \mathbb{K}\left[x_1, \ldots, x_n \right]$ s.t.
	\begin{equation}
		f(a) = F(a) \quad \,\forall\,  a \in X
	.\end{equation} 
\end{defn}
\begin{rem}
	Notice that, in the above definition, $F$ is unique up to an element in $\mathbb{I}(X)$.
\end{rem}

\begin{defn}[Coordinate ring]
	The coordinate ring of $X$ is
	\begin{equation}
		\underbrace{\K[X] := \mathbb{A}(X)}_{\text{alternative notations}} := \mathbb{K}\left[x_1, \ldots, x_n \right]/\mathbb{I}(X)
	.\end{equation} 
	It is the set of all regular functions on $X$
\end{defn}

\begin{constr}[Localization]
	Let $R$ be a ring.
	A set $S \subset R$ is \textbf{multiplicatively closed} iff $1 \in S$ and $\,\forall\, f,g \in S$ then also $fg \in S$.\newline
	We construct on $R \cross S$ the equivalence relation
	\begin{equation}
		\left(f, g\right) \sim \left(f', g'\right) \iff \exists\, h \in S \text{ s.t. } h \left( fg' - f'g \right) = 0
	.\end{equation} 
	We then define the localization of $R$ at $S$ as
	\begin{equation}
	S^{-1} R := (R \cross S)/\sim
	.\end{equation}
	We will denote the elements of the localization as fractions:
	\begin{equation}
	\left[ \left(f, g\right) \right] =: \frac{f}{g}
	.\end{equation} 
\end{constr} 
\begin{rem}\leavevmode\vspace{-.2\baselineskip}
	\begin{itemize}
		\item If $R$ is a \textbf{domain}, then we can always take $h = 1$ in the definition of the equivalence relation.
			In this case, if $S \subset S'$ are multiplicatively closed subsets, then
			\begin{align}
				S^{-1}R &\hookrightarrow (S')^{-1} R\\
				\frac{f}{g} &\mapsto \frac{f}{g}
			\end{align} 	
			is injective, provided $0 \not\in S'$.
		\item $S^{-1} R$ is a ring, with addition and multiplication defined as usual with fractions:
			\begin{align}
				\frac{f}{g} +  \frac{f'}{g'} &:= \frac{fg' + f'g}{gg'},\\
				\frac{f}{g} \cdot \frac{f'}{g'} &:= \frac{ff'}{gg'}
			.\end{align} 
	\end{itemize}
\end{rem}

\begin{defn}[Local ring]
	Let $R$ be a ring. We say that $R$ is \textbf{local} iff it has a unique maximal ideal.
\end{defn}

\begin{ex}\leavevmode\vspace{-.2\baselineskip}
	\begin{enumerate}
		\item Let $a \in R$ be an arbitrary element.
			Let $S := \left\{ a^n \ \middle|\ n \in \N \right\}$, then we define
			\begin{equation}
			R_a := S^{-1} R
			,\end{equation} 
			and we call it the localization of $R$ at $a$.
		\item Let $\mathcal{P} \subset R$ a \textbf{prime} ideal of $R$. let $S := R \setminus \mathcal{P}$, then we denote with
			\begin{equation}
				R_{\mathcal{P}} := S^{-1} R
			\end{equation} 
			the localization of $R$ at $\mathcal{P}$.
			Here we can describe the ideals of $R_{\mathcal{P}}$, they are of the form
			\begin{equation}
				\left\{ \frac{f}{g} \ \middle|\  f \in \mathcal{I}, g \not\in \mathcal{P} \right\}
			,\end{equation} 
			for some $\mathcal{I} \subset \mathcal{P}$ ideal of $R$.
			
			In particular $R_{\mathcal{P}}$ has a unique maximal ideal
			\begin{equation}
			M = \left\{ \frac{f}{g} \ \middle|\ f \in \mathcal{P}, g \not\in \mathcal{P} \right\}
			\end{equation}
			hence it is a \textbf{local} ring.
		\item Let $S := R \setminus \left\{ \text{zero divisors} \right\}$.
			This clearly is a multiplicatively closed subset, since $\left\{ \text{zero divisors} \right\}$ is a prime ideal of $R$.

			If $R$ is an integral domain, then $S = R \setminus \left\{ 0 \right\}$
			and we denote the localization as
			\begin{equation}
				Q(R) := S^{-1} R
			,\end{equation} 
			the field of fractions of R.
	\end{enumerate}
\end{ex} 

\begin{rem}
	When $X$ is an affine \textbf{variety}, then $\mathbb{I}(X)$ is prime, hence $\K[X]$ is an integral domain.
\end{rem}

\begin{defn}[Field of rational functions]
	Let $X \subset \mathbb{A}^n$ be an affine variety.
	We define $\K(X) := Q \left( \K[X] \right)$ to be the field of \textbf{rational functions} on $X$.
\end{defn}

\begin{rem}
	We call the elements of  $\K(X)$ rational functions even if their values may not be defined on all points of $X$.
\end{rem}

\begin{defn}[Regular functions]\leavevmode\vspace{-.2\baselineskip}
	\begin{enumerate}
		\item We say that $\varphi \in \K(X)$ is \textbf{regular at} $p \in X$ iff
			\begin{equation}
				\varphi = \frac{f}{g}, \text{ with }  f,g \in \K[X] \text{ and } g(p) \neq 0
			,\end{equation} 
			i.e. $\varphi$ is well defined at $p$,
		\item We define the \textbf{local ring of} $X$ at $p \in X$ to be
			\begin{equation}
				\mathcal{O}_{X,p} := \left\{ \varphi \in \K(X) \ \middle|\ \varphi \text{ is regular at } p \right\}
			,\end{equation} 
		\item For any nonempty open $U \subset X$, we define
			\begin{equation}
				\mathcal{O}_X (U) := \bigcap_{p \in U} \mathcal{O}_{X,p}
			,\end{equation} 
			the ring of regular functions on $U$.
	\end{enumerate}
\end{defn}

\begin{rem}
	In general
	\begin{equation}
		\mathcal{O}_X (U) \neq \left\{ \frac{f}{g} \ \middle|\ f,g \in \K[X],\ g(p) \neq 0\ \,\forall\, p \in U  \right\}
	.\end{equation} 
\end{rem}

\begin{defn}[$\K$-algebra]
	A $\K$-algebra is a ring $R$ which is also a $\K$-Vector Space, s.t. the ring multiplication is $\K$-bilinear, i.e.
	\begin{equation}
		\lambda(fg) = (\lambda f)g = f(\lambda g)
	,\end{equation} 
	for $f,g \in R$ and $\lambda \in \K$.

	Moreover we define morphisms of $\K$-algebras as those maps between $\K$-algebras that are both $\K$-linear and ring homomorphisms.
\end{defn}

\begin{rem}
	Rings of functions are always $\K$-algebras
\end{rem}

\begin{lem}[Identity principle]
	Let $U \subset V \subset X$ be nonempty open subsets of an affine variety $X$.
	If $\phi_1, \phi_2 \in \mathcal{O}_X(V)$ coincide when restricted to $U$, then they are the same function also on $V$, i.e. for $\phi_1, \phi_2 \in \mathcal{O}_X(V)$
	\begin{equation}
		\left.\phi_1\right|_{U} = \left.\phi_2\right|_{U} \in \mathcal{O}_X(U) \implies \phi_1 = \phi_2 \in \mathcal{O}_X(V)
	.\end{equation} 
	It is the same property that holds for holomorphic functions.
\end{lem} 

\begin{rem}
	Recall that an affine variety $X$ is irreducible, this means that any open subset in $X$ is dense.
	With this in mind the above result should not sound too strange.
	It actually seems less strong than what can be proved for holomorphic functions.
\end{rem}

\begin{defn}[Distinguished open subsets]
	Let $X \subset \mathbb{A}^n$ be an affine variety and $f \in \K[X]$ a regular function on $X$, then
	\begin{equation}
		X_f := \left\{ p \in X \ \middle|\ f(p) \neq 0 \right\} = X \setminus \mathbb{V}\left( F \right)
	,\end{equation} 
	for some $F \in \mathbb{K}\left[x_1, \ldots, x_n \right]$ s.t. $\left[ F \right] = f \in \K[X]$, is an open subset of $X$ (in the Zariski topology).
	It is in fact the complement of a hypersurface, which is closed in $X$.
	Such open subsets are called \textbf{distinguished open} subsets of $X$.
\end{defn}

\begin{rem}
	The distinguished open subsets form a base for the Zariski topology on $X$.
	It means that any open subset can be written as a union of distinguished open subsets.

	By Hilbert's basis theorem we can say even more: every closed subset is the intersection of finitely many hypersurfaces, hence (taking the complements) every open subset is a finite union of distinguished open subsets.
\end{rem}

\begin{prop}
	For any $f \in \K[X]$ we have
	\begin{equation}
		\mathcal{O}_X(X_f) = \left\{ \phi\in \K(X) \ \middle|\ \phi = \frac{g}{f^r} \text{ for some } r \geq 0,\ g \in \K[X] \right\}
	.\end{equation} 
	In particular, $\mathcal{O}_X(X_f) \cong K[X]_f$.
\end{prop} 
\begin{cor}
	If $f = 1$, then $X_f = X$, hence the proposition grants
	\begin{equation}
		\mathcal{O}_X(X) \cong \K[X]
	,\end{equation} 
	hence the rational functions which are regular at each point of $X$ are exactly the regular functions on  $X$.
\end{cor} 

\begin{rem}
	Also the local ring $\mathcal{O}_{X,p}$ is a localization of $\mathbb{K}\left[X\right]$:
	\begin{equation}
		\mathfrak{m}_{X,p} := \left\{ g \in \K[X] \ \middle|\ g(p) = 0 \right\} = \mathbb{I}(\{p\})/\mathbb{I}(X) \subset \K[X]
	\end{equation}
	is a maximal, hence prime, ideal.
	This is true, since the map
	\begin{align}
		\K[X]/\mathfrak{m}_{X,p} &\to \K\\
		\left[ g \right] &\mapsto g(p)
	\end{align} 
	is an isomorphism.
	Moreover one can prove that the natural map
	\begin{align}
		\mathcal{O}_{X,p} &\to \K[X]_{\mathfrak{m}_{X,p}}\\
		\frac{f}{g} &\mapsto \frac{f}{g}
	,\end{align} 
	with $g(p) \neq 0$, is an isomorphism.
\end{rem}

\begin{rem}
	In general, for $U \subset X$ an arbitrary open subset, in order to compute $\mathcal{O}_X(U)$ we proceed as follows:
	\begin{itemize}
		\item we write $U$ as the union of finitely many distinguished open subsets (coming from the generators of $\mathbb{I}(X\setminus U)$),
		\item we use the decomposition $U = X_{f_1} \cup_{} X_{f_2} \cup_{} \ldots \cup_{} X_{f_l}$ and $\mathcal{O}_X(X_{f_j}) = \K[X]_{f_j}$ to figure out what $\mathcal{O}_X(U)$ is.
	\end{itemize}
\end{rem}

\begin{ex}
	Let $X = \mathbb{A}^{2}$ and $X \supset U := \mathbb{A}^{2} \setminus \left\{ \left( 0,0 \right) \right\}$.
	We can write $U = \mathbb{A}_x^2 \cup \mathbb{A}_y^2$.
	Carrying out some computations on these distinguished open subsets we can arrive at the conclusion that
	\begin{equation}
	\mathcal{O}_{\mathbb{A}^{2}} \left( \mathbb{A}^{2} \setminus \left\{ \left(0, 0\right) \right\} \right) =
	\mathbb{K}[x,y] = \mathbb{K}[\mathbb{A}^{2}]
	.\end{equation} 
	This result can be interpeted as:
	"the point $\left(0, 0\right)$ is a removable singularity".
\end{ex} 

\begin{rem}
	Since we can check regularity at a neighborhood of single points, then regularity is a \textbf{local} property.
\end{rem}

