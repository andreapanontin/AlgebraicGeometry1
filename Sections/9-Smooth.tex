\subsection{Smooth varieties}
\begin{defn}[Tangent space at a point]
	Let $X \subset \mathbb{A}^{n}$ be an affine variety and $p \in X$.
	After a linear change of coordinates
	\begin{equation}
	X_1 := x_1 - p_1, \qquad X_2 := x_2 - p_2, \qquad \ldots, \qquad
	X_n := x_n - p_n
	\end{equation} 
	we can assume that $p = \left( 0, \ldots, 0 \right)$.
	We define the \textbf{tangent space} to $X$ at $p$ to be
	\begin{equation}
	T_{X,p} := \mathbb{V}\left( \left\{ f_1 \in \mathbb{K}\left[x_1, \ldots, x_n \right] \ \middle|\ 
	f \in \mathbb{I}(X)\right\} \right)
	,\end{equation} 
	in which $f_1$ represents the linear part of $f = \sum_{j \in \N}^{} f_j$, with
	$f_j \in \mathbb{K}\left[x_1, \ldots, x_n \right]_j$ are the homogeneous components of $f$.
\end{defn}

\begin{rem}[]
	Consider $p = \left( p_1, \ldots, p_n \right)$.
	We can consider the Taylor expansion of $f \in \mathbb{K}\left[x_1, \ldots, x_n \right]$ at $p$
	(after all it is a polynomial):
	\begin{equation}
		f \left( x_1, \ldots, x_n \right) =
		f(p) + \sum_{i=1}^{n} \frac{\partial f}{\partial x_i} (p) \left( x_i - p_i \right) +
		h.o.t.
	.\end{equation} 
	Then $T_{X,p}$ is defined by the linear equations
	\begin{equation}
		\sum_{i=1}^{n} \frac{\partial f}{\partial x_i} (p) \left( x_i - p_i \right) = 0
	,\end{equation} 
	for all $f \in \mathbb{I}(X)$ (for which, hence, $f(p) = 0$).
\end{rem}

\begin{rem}[]
	The degree $1$ part of an ideal is always contained in the initial ideal,
	i.e. $\mathcal{I}_1 \subset \mathcal{I}^{in}$.
	It follows that $C_{X,p} \subset T_{X, p}$, i.e. the tangent cone is contained in the tangent space.
\end{rem}
 
Let's give a more intrinsic definition to the tangent space:
\begin{lem}
	Let $p \in X \subset \mathbb{A}^{n}$ a point of an affine variety.
	Then the Vector Space $T_{X,p} \subset \K^n$ is isomorphic to the space of
	linear forms on $\mathfrak{m}_{X,p}/\mathfrak{m}^2_{X,p}$, where
	\begin{equation}
		\mathfrak{m}_{X,p} = \left\{ \varphi \in \mathcal{O}_{X,p} \ \middle|\ \varphi(p) = 0 \right\}
	\subset \mathcal{O}_{X,p} \subset \K(X)
	,\end{equation} 
	which can be identified with the germs of functions vanishing at $p$.
	Analogously, for $\mathfrak{m}^2_{X,p}$, we have
	\begin{equation}
	\mathfrak{m}^2_{X,p} = \left\langle \varphi \cdot \psi \ \middle|\ 
\varphi, \psi \in \mathfrak{m}_{X,p} \right\rangle
	\subset \mathcal{O}_{X,p} \subset \K(X)
	,\end{equation} 
	i.e. it contains the germs of functions vanishing at order $\geq 2$ at $p$.
	Then
	\begin{equation}
		T_{X,p} \cong \left( \mathfrak{m}_{X,p}/\mathfrak{m}^2_{X,p} \right)^{\vee}
	.\end{equation} 
\end{lem} 
\begin{proof}
	We can assume, wlog, that $p = \left( 0, \ldots, 0 \right)$.
Then, the coordinates $x_1, \ldots, x_n$ of $\mathbb{A}^{n} \cong \K^n$ define linear forms in $V^{\vee}$.
In particular $x_1, \ldots, x_n$ is a basis of $V^{\vee} = \mathrm{Hom}_{\K}\left( V, \K \right)$.
Let, now, $\mathcal{I} := \mathbb{I}(X)$ and $\mathcal{I}_1 \subset \mathbb{K}\left[x_1, \ldots, x_n \right]_1 = V^{\vee}$
be a linear subspace.
\begin{description}
	\item[Step 1:] \textit{$V/\mathcal{I}_1$ is canonically isomorphic to
		$T^{\vee}_{X,p} = \mathrm{Hom}_{\K}\left( T_{X,p}, \K \right)$}.

		We define the restriction morphism $\varphi: V^{\vee} \to T_{X,p}^{\vee}$
		that sends any linear form $l \in V^{\vee}$ to itself, as a linear form in $T^{\vee}_{X,p}$.
		This is clearly a linear map.
		Moreover $T^{\vee}_{X,p} = \mathrm{span} \left( \varphi(x_1), \ldots, \varphi(x_n) \right)$, 
		hence $\varphi$ is surjective.
		Finally $T_{X,p} = \mathbb{V}\left( \mathcal{I}_1 \right)$, hence
		$\ker \varphi = \mathcal{I}_1$.

	\item[Step 2:] \textit{There is a natural isomorphism}
		\begin{align}
			\psi: V^{\vee} / \mathcal{I}_1 &\to \mathfrak{m}_{X,p}/\mathfrak{m}^2_{X,p} \\
			x_j &\mapsto [x_j]
		.\end{align} 
		Recall that $p = 0$, hence any $x_j$ identifies an equivalence class in $\mathfrak{m}_{X,p}/\mathfrak{m}^2_{X,p}$.
		\textit{Injectivity:} Let $0 \neq \alpha \in \mathbb{K}\left[x_1, \ldots, x_n \right]_1$ be 
		s.t. $[\alpha] = 0 \in \mathfrak{m}_{X,p}/\mathfrak{m}^2_{X,p}$.
		Then $\alpha \in \mathcal{O}_{X,p}$ belongs to $\mathfrak{m}^2_{X,p}$, i.e.
		\begin{equation}
			\alpha = \frac{F_1}{G_1} \cdot \frac{F_2}{G_2} \in \K \left( x_1, \ldots, x_n \right)
		,\end{equation} 
		for $F_i, G_i \in \mathbb{K}\left[x_1, \ldots, x_n \right]$,
		with $F_i(p) = 0$ and $G_i(p) \neq 0$ for $i = 1,2$.
		Then
		\begin{equation}
		G_1G_2 \alpha = F_1F_2 \in \mathbb{K}\left[x_1, \ldots, x_n \right]
		.\end{equation} 
		Then, since $\alpha$ is of degree $1$, it is irreducible (hence prime).
		Since $G_1$ cannot divide $F_i$ (they do not vanish at $p$, then $\alpha$ divides the product $F_1F_2$,
		then $\alpha$ divides one of the two, let's say $F_1 = \alpha \widetilde{F_1}$.
		Then $G_1G_2 = \widetilde{F_1}F_2$, but
		\begin{equation}
			0 \neq \left( G_1G_2 \right)(p) = \left( \widetilde{F_1}F_2 \right)(p) = 
			\widetilde{F_1}(p) F_2(p) = 0
		.\end{equation} 
		This is a contradiction, hence $\alpha = 0$.

		\textit{Surjectivity:} Let $\phi = \frac{f}{g} \in \mathfrak{m}_{X,p}$, wlog $g(p) = 1$.
		We want to prove that
		\begin{equation}
			\phi' := \sum_{i}^{} \frac{\partial \phi}{\partial x_i} (p) x_i
			\qquad \text{ where } \qquad
			\frac{\partial \phi}{\partial x_i} =
			\frac{\frac{\partial f}{\partial x_i} \cdot g - \frac{\partial g}{\partial x_i} \cdot f }{g^2}
		\end{equation} 
		is the inverse image of $\phi$, i.e. $\psi(\phi') - \phi \in \mathfrak{m}^2_{X,p}$.
		Let's multiply by $g$, to obtain
		\begin{align}
			g \left( \phi' - \phi  \right) &=
			f - \frac{g}{g^2} \sum_{i = 1}^{n} \left( \frac{\partial f}{\partial x_i} (p) g(p) -
			\frac{\partial g}{\partial x_i} (p) f(p) \right)x_i\\
			&\equiv f - \frac{g(p)}{g^2(p)} \sum_{i=1}^{n} \frac{\partial f}{\partial x_i} (p) x_i
			\mod \mathfrak{m}^2_{X,p}\\
			&\equiv f - \sum_{i=1}^{n} \frac{\partial f}{\partial x_i} (p) x_i \mod \mathfrak{m}^2_{X,p}
		.\end{align} 
\end{description} 
\end{proof}

\begin{rem}[]
	The same proof yields
	\begin{equation}
	T_{X,p}^{\vee} \cong M/M^2 \qquad \text{ for } \qquad
	M = \mathbb{I}(p)/\mathbb{I}(X) \subset \K[X]
	.\end{equation} 
\end{rem}
\begin{rem}[]
	The identification $T_{X,p}^{\vee} \cong \mathfrak{m}_{X,p}/\mathfrak{m}^2_{X,p}$ is
	intrinsic, i.e. independent of the embedding in $\mathbb{A}^{n}$.
	In particular we can use it to define $T_{X,p}$ for $X$ any
	variety and $p \in X$ any point.
	In this case $T_{X,p}$ is an abstract Vector Space (not
	embedded in $\mathbb{A}^{n}$ anymore).
\end{rem}
 
\begin{rem}[Comparison with tangent cones]
	Let $X \subset \mathbb{A}^{n}$ be an affine variety, and $p \in X$.
	WLOG we assume that $p = \left( 0, \ldots, 0 \right)$.
	Then, for $f_0 + f_1 + \ldots + f_d = f \in \mathbb{K}\left[x_1, \ldots, x_n \right]$,
	with $f(p) = 0$ (i.e. $f_0 = 0$), we have
	$f_1 \neq 0 \implies \mathrm{mindeg}\, f = 1$ and $f_1 = f^{in}$.
	In particular $\mathbb{I}\left( X \right)_{in} = \left\{ f^{in} \ \middle|\ 
	f \in \mathbb{I}\left( X \right) \right\} \supset \mathbb{I}\left( X \right)_1$.
	Then $C_{X,p} = \mathbb{V}\left( \mathbb{I}\left( X \right)_{in} \right) \subset
	\mathbb{V}\left( \mathbb{I}\left( X \right)_1 \right) = T_{X,p}$ in $\mathbb{A}^{n}$.

	In particular we obtain that $\dim X = \dim C_{X,p} \leq \dim T_{X,p}$.
	The dimension of the tangent (linear space) might be bigger than the dimension of the cone.
\end{rem}

\begin{defn}[Regular and singular points/varieties]
	Given a variety $X$ we say that a point $p \in X$ is a \textbf{regular point} of $X$
	iff $\dim T_{X,p} = \dim X$.
	(Alternatively it can be called \textbf{simple}, \textbf{smooth}, \textbf{non-singular} point).
	If, instead, $\dim T_{X,p} > \dim X$, then $p$ is called a \textbf{singular point}.

	A variety $X$ is called \textbf{singular} iff it has at least one singular point.
	Non-singular varieties are allso called \textbf{regular} or \textbf{smooth}.
	
	Finally, we define the regular locus of $X$ to be
	\begin{equation}
	X_{\text{reg}} := \left\{ p \in X \ \middle|\ X \text{ is regular at } p \right\}
	\end{equation} 
	and the singular locus of $X$ as
	\begin{equation}
	X_{\text{sing}} := X \setminus X_{\text{reg}}
	.\end{equation} 
\end{defn}

\begin{ex}[Curves in $\mathbb{A}^{2}$]
	Consider the parabola $X_1 := \mathbb{V}\left( x (x-1) - y \right)$,
	the nodal cubic $X_2 := \mathbb{V}\left( x^2 + x^3 - y^2 \right)$
	and the cuspidal cubic $X_3 := \mathbb{V}\left( x^3 - y^2 \right)$.
	Then $T_{X_1, \mathbf{0}} = C_{X_1, \mathbf{0}}$, hence the parabola is regular at $\mathbf{0}$.
	Instead $T_{X_2, \mathbf{0}} = T_{X_3, \mathbf{0}} = \mathbb{A}^{2}$, 
	hence $X_2$ and $X_3$ are both singular at $\mathbf{0}$.

	In fact the blow-ups at $\mathbf{0}$ are as follows.
	For $\widetilde{X}_1 \to X_1$ is an isomorphism, in fact it coincides with the strict transform.
	For $\widetilde{X}_2 \to X_2$ it is not even injective: the two tangent directions corresponing
	to the two components of the tangent cone $C_{X_2, \mathbf{0}}$ get both mapped to $\mathbf{0}$.
	$\widetilde{X}_3 \to X_3$, instead, is injective.
	Indeed both $\widetilde{X}_2$ and $\widetilde{X}_3$ are smooth, whereas $X_2$ and $X_3$ are singular
	(and this gives the fact that the above maps cannot be iso).
	Moreover we call the above maps from the smooth varieties, obtained by
	blowing up the singular locus, a resolution of singularities.
	(Those are birational map inducing isomorphisms $p^{-1} \left( X_{i, \text{reg}} \right) \to X_{i, \text{reg}}$
	s.t. $\widetilde{X}_i$ is smooth).
	In particular the maps $\widetilde{X}_i \xrightarrow{p} X_i$ (for $i = 2,3$)
	are resolutions of the singularities at $\mathbf{0}$.

	Let's check, for instance, that $\widetilde{X}_3$ is non-singular.
	We only need to check what happens at the preimage of $\mathbf{0}$.
	\begin{equation}
		\widetilde{\mathbb{A}^{2}} = \left\{ \left((x,y), [\zeta,\eta]\right) \ \middle|\ 
		x \eta = y \zeta\right\} \subset \mathbb{A}^{2} \cross \mathbb{P}^{1}
	.\end{equation} 
	The local coordinates at $\left( p, [1,0] \right)$ are $(u,v) \in \mathbb{A}^{2}$.
	Recall that the chart we are using is induced by
	\begin{align}
		\mathbb{A}^{2} &\to \widetilde{\mathbb{A}^{2}} = \mathbb{A}^{2} \cross \mathbb{P}^{1} \\
		\left(u, v\right) &\mapsto \left((v, uv), [1, u] \right)
	\end{align} 
	for $u = \eta/\zeta$ (for $\zeta \neq 0$) and $v = x$.
	The equation for $X_3$ is $y^2 - x^3 = 0$.
	We substitute $x = v$ and $y = uv$ and obtain $(uv)^2 - v^3 = v^2 \left( u^2 - v \right) = 0$.
	This has two components:
	$v = 0$, which defines the exceptional line of $\widetilde{\mathbb{A}^{2}} \to \mathbb{A}^{2}$,
	which is not contained in the stric transform $\widetilde{X}_3$, by definition.
	And $u^2 - v = 0$, which gives the local equation for the strict transform $\widetilde{X}_3$.
	It has a nontrivial linear part, hence $\widetilde{X}_3$ is smooth at $\left(p, [1,0]\right)$.

	With similar computations one can check that $\widetilde{X}_2$ is smooth
	at $(u,v) = (\pm1, 0)$, since the local equations are $v = 1 - u^2 = (1-u)(1+u)$.
\end{ex} 

In general, one needs to be careful, since one needs more than
one blow-up to obtain a resolution of singularities.
\begin{defn}[Jacobi matrix]
	Given $F := \left( f_1, \ldots, f_r \right)$ an $r$-tuple of polynomials in $x_1, \ldots, x_n$,
	we define the \textbf{Jacobi matrix} of $F$ at $p$ as
	\begin{equation}
		J_{F,p} := \left( \frac{\partial f_i}{\partial x_j}  \right)_{i,j}
		\qquad \text{ for } i =1, \ldots, r \text{ and } j = 1, \ldots, n
	.\end{equation} 
\end{defn}

\begin{prop}[Jacobi criterion]\leavevmode\vspace{-.2\baselineskip}
	\begin{description}
		\item[Affine case:] Let $X \subset \mathbb{A}^{n}$ be an affine variety, and
			$f_1, \ldots, f_r$ be generators of $\mathbb{I}\left( C \right)$.
			Then $p \in X$ is regular iff
			\begin{equation}
			\mathrm{rk}\, J_{\left( f_1, \ldots, f_r \right), p} \geq n - \dim X
			.\end{equation} 
		\item[Projective case:] Let $X \subset \mathbb{P}^{n}$ be a projective variety
			and $f_1, \ldots, f_r$ homogeneous generators of $\mathbb{I}_p\left( X \right)$.
			Then $p \in X$ is regular iff
			\begin{equation}
			\mathrm{rk}\, J_{\left( f_1, \ldots, f_r \right), p} \geq n - \dim X
			.\end{equation} 
			Notice that, even though the Jacobi matrix itself depends on the choice of
			homogeneous coordinates for $p$, its rank does not change (each row gets multiplied by
			a nonzero constant).
	\end{description}
	Moreover, in both cases, if $\mathrm{rk}\, J_{F,p} = r$,
	then $X$ is smooth of dimension $n - r$ at $p$.
\end{prop} 

\begin{rem}\leavevmode\vspace{-.2\baselineskip}
	\begin{itemize}
		\item Set $d := \dim X$, then in the above notation
			\begin{align}
				X_{\text{sing}} &= \left\{ p \in X \ \middle|\ 
			\mathrm{rk}\, J_{\left( f_1, \ldots, f_r \right), p} \leq n-d-1\right\}\\
				&= \mathbb{V}\left( \left\{ (n-d) \cross (n-d) \text{-minors of }
				J_{\left( f_1, \ldots, f_r \right), p}\right\} \right)		
			,\end{align} 
			hence $X_{\text{sing}} \subset X$ is a closed subset.
		\item $X_{\text{reg}} \subset X$ is always a nonempty open subset.
			In particular it is always dense.
			Let's give a sketch of the argument:
			Since $d := \dim X$, there is a surjective projection on $\mathbb{A}^{d}$.
			Though let's stop a step earlier and consider
			$\pi: X \to X' \subset \mathbb{A}^{d+1}$. Clearly $X'$ is a
			hypersurface in $\mathbb{A}^{d+1}$.
			Making a good choice of the projection there is a non empty open
			subset of $X$, on which $\pi$ is an isomorphism.
			Hence $\pi$ will be a birational map (it is a consequence of the
			primitive element theorem).
			It is easy to check that $X'_{\text{reg}} \neq \emptyset$, for hypersurfaces.
			Then (one should check that the inclusion holds)
			$X_{\text{reg}} \supset \pi^{-1} \left( X'_{\text{reg}} \cap V \right)$
			(for $V = \pi(U)$) is non empty.
	\end{itemize}
\end{rem} 

\begin{rem}[]
	If $X$ is a hypersurface, and
	\begin{itemize}
		\item $X = \mathbb{V}\left( f \right) \subset \mathbb{A}^{n}$, for a square-free polynomial $f$
			(not necessairily irreducible).
			A point $p \in \mathbb{A}^{n}$ is a singular point of $X$ iff
			\begin{equation}
				f(p) = \frac{\partial f}{\partial x_1} (p) = \ldots
				= \frac{\partial f}{\partial x_n} (p) = 0
			.\end{equation} 
		\item $X = \mathbb{V}_p\left( f \right) \subset \mathbb{P}^{n}$, for a suqare-free 
			and homogeneous polynomial $f$. Then
			\begin{equation}
			p \in X_{\text{sing}} \iff
			\frac{\partial f}{\partial x_0} (p) = \frac{\partial f}{\partial x_1} (p) =
			\ldots = \frac{\partial f}{\partial x_n} (p) = f(p) = 0
			.\end{equation} 
			Moreover one can easily check that, set $d := \deg f$,
			multiplying each partial derivative by the corresponding linear form
			\begin{equation}
				d \cdot f (x) = 
				x_0 \frac{\partial f}{\partial x_0} (x) + \ldots +
				x_n \frac{\partial f}{\partial x_n}
			.\end{equation} 
			(Clearly this only works for homogeneous polynomials).
			Moreover, if $d$ is invertible in the ground field (e.g.
			if the field is of char $0$), then one can omit the condition $f(p) = 0$.
			Then we have
			\begin{equation}
			X_{\text{sing}} = \mathbb{V}_p\left( \frac{\partial f}{\partial x_0}, \ldots,
			\frac{\partial f}{\partial x_n} \right)
			.\end{equation} 
	\end{itemize}
\end{rem}

\begin{ex}[Fermat hypersurfaces]
	Consider $n,d \geq 1$. The Fermat hypersurfaces of degree $d$ in $\mathbb{P}^{n}$ are defined by
	\begin{equation}
	X_d := \mathbb{V}_p\left( x_0^d + \ldots + x_n^d \right) \subset \mathbb{P}^{n}
	.\end{equation} 
	This is a degree $d$ hypersurface if $\mathrm{char}\, \K = 0, p$ for $p \nmid d$
	(otherwise we'd have $x_0^d + \ldots x_n^d = \left( x_0 + \ldots + x_n \right)^d$).
	Then the Jacobi matrix is
	\begin{equation}
		J_f = \left( d x_0^{d-1}, d x_1^{d-1}, \ldots, d x_n^{d-1} \right)
	.\end{equation} 
	Hence $X_d$ is a smooth hypersurface by the Jacobi criterion.
\end{ex}  

\begin{ex}[Twisted cubic curve]
	\begin{equation}
		X = \left\{ [s^3, s^2t, st^2, t^3] \ \middle|\ [s,t] \in \mathbb{P}^{1} \right\}
	.\end{equation} 
	Since $X \cong \mathbb{P}^{1}$, we already know that $X$ is regular.
	Let's check it with the Jacobi criterions:
	\begin{equation}
	X = \mathbb{V}_p\left( x_0x_2 - x_1^2, x_0x_3 - x_1x_2, x_1x_3 - x_2^2 \right)
	\end{equation} 
	(the above is determined by the determinant of the minors of the matrix 
	$\begin{pmatrix} x_0 & x_1 & x_2\\ x_1 & x_2 & x_3 \end{pmatrix}$).
	The Jacobi matrix is
	\begin{equation}
	J = 
	\begin{pmatrix}
		x_2 & -2 x_1 & x_0 & 0\\
		x_3 & -x_2 & -x_1 & x_0\\
		0 & x_3 & -2 x_2 & x_1\\
	\end{pmatrix} 
	.\end{equation} 
	We need to check that, at every point $\mathrm{rk}\, J \geq \dim \mathbb{P}^{3} - \dim X = 3 - 1 = 2$.
	We need to find $2 \cross 2$ minors that do not vanish simultaneosly:
	\begin{equation}
	\begin{vmatrix}
		x_0 & 0\\
		-x_1 & x_0
	\end{vmatrix} = x_0^2
	\qquad \text{ and } \qquad
	\begin{vmatrix}
		x_3 & -x_2 \\
		0 & x_3
	\end{vmatrix} = x_3
	.\end{equation} 
	The first vanishes only at $[0,0,0,1] \in X$, whereas the second at $[1,0,0,0] \in X$, 
	i.e. they have no common zeros on $X$, hence $\mathrm{rk}\, J \geq 2$ at all $p \in X$.
	In particular $X$ is regular.
\end{ex} 	
