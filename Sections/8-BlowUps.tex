\section{Blow-ups}

\begin{defn}[Rational map]
	Let $X,Y$ be varieties.
	A \textbf{rational map} $f: X \dashrightarrow Y$ is a morphism $f: U \to Y$, for some
	$\emptyset \neq U \subset X$ open.
	(Notice that we denote rational maps via a dashed arrow, since it might not be defined on the whole variety $X$).
	If $f: U \to Y$ and $f': V \to Y$ are morphisms with
	$\emptyset \neq U \subset X \supset V \neq \emptyset$ both open, then
	$f$ and $g$ define the same rational map iff
	\begin{equation}
	\left.f\right|_{U \cap V} = \left.g\right|_{U \cap V} 
	.\end{equation} 
	(Recall that, since $X$ is irreducible, then $U \cap V \neq \emptyset$).
\end{defn}

\begin{defn}[]\leavevmode\vspace{-.2\baselineskip}
	\begin{itemize}
		\item A rational map $f: X \dashrightarrow Y$, given by a morphism $f: U \to Y$, 
			is dominant iff $f(U)$ contains a non-empty open subset $V \subset Y$
			(in particular its image is dense in $Y$).
		\item If $f: X \dashrightarrow Y$ is dominant, for every $g: Y \dashrightarrow Z$, 
			the composition $g \circ f: X \dashrightarrow Z$ is a rational map.
		\item A birational map is a rational map $f: X \dashrightarrow Y$ with a rational inverse,
			i.e. $f$ is a dominant rational map s.t. there exists a dominant $g: Y \dashrightarrow X$ satisfying
			\begin{equation}
			g \circ f = id_X \qquad \text{ and } \qquad
			f \circ g = id_Y \qquad
			\text{(as rational maps)}
			.\end{equation} 
		\item Two varieties $X$ and $Y$ are said to be birational,
			or birationally equivalent, iff there is a birational map 
			$f: X \dashrightarrow Y$.
	\end{itemize}
\end{defn}

\begin{thm}[]
$X$ and $Y$ are birational iff there exist non-empty open subsets
$U \subset X$ and $V \subset Y$ that are isomorphic to each other.
\end{thm}

Let's now construct the most important class of birational maps: the blow-ups.
Since this is a local construction, we'll start from affine varieties.
\begin{defn}[Construction of blow-up]
	Let $X \subset \mathbb{A}^{n}$ be an affine variety, and $f_0, \ldots, f_r \notin \mathbb{I}(X)$
	be polynomials in $x_1, \ldots, x_n$.
	Then $U := X \setminus \mathbb{V}\left( f_0, \ldots, f_r \right)$ is a non-empty open
	subset of $X$.
	We can consider the morphism
	\begin{align}
		f: U &\to \mathbb{P}^{r} \\
		x &\mapsto \left[ f_0(x) , \ldots , f_r(x) \right]
	.\end{align} 
	The graph $\Gamma := \left\{ \left(p, f(p)\right) \ \middle|\ p \in U \right\} \subset X \cross \mathbb{P}^{r}$
	is isomorphic to $U$ via the projection $\left(p, q\right) \mapsto p$.
	We define the \textbf{blow-up} of $X$ at $\left( f_0, \ldots, f_r \right)$ to be 
	the Zariski closure
	\begin{equation}
	\widetilde{X} := \overline{\Gamma} \subset X \cross \mathbb{P}^{r}
	.\end{equation} 
\end{defn}

\begin{rem}[]\leavevmode\vspace{-.2\baselineskip}
	\begin{itemize}
		\item By definition $\widetilde{X}$ is closed in $X \cross \mathbb{P}^{r}$, hence it
			is a quasi-projective variety.
		\item $\widetilde{X}$ is irreducible, since $U \cong \Gamma$ is irreducible.
		\item $\dim \widetilde{X} = \dim X$.
		\item The projections $p_X: X \cross \mathbb{P}^{r} \to X$ and
			$p_{\mathbb{P}^{r}}: X \cross \mathbb{P}^{r} \to \mathbb{P}^{r}$
			induce the projection morphisms
			\begin{equation}
			\pi: \widetilde{X} \to X \qquad \text{ and } \qquad
			p: \widetilde{X} \to \mathbb{P}^{r}
			.\end{equation} 
	\end{itemize}
\end{rem}

\begin{ex}
	Let $r = 0$, then the blow-up of $X$ at $f_0 \notin \mathbb{I}(X)$ is isomorphic to $X$.
	In fact $\widetilde{X} \subset X \cross \mathbb{P}^{0} \cong X$, since $\mathbb{P}^{0}$ is just a point.
	Then an irreducible subvariety of $X \cross \mathbb{P}^{0}$, with the same dimension as $X$,
	is isomorphic to $X$.	
\end{ex} 

\begin{ex}
	Let $X = \mathbb{A}^{2}$, with coordinates $x_0, x_1$.
	Set $f_0 := x_0$ and $f_1 := x_1$.
	Then $\widetilde{X} \subset \mathbb{A}^{2} \cross \mathbb{P}^{1}$.
	Then
	\begin{align}
		f: U = \mathbb{A}^{2}\setminus \left\{ (0, 0) \right\} &\to \mathbb{P}^{1} \\
		(x_0, x_1) &\mapsto \left[ x_0 , x_1 \right]
	.\end{align} 
	The graph of $f$ is
	\begin{equation}
		\Gamma = \left\{ \left((x_0, x_1), [y_0,y_1] \right) \ \middle|\ 
		x_0y_1 = x_1y_0 \right\} \subset U \cross \mathbb{P}^{1}
	.\end{equation} 
	Then, if we compute its closure, we obtain
	\begin{equation}
		\widetilde{X} = \left\{ \left((x_0, x_1), [y_0,y_1] \right) \ \middle|\ 
		x_0y_1 = x_1y_0 \right\} \subset \mathbb{A}^{2} \cross \mathbb{P}^{1}
	,\end{equation} 
	in which we allow $(x_0, x_1) = (0,0)$.
	Denoting $\pi: \widetilde{X} \to X$ the projection, we obtain
	\begin{align}
		\pi^{-1}\left( (x_0, x_1) \right) &= \left\{ \left((x_0, x_1), [x_0,x_1] \right) \right\}
		\qquad \text{ if } (x_0, x_1) \neq (0,0)\\
		\pi^{-1}\left( (x_0, x_1) \right) &= \left( (0,0) \right) \cross \mathbb{P}^{1} \cong \mathbb{P}^{1}
	.\end{align} 
\end{ex} 

\begin{defn}[Strict transform]
	Let $Y \subset X$ be a closed subset with $Y \cap U \neq \emptyset$,
	then the Zariski closure of $Y \cap U$ in $\widetilde{X}$ is called the
	\textbf{strict transform} $\widetilde{Y}$ of $Y$.
	This is the same as the blow-up of $Y$ at $(f_0, \ldots, f_r)$.
\end{defn}

\begin{ex}
	In the above example the strict transform is interesting for a curve $C$ in $\mathbb{A}^{2}$
	passing through $(0,0)$.
	Since $C$ is a curve, then $C = \mathbb{V}\left( g \right)$, for
	\begin{equation}
		g(x_0,x_1) = \sum_{i,j \geq 0}^{} a_{ij} x_0^i x_1^j = 
		a_{00} + a_{10} x_0 + a_{01} x_1 + h.o.t
	.\end{equation} 
	Let's assume that $a_{00} = 0$, i.e. $(0.0) \in C$, and $(a_{01}, a_{10}) \neq (0,0)$, so that
	\begin{equation}
	L := \mathbb{V}\left( a_{10} x_0 + a_{01} x_1 \right)
	\end{equation} 
	is a line, which we call the tangent line to $C$ at $(0,0)$.
	If $\left((x_0,x_1), [y_0,y_1]\right) \in \widetilde{D}$, then it satisfies
	\begin{equation}\label{eqn:strtrexcond}
	\begin{cases}
		a_{10}x_0 + a_{01}x_1 + a_{20}x_0^2 + \ldots = 0\\
		x_0y_1 - x_1y_0 = 0
	\end{cases} 
	.\end{equation} 
	But the solution set of \eqref{eqn:strtrexcond} is larger than $\widetilde{C}$, since
	it contains the whole $\pi^{-1} \left( (0,0) \right) \cong \mathbb{P}^{1}$.
	But $\widetilde{C}$ is irreducible, hence it cannot contain such additional component.

	To get the equations for $\widetilde{C}$, we work first on $\mathbb{A}^{2} \setminus \left\{ x_0x_1 = 0 \right\}$.
	There we can multiply the equation of $C$ by $\frac{y_0}{x_0}$, which is
	a regular function on $\mathbb{A}^{2} \setminus \left\{ x_0x_1 = 0 \right\} \cross \mathbb{P}^{1}$,
	and obtain
	\begin{equation}
	a_{10}y_0 + a_{01} \frac{y_0}{x_0} x_1 + a_{20} y_0 x_0 +
	a_{11} y_0 x_1 + \ldots
	.\end{equation} 
	Using $\frac{y_0}{x_0} = \frac{y_1}{x_1}$ this is equivalent to
	\begin{equation}
	\begin{cases}
		a_{10}y_0 + a_{01} y_1 + a_{20}y_0x_0 +
		a_{11}y_0x_1 + a_{02}y_1x_1 + \ldots = 0\\
		x_0y_1 - x_1y_0 = 0
	\end{cases} 
	.\end{equation} 
	This equation should hold on $\widetilde{C}$.
	For every $C \ni (x_0, x_1) \neq (0,0)$, the above equations give the
	single point $\left((x_0,x_1), [x_0,x_1]\right)$.
	Instead over $(0,0)$ we obtain the point $\left((0,0), [y_0,y_1]\right)$
	with $a_{10}y_0 + a_{01}y_1 = 0$, which is exactly the equation of the tangent line at $(0,0)$.
\end{ex} 

\begin{ex}[Blow-up of $\mathbb{A}^{2}$ at $(x_0,x_1)$]
	It is a subset
	\begin{equation}
	\widetilde{\mathbb{A}^{2}} \subset \mathbb{A}^{2} \cross \mathbb{P}^{1}
	,\end{equation} 
	whose points are 
	\begin{equation}
	\left( (x_0, x_1), [y_0,y_1] \right)\quad \text{ s.t. }\quad \mathrm{rk}\, 
	\begin{pmatrix}
		x_0 & x_1\\y_0 & y_1
	\end{pmatrix} = 1
	.\end{equation} 
	Moreover it comes endowed with two morphisms (since they are given by the
	restriction of the projection morphisms)
	\begin{itemize}
		\item $p: \widetilde{\mathbb{A}^{2}} \to \mathbb{P}^{1}$ that extends the map
			used to define the projective space $\mathbb{P}^{1}$
			\begin{align}
				\K^2\setminus\left\{ \textbf{0} \right\} &\to \mathbb{P}^{1} \\
				(y_0,y_1) &\mapsto [y_0,y_1]
			.\end{align} 
			Then the fibers of $p$ are isomorphic to the lines of $\mathbb{A}^{2}$
			through the origin via $\pi$.
		\item $\pi: \widetilde{\mathbb{A}^{2}} \to \mathbb{A}^{2}$, which is an isomorphism
			outside $(0,0) = \mathbf{0}$.
			In fact $\pi^{-1}(\left\{ \mathbf{0} \right\}) = \left\{ \mathbf{0} \right\} \cross \mathbb{P}^{1}$.
	\end{itemize}
	In particular we can say that the points of $\pi^{-1}\left( \left\{ \mathbf{0} \right\} \right)$ correspond
	to the directions of the lines through the origin $\mathbf{0}$.

	Looking at the strict transform, then we can identify $\pi^{-1}(\left\{ \mathbf{0} \right\})$
	with the set of tangent directions to $\mathbb{A}^{2}$ at the origin $\mathbf{0}$.
\end{ex} 

\begin{lem}
	The blow-up of an affine variety $X$ at $(f_0, \ldots, f_r)$
	only depends on the ideal generated by $(f_0, \ldots, f_r)$ in $\mathbb{K}[X]$.
\end{lem} 

\begin{defn}[Blow-up at an ideal]
	Given an affine variety $X$ and $\mathcal{I} \subset \mathbb{K}[X]$ an ideal,
	we define the blow-up of $X$ at $\mathcal{I}$ as the
	blow-up of $X$ at $(f_0, \ldots, f_r)$, for any $f_0, \ldots, f_r$ generators 
	of $\mathcal{I}$.

	In particular, for $Y \subset X$ a closed subset, we can define the blow-up of $X$
	along $Y$ as the blow-up of $X$ at $\mathbb{I}(Y)$.
\end{defn}

\begin{rem}[]
	The construction of the blow-up is local.
	In fact let $X$ be an arbitrary variety and $Y \subset X$ a closed subset.
	If $\left\{ U_i \right\}_{i \in I}$ is an affine open cover of $X$, 
	let us denote by $\widetilde{U}_i$ the blow-up of the affine variety $U_i$
	along $Y \cap U_i$.

	It is easy to check that we can glue together the $\widetilde{U}_i$ in 
	a natural way, to construct a variety $\widetilde{X}$, which we will call the
	blow-up of $X$ along $Y$.

	Actually we can blow-up a sheaf of ideals on $X$.
	In fact the construction of the blow-up of an ideal can be generalized to arbitrary varieties.
	In order to do so we need
	\begin{itemize}
		\item $I(U) \subset \mathcal{O}_{X} \left( U \right)$, for each open subset $U \subset X$,
		\item gluing conditions.
	\end{itemize}
\end{rem}
Let's specialize the above remark to the case of projective varieties.
\begin{defn}[Blow-up for projective varieties]
	Let $X$ be a projective variety, and let
	$f_0, \ldots, f_r$ homogeneous polynomials of degree $d$, not in $\mathbb{I}_p(X)$.
	Let $U := X \setminus \mathbb{V}_p\left( f_0, \ldots, f_r \right)$ and
	\begin{equation}
		\Gamma := \left\{ \left(x, [f_0(x), \ldots, f_r(x)]\right) \ \middle|\ 
		x \in U\right\} \subset X \cross \mathbb{P}^{r}
	.\end{equation} 
	Then we define $\widetilde{X} := \Gamma \cross \mathbb{P}^{r}$ to be the \textbf{blow-up}
	of $X$ at $(f_0, \ldots, f_r)$.
\end{defn}

\begin{rem}[]
	In particular, the blow-up of a projective variety is a projective variety.
\end{rem}

\begin{lem}\leavevmode\vspace{-.2\baselineskip}
	\begin{enumerate}
	\item $\widetilde{X} \subset \left\{ \left(x, [y_0, \ldots, y_r]\right) \ \middle|\ 
		y_i f_j(x) - y_j f_i(x) = 0, \text{ for all } i,j = 0, \ldots, r\right\}
		\subset X \cross \mathbb{P}^{r}$.
	\item The inverse image $E := \pi^{-1} \left( \mathbb{V}_p\left( f_0, \ldots, f_r \right) \right)$
		is of pure dimension $\dim X - 1$.
		$E$ is called the {\em exceptional hypersurface} of the blow-up.
	\end{enumerate}
\end{lem} 

\begin{rem}[]
	The equations $y_i f_j(x) = y_j f_i(x)$, in general, are not the only
	equations defining $\widetilde{X}$ inside $X \cross \mathbb{P}^{r}$.

	In case they actually are the only ones, then
	\begin{equation}
		E = \left( X \cap \mathbb{V}\left( f_0, \ldots, f_r \right) \right) \cross \mathbb{P}^{r}
	.\end{equation} 
	The dimensions are: $\dim E = \dim X -1$, whereas $X \cap \mathbb{V}\left( f_0, \ldots, f_r \right)$ might have
	dimension $> \dim X - r -1$, then the r.h.s. might have dimension $> \dim X -1$.
\end{rem}

\begin{defn}[Initial polynomial/ideal]
	Consider $0 \neq f \in \mathbb{K}\left[x_1, \ldots, x_n \right]$.
	\begin{itemize}
		\item We can decompose in homogeneous components $f = \sum_{i \in \N}^{} f_i$,
			for $f_i \in \mathbb{K}\left[x_1, \ldots, x_n \right]_i$
			homogeneous of degree $i$.
			The minimal $i \in \N$ s.t. $f_i \neq 0$ is called the \textbf{minimal degree},
			denoted with $\mathrm{mindeg}\, f$, of $f$.
			For $i = \mathrm{mindeg}\, f$, we call $f^{in}:= f_i$ the \textbf{initial polynomial}
			of $f$.
		\item For an ideal $(0) \neq \mathcal{I} \subset \mathbb{K}\left[x_1, \ldots, x_n \right]$,
			we define the \textbf{initial ideal} of $\mathcal{I}$ to be
			\begin{equation}
			\mathcal{I}^{in} :=
			\left( f^{in} \ \middle|\ f \in \mathcal{I} \right)
			,\end{equation} 
			the homogeneous ideal generated by the initial polynomials of the elements of $\mathcal{I}$.
	\end{itemize}
\end{defn}

\begin{rem}[]
	Notice that, given $\mathcal{I} = \left( g_1, \ldots, g_l \right)$, then it is not always
	true that $\mathcal{I}^{in} = \left( g_1^{in}, \ldots, g_l^{in} \right)$.
\end{rem}

\begin{rem}[]
	Let's now study the blow-up of an affine variety $X \subset \mathbb{A}^{n}$ at a point,
	wlog $\left( 0, \ldots, 0 \right)$.
	Let us denote by $\widetilde{\mathbb{A}^{n}}$ the blow up of $\mathbb{A}^{n}$
	at $\left( x_1, \ldots, x_n \right)$.
	We want to cover it with affine varieties:
	let's start covering $\mathbb{P}^{n-1}$ with the affine varieties
	$U_i := \left\{ \left[ y_1 , \ldots , y_n \right] \in \mathbb{P}^{n-1} \ \middle|\ y_i \neq 0 \right\}$.
	In particular, for $i = 1$, we have $y_1 \neq 0$.
	Then, by the condition on the rank we obtain $x_j = x_1 y_j$ for the elements in
	$\widetilde{\mathbb{A}^{n}}$, hence we can define a map $\varphi$, with which we can cover the blow-up
	\begin{align}
		\varphi: \mathbb{A}^{n} &\to \widetilde{\mathbb{A}^{n}} \subset \mathbb{A}^{n} \cross \mathbb{P}^{n-1}
		\supset \mathbb{A}^{n} \cross U_1\\
		\big( x_1, \underbrace{x_1y_2, \ldots, x_1y_n}_{\text{inhom. coord. on }U_1} \big) &\mapsto 
		\left(\left( x_1, x_1y_2, \ldots, x_1y_n \right), \left[ 1, y_2 , \ldots , y_n \right]\right)
	,\end{align} 
	Then, by choosing the various affine charts of $\mathbb{P}^{n-1}$, we can cover
	the blow-up $\widetilde{\mathbb{A}^{n}}$ by affine spaces.

	Then the blow-up of any $X := \mathbb{V}\left( \mathcal{I} \right) \subset \mathbb{A}^{n}$ containing $\mathbf{0}$
	is defined, on this coordinate patch, by the vanishing of
	\begin{equation}
		\frac{f(x_1, x_1y_2, \ldots, x_1y_n)}{x_1^{\mathrm{mindeg}\, f}} \qquad
		\text{ for all } f \in \mathcal{I}, f \neq 0
	.\end{equation} 
	The exceptional hypersurface $E \subset \widetilde{X}$ is
	\begin{equation}
	\left\{ \mathbf{0} \right\} \cross \mathbb{V}_p\left( \mathcal{I}^{in} \right)
	\subset \left\{ \mathbf{0} \right\} \cross \mathbb{P}^{n-1}
	.\end{equation} 
	The geometric interpretation of $E$ is similar to taking the lower terms of
	the Taylor expansion of a function.
	In fact $\mathbb{I}(X)^{in}$ approximates $\mathbb{I}(X)$ by keeping only the terms of smallest degree.
	Since $\mathbb{I}^{in}$ is a homogeneous ideal, then
	$\hat{Y} := \mathbb{V}\left( \mathbb{I}(X)^{in} \right) \subset \mathbb{A}^{n}$ is
	a cone with vertex in the origin.
	By construction, $\hat{Y}$ is the cone (of the same dimension as $X$) that approximates $X$ best around $\mathbf{0}$.
	In fact
	\begin{equation}
		\dim \hat{Y} = \dim \underbrace{\mathbb{V}_p\left( \mathbb{I}(X)^{in} \right)}_{E} + 1 =
		\dim X - 1 + 1 = \dim X
	.\end{equation} 
\end{rem}

\begin{defn}[Tangent cone]
	Let $p \in X \subset \mathbb{A}^{n}$ a point of the Zariski closed subset $X$
	(wlog $p = \left( 0, \ldots, 0 \right) \in \mathbb{A}^{n}$).
	Then the tangent cone to $X$ at the point $p$ is
	\begin{equation}
		C_{X,p} := \mathbb{V}\left( \mathbb{I}(X)^{in} \right) \subset \mathbb{A}^{n}
	.\end{equation} 
\end{defn}
\begin{rem}[]
	Then the exceptional hypersurface $E \subset \widetilde{X}$ is the projectivization of this cone
	and corresponds to the limit tangent directions to $X$ at $p$.
\end{rem}

\begin{ex}\leavevmode\vspace{-.2\baselineskip}
\begin{enumerate}
	\item Let $X:= \mathbb{V}\left( y - x^2 + x \right) \subset \mathbb{A}^{2}$.
		Let $f(x,y) = y-x^x + x$, then $\mathrm{mindeg}\, f = 1$.
		Then $C_{X,0} = \mathbb{V}\left( y + x \right) \subset \mathbb{A}^{2}$ is the tangent line to $X$
		at the origin $\mathbf{0}$.
	\item Let $X:= \mathbb{V}\left( x^2 + x^3 -y^2 \right) \subset \mathbb{A}^{2}$.
		Let $f(x,y) = x^2 + x^3 - y^2$, then $\mathrm{mindeg}\, f = 2$.
		Then $C_{X,0} = \mathbb{V}\left( x^2 - y^2 \right) = \mathbb{V}\left( x - y \right) \cup
		\mathbb{V}\left( x + y \right) \subset \mathbb{A}^{2}$ is the tangent line to $X$
		at the origin $\mathbf{0}$.
		If we blow up $X$ at $\mathbf{0}$, then $\widetilde{X}$ separates the two branches 
		of $X$ at the origin $\mathbf{0}$.
	\item Let $X:= \mathbb{V}\left( x^3 -y^2 \right) \subset \mathbb{A}^{2}$.
		Let $f(x,y) = x^3 - y^2$, then $\mathrm{mindeg}\, f = 2$.
		Then $C_{X,0} = \mathbb{V}\left( y^2 \right) = \mathbb{V}\left( y \right)
		\subset \mathbb{A}^{2}$ is the tangent line to $X$
		at the origin $\mathbf{0}$.
		But this line comes with some nontrivial multiplicity.
		In fact, if we blow-up the origin $\widetilde{X}$ is tangent
		to $\left\{ \mathbf{0} \right\} \cross \mathbb{P}^{1} = E \subset \widetilde{\mathbb{A}^{2}}$
		at its point corresponding to the line $y = 0$.
\end{enumerate}
\end{ex} 

\begin{rem}
	Given a rational map $f: X \dashrightarrow \mathbb{P}^{r}$, given by
	$x \mapsto \left[ f_0(x) , \ldots , f_f(x) \right]$ for $r+1$ polynomials of the same degree
	(not all of them in the homogeneous ideal of $X$), then
	$f$ is a well defined morphism only on an open subset
	$U := X \setminus \mathbb{V}\left( f_0, \ldots, f_r \right)$.
	Then we have a closed subset on which $f$ is not determined: $\mathbb{V}\left( f_0, \ldots, f_r \right)$.
	We will call it the \textit{indeterminacy locus} of $f$.

	However, we can always extend $f$ to a morphism $\tilde{f}: \widetilde{X} \to \mathbb{P}^{r}$, 
	where $\widetilde{X}$ is a the blow-up of $X$ at $\left( f_0, \ldots, f_r \right)$, and
	$f$ is defined by the composition $\widetilde{X} \hookrightarrow X \cross \mathbb{P}^{r} \xrightarrow{\pi} \mathbb{P}^{r}$.
	If we blow-up the indeterminacy locus, we can always extend $f: X \dashrightarrow \mathbb{P}^{r}$
	to a projective variety birational to $X$.
	
	If, moreover, we start from a birational map $f: X \dashrightarrow Y$ this gives rise to interesting situations.
	Let's look at how to extend such a map to isomorphisms $\widetilde{X} \to \widetilde{Y}$.
\end{rem} 

\begin{ex}[Cremona transformation]
	It is a birational map
	\begin{align}
		f: \mathbb{P}^{2} &\dashrightarrow \mathbb{P}^{2} \\
		\left[ x_0, x_1 , x_2 \right] &\mapsto [ x_1x_2, x_0x_2, x_0x_1]
	.\end{align} 
	Its indeterminacy locus is $\mathbb{V}_p\left( x_1x_2, x_0x_2, x_0x_1 \right) = \left\{ 
	p_0 := [1,0,0], p_1 := [0,1,0], p_3 := [0,0,1] \right\}$.
	Since there are no relations in $S(\mathbb{P}^{2}) = \mathbb{K}\left[x_0, x_1, x_2 \right]$,
	we cannot extend $f$ to a morphism outside $U:= \mathbb{P}^{2} \setminus \left\{ p_0, p_1, p_2 \right\}$.

	Let's show that $f$ is a birational map: in such case there are $U \subset \mathbb{P}^{2} \supset V$
	open subsets s.t. $\left.f\right|_{U}: U \to V$ is an isomorphism.
	In fact, on $V := \mathbb{V}_p\left( x_0x_1x_2 \right)$, we can divide by $x_0x_1x_2$, and obtain that
	\begin{equation}
		f([x_0, x_1, x_2]) = [x_1x_2, x_0x_2, x_0x_1] = 
		[ \frac{1}{x_0}, \frac{1}{x_1}, \frac{1}{x_2}]
	.\end{equation} 
	Clearly, when restricted to $V$, $f \circ f = id_V$, hence $\left.f\right|_{U}: V \to V$
	is an isomorphism (f is an involution of $V$).
	Moreover, denoting by $L_i := \mathbb{V}_p\left( x_i \right)$, we can notice that
	$f(L_i) = p_i$, where $p_i$ are the points defined above.
	In fact $f(L_0) = f([0,x_1,x_2]) = [x_1x_2, 0, 0] = p_0$.

	Finally we can extend $f$ to a morphism by using blow-ups.
	Let's take $\widetilde{\mathbb{P}^{2}}$ the blow-up of $\mathbb{P}^{2}$ at
	$(x_1x_2, x_0x_2, x_0x_1)$, i.e. the blow-up at $p_0, p_1, p_2$.
	Then we can extend $f$ to $\tilde{f}: \widetilde{\mathbb{P}^{2}} \to \widetilde{\mathbb{P}^{2}}$.
	In fact $\widetilde{\mathbb{P}^{2}} \subset \mathbb{P}^{2} \cross \mathbb{P}^{2}$, and
	\begin{equation}
		\left( [x], [u] \right) \in \widetilde{\mathbb{P}^{2}} \iff
		\mathrm{rk}
		\begin{pmatrix}
			x_1x_2 & x_0x_2 & x_0x_1\\
			u_0 & u_1 & u_2
		\end{pmatrix} = 1
	.\end{equation} 
	Notice that this is equivalent (in general), to asking that $u_0, u_1, u_2$ is proportional to $1/x_0, 1/x_1, 1/x_2$,
	hence the condition is symmetric in $[x]$ and $[u]$.
	The we can define
	\begin{equation}
		\tilde{f} \left( [x_0, x_1, x_2], [u_0, u_1, u_2] \right) =
		\left( [u_0, u_1, u_2], [x_0, x_1, x_2] \right)
	,\end{equation} 
	and $[u_0, u_1, u_2] = f([x_0, x_1, x_2])$, where defined.
	Then this is clearly an extension of $f$ and also clearly an isomorphism.
	We only need to check that its image is $\widetilde{\mathbb{P}^{2}}$.
	In order to do so we only need to check that the image of the three exceptional divisors
	is contained in $\widetilde{\mathbb{P}^{2}}$.
	This is true, since $f$ is a birational inverse to itself.

	More explicitly, in $\widetilde{\mathbb{P}^{2}}$ the points $p_i$ get mapped to the exceptional lines $E_i$, and $L_i$
	get mapped to their strict transform $\widetilde{L_i}$.
	Then $\tilde{f}$ sends $\widetilde{L_i}$ to $E_i$ and vice-versa.
\end{ex} 

\begin{ex}
	Another example of a couple of birational projective varieties is given by
	\begin{equation}
	f: \mathbb{P}^{2} \dashrightarrow \mathbb{P}^{1} \cross \mathbb{P}^{1}
	,\end{equation} 
	since they contain the isomorphic open subsets $\mathbb{A}^{2} \cong \mathbb{A}^{1} \cross \mathbb{A}^{1}$.
	In particular we can explicitly write $f$ as, in the affine chart $U_0 \subset \mathbb{P}^{2}$
	\begin{equation}
	[1,x,y] \xrightarrow{f} s_{1,1} \left( [1,x], [1,y] \right) = [1, x, y, xy]
	.\end{equation} 
	In homogeneous coordinates (homogenizing everything with respect to $x_0$) it becomes
	\begin{equation}
		f \left( [x_0, x_1, x_2] \right) = [x_0^2, x_0x_1, x_0x_2, x_1x_2]
	.\end{equation} 
	Its indeterminacy locus is exactly
	\begin{equation}
	\mathbb{V}_p\left( x_0^2, x_0x_1, x_0x_2, x_1x_2 \right) =
	\mathbb{V}_p\left( x_0, x_1x_2 \right) = 
	\left\{ q_1 := [0,1,0], q_2 := [0,0,1] \right\}
	.\end{equation} 
	To extend $f$ to a morphism (to resolve $f$) we need to blow-up $\mathbb{P}^{2}$ at
	$\mathcal{I} = \left( x_0^2, x_0x_1, x_0x_2, x_1x_2 \right) \subsetneq \mathbb{I}_p(\left\{ q_1, q_2 \right\})$
	(notice that they have the same zero locus: they are related by saturations).

	Let $\widetilde{\mathbb{P}^{2}} \subset \mathbb{P}^{2} \cross \mathbb{P}^{3}$ be the blow-up at $\mathcal{I}$.
	Then $\widetilde{\mathbb{P}^{2}}$ is isomorphic to the blow-up
	of $\mathbb{P}^{2}$ at $q_1$ and $q_2$.
	In $U_1 = \left\{ x_1 \neq 0 \right\} \ni q_1$, we can take $x_1 = 1$ and then $\mathcal{I}$
	restricts to 
	\begin{equation}
	\left( x_0^2, x_0, x_0x_2, x_2 \right) = \left( x_0, x_2 \right) =
	\mathbb{I}(q_2) \subset \mathbb{K}\left[x_0, x_2 \right] = \mathbb{K}[U_1]
	.\end{equation} 
	andalogously, on $U_2$, $\mathcal{I}$ restricts to 
	\begin{equation}
		\left( x_0^2, x_0x_1, x_0, x_1 \right) = \left( x_0, x_1 \right) =
		\mathbb{I}(q_1) \subset \mathbb{K}\left[x_0, x_1 \right] = \mathbb{K}[U_2]
	.\end{equation} 
	The birational inverse of $f$ is
	\begin{align}
		g: \mathbb{P}^{1} \cross \mathbb{P}^{1} &\to \mathbb{P}^{2} \\
		[u_0, u_1, u_2, u_3] &\mapsto [u_0, u_1, u_2]
	.\end{align} 
	This is the projection from $p = [0,0,0,1] = s_{1,1} \left( [0,1],[0,1] \right)$
	on $\mathbb{P}^{1} \cross \mathbb{P}^{1}$.
	Hence $p$ is the indeterminacy locus and we need
	to blow it up, to obtain
	\begin{equation}
	\widetilde{\mathbb{P}^{1} \cross \mathbb{P}^{1}} \subset \mathbb{P}^{1} \cross \mathbb{P}^{1} \cross \mathbb{P}^{2}
	\subset \mathbb{P}^{3} \cross \mathbb{P}^{2}
	.\end{equation} 
	Now we want to define an isomorphism between $\widetilde{\mathbb{P}^{2}}$ and $\widetilde{\mathbb{P}^{1} \cross \mathbb{P}^{1}}$.
	\begin{align}
		\tilde{f}: \widetilde{\mathbb{P}^{2}} &\to \widetilde{\mathbb{P}^{1} \cross \mathbb{P}^{1}} \\
		\left( [x_0, x_1, x_2], [y_0, y_1, y_2, y_3] \right) &\mapsto 
		\left( [y_0,y_1, y_2, y_3], [x_0, x_1, x_2] \right)
	,\end{align} 
	where the points $\left( [x_0, x_1, x_2], [y_0, y_1, y_2, y_3] \right) \in \widetilde{\mathbb{P}^{2}}$
	have to satisfy
	\begin{equation}
	1 = \mathrm{rk}\, 
	\begin{pmatrix}
		y_0 & y_1 & y_2 & y_3\\
		x_0^2 & x_0x_1 & x_0x_2 & x_1x_2
	\end{pmatrix}
	\Rightarrow \mathrm{rk}\, 
	\begin{pmatrix}
		y_0 & y_1 & y_2\\
		x_0^2 & x_0x_1 & x_0x_2
	\end{pmatrix} = 1
	\ \ \text{ for } x_0 \neq 0
	.\end{equation} 
	This is the restriction to $\widetilde{\mathbb{P}^{2}}$ of an isomorphism between
	$\mathbb{P}^{2} \cross \mathbb{P}^{3}$ and $\mathbb{P}^{3} \cross \mathbb{P}^{2}$.
	Hence we only have to check that
	\begin{equation}
		\tilde{f} \left( \widetilde{\mathbb{P}^{2}} \right) \subset \widetilde{\mathbb{P}^{1} \cross \mathbb{P}^{1}}
		\qquad \text{ and } \qquad
		\tilde{f}^{-1} \left( \widetilde{\mathbb{P}^{1} \cross \mathbb{P}^{1}} \right) \subset
		\widetilde{\mathbb{P}^{2}}
	.\end{equation} 
	Since $\widetilde{\mathbb{P}^{2}}$ is an irreducible projective variety, it is sufficient 
	to show the above inclusions for dense open subsets of the required varieties.
	Take $\mathbb{A}^{2} \cong \mathbb{P}^{2} \setminus \mathbb{V}_p\left( x_0 \right) \subset \widetilde{\mathbb{P}^{2}}$,
	where $\mathbb{V}_p\left( x_0 \right)$ is the line passing trhough $q_1$ and $q_2$.
	We know that
	\begin{equation}
		\tilde{f} \left(  \mathbb{A}^{2} \right) = f \left( \mathbb{A}^{2} \right) =
		\mathbb{A}^{1} \cross \mathbb{A}^{1} \subset
		\mathbb{P}^{1} \cross \mathbb{P}^{1} \setminus \mathbb{V}_p\left( x_0y_0 \right)
		\hookrightarrow \widetilde{\mathbb{P}^{1} \cross \mathbb{P}^{1}}
	.\end{equation} 
	The result for $\tilde{f}^{-1}$ follows analogously.
\end{ex} 

\begin{lem}
	$\mathbb{P}^{1} \cross \mathbb{P}^{1}$ blown-up at one point is isomorphic to the blow-up 
	of $\mathbb{P}^{2}$ at two points.
\end{lem} 
\begin{proof}
	The proof is a generalzation of the above example.
\end{proof}
